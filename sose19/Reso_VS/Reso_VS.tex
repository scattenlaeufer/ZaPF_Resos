\documentclass[DIV=calc]{scrartcl}
\usepackage[utf8]{inputenc}
\usepackage[T1]{fontenc}
\usepackage[ngerman]{babel}
\usepackage{graphicx}
\usepackage[draft, markup=underlined]{changes}
\usepackage{csquotes}

\usepackage{ulem}
%\usepackage[dvipsnames]{xcolor}
\usepackage{paralist}
\usepackage{fixltx2e}
%\usepackage{ellipsis}
\usepackage[tracking=true]{microtype}

\usepackage{lmodern}                        % Ersatz fuer Computer Modern-Schriften
%\usepackage{hfoldsty}

%\usepackage{fourier}             % Schriftart
\usepackage[scaled=0.81]{helvet}     % Schriftart

\usepackage{url}
\usepackage{tocloft}             % Paket für Table of Contents

\usepackage{xcolor}
\definecolor{urlred}{HTML}{660000}

\usepackage{hyperref}
\hypersetup{
    colorlinks=true,
    linkcolor=black,    % Farbe der internen Links (u.a. Table of Contents)
    urlcolor=black,    % Farbe der url-links
    citecolor=black} % Farbe der Literaturverzeichnis-Links

\usepackage{mdwlist}     % Änderung der Zeilenabstände bei itemize und enumerate
\usepackage{draftwatermark} % Wasserzeichen ``Entwurf''
\SetWatermarkText{}

\parindent 0pt                 % Absatzeinrücken verhindern
\parskip 12pt                 % Absätze durch Lücke trennen

\setlength{\textheight}{23cm}
\usepackage{fancyhdr}
\pagestyle{fancy}
\fancyhead{} % clear all header fields
\cfoot{}
\lfoot{Zusammenkunft aller Physik-Fachschaften}
\rfoot{www.zapfev.de\\stapf@zapf.in}
\renewcommand{\headrulewidth}{0pt}
\renewcommand{\footrulewidth}{0.1pt}
\newcommand{\gen}{*innen}
\addto{\captionsngerman}{\renewcommand{\refname}{Quellen}}

%%%% Mit-TeXen Kommandoset
\usepackage[normalem]{ulem}
\usepackage{xcolor}

\newcommand{\replace}[2]{
    \sout{\textcolor{blue}{#1}}~\textcolor{blue}{#2}
}

\newcommand{\delete}[1]{
    \sout{\textcolor{red}{#1}}
}

\newcommand{\add}[1]{
    \textcolor{green}{#1}
}



\begin{document}
    \hspace{0.87\textwidth}
    \begin{minipage}{120pt}
        \vspace{-1.8cm}
        \includegraphics[width=80pt]{../../logo.pdf}
        \centering
        \small Zusammenkunft aller Physik-Fachschaften
    \end{minipage}
    \begin{center}
        \huge{Resolution der Zusammenkunft aller Physik-Fachschaften}\vspace{.25\baselineskip}\\
        \normalsize
    \end{center}
    \vspace{1cm}

\section*{Verfasste Studierendenschaften}
Aufgrund jüngster Hetze gegen und Forderungen zur Abschaffung von Vwerfassten Studierendenschaften\footnote{u.a. Drucksache 7/3844 des Landtages von Sachsen Anhalt (https://www.landtag.sachsen-anhalt.de/fileadmin/files/drs/wp7/drs/d3844aan.pdf), Zwangsmitgliedschaft in der Studierendenschaft abschaffen”der Jungen Liberalen NRW (https://julis-nrw.de/beschlusssammlung/zwangsmitgliedschaft-in-der-studierendenschaft-abschaffen/), Grundsatzprogramm der Jungen Union von 2012} wollen wir unseren Standpunkt aus dem Wintersemester 2009/2010 bekräftigen und erneuern.
Die Zusammenkunft aller Physikfachschaften (ZaPF) unterstützt Verfasste Studierendenschaften vorbe-
haltlos und fordern weiterhin die Wiedereinführung der verfassten Studierendenschaft in Bayern. Bestre-
bungen die Rechte verfasster Studierendenschaften zu beschneiden oder sie gar abzuschaffen lehnen wir
ab.\\
Zur Vertretung und Wahrung der Rechte und Interessen aller Studierenden sind frei gewählte Vertretungen
Verfasster Studierendenschaften notwendig. Zur Ausübung dieser Funktion sollen sie insbesondere mit
folgenden Rechten ausgestattet sein:\\
\begin{itemize}
\item Sich selbst eine Satzung zu geben,
\item Beiträge zu erheben und ihre Finanzen selbst zu verwalten,
\item sowie sich politisch zu äußern.
\end{itemize}
Darüber hinaus stellt die verfasste Studierendenschaft eine Solidargemeinschaft dar. Eine Möglichkeit sich
aus dieser zu lösen, wie sie in zwei Bundesländern traurigerweise existiert, verurteilen wir und fordern
weiterhin eine verpflichtende Mitgliedschaft.\\


\vspace*{\fill}
\begin{flushright}
Verabschiedet am 11.06.2019 in Bonn
\end{flushright}

\end{document}
