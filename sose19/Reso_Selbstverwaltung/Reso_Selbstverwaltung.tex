\documentclass[DIV=calc]{scrartcl}
\usepackage[utf8]{inputenc}
\usepackage[T1]{fontenc}
\usepackage[ngerman]{babel}
\usepackage{graphicx}
\usepackage[draft, markup=underlined]{changes}
\usepackage{csquotes}

\usepackage{ulem}
%\usepackage[dvipsnames]{xcolor}
\usepackage{paralist}
\usepackage{fixltx2e}
%\usepackage{ellipsis}
\usepackage[tracking=true]{microtype}

\usepackage{lmodern}                        % Ersatz fuer Computer Modern-Schriften
%\usepackage{hfoldsty}

%\usepackage{fourier}             % Schriftart
\usepackage[scaled=0.81]{helvet}     % Schriftart

\usepackage{url}
\usepackage{tocloft}             % Paket für Table of Contents

\usepackage{xcolor}
\definecolor{urlred}{HTML}{660000}

\usepackage{hyperref}
\hypersetup{
    colorlinks=true,
    linkcolor=black,    % Farbe der internen Links (u.a. Table of Contents)
    urlcolor=black,    % Farbe der url-links
    citecolor=black} % Farbe der Literaturverzeichnis-Links

\usepackage{mdwlist}     % Änderung der Zeilenabstände bei itemize und enumerate
\usepackage{draftwatermark} % Wasserzeichen ``Entwurf''
\SetWatermarkText{}

\parindent 0pt                 % Absatzeinrücken verhindern
\parskip 12pt                 % Absätze durch Lücke trennen

\setlength{\textheight}{23cm}
\usepackage{fancyhdr}
\pagestyle{fancy}
\fancyhead{} % clear all header fields
\cfoot{}
\lfoot{Zusammenkunft aller Physik-Fachschaften}
\rfoot{www.zapfev.de\\stapf@zapf.in}
\renewcommand{\headrulewidth}{0pt}
\renewcommand{\footrulewidth}{0.1pt}
\newcommand{\gen}{*innen}
\addto{\captionsngerman}{\renewcommand{\refname}{Quellen}}

%%%% Mit-TeXen Kommandoset
\usepackage[normalem]{ulem}
\usepackage{xcolor}

\newcommand{\replace}[2]{
    \sout{\textcolor{blue}{#1}}~\textcolor{blue}{#2}
}

\newcommand{\delete}[1]{
    \sout{\textcolor{red}{#1}}
}

\newcommand{\add}[1]{
    \textcolor{green}{#1}
}



\begin{document}
    \hspace{0.87\textwidth}
    \begin{minipage}{120pt}
        \vspace{-1.8cm}
        \includegraphics[width=80pt]{../../logo.pdf}
        \centering
        \small Zusammenkunft aller Physik-Fachschaften
    \end{minipage}
    \begin{center}
        \huge{Resolution der Zusammenkunft aller Physik-Fachschaften}\vspace{.25\baselineskip}\\
        \normalsize
    \end{center}
    \vspace{1cm}

\section*{Universitäte Selbstverwaltung}

Die ZaPF widerspricht allen Bestrebungen zum Rückbau universitärer Demokratie und einer Rückkehr zu den Zuständen der Ordinarienuniversität, wie sie vor 1968 existierte. \\

Die Universität ist nicht nur der Arbeitsplatz von Professuren, sondern ein Ort an dem viele verschiedene Menschen lehren, lernen und arbeiten. Die idealen Bedingungen dafür können nur durch Teilhabe und die Vertretung der Interessen aller her- und sichergestellt werden. Dafür müssen alle Statusgruppen angemessen in allen Bereichen, insbesondere allen relevanten Räten und Senaten, vertreten sein. \\

Da die Universität ein Abbild der gesamten Gesellschaft darstellen sollte, muss auch benachteiligten Gruppen der Gesellschaft der Zugang zu Universitäten und universitärer Bildung ermöglicht werden. Frauen- und gleichstellungsbeauftragte Personen haben sich dafür als bewährtes Mittel erwiesen. Ihre Teilnahme an oben genannten universitären Gremien ist daher unerlässlich.

\vspace*{\fill}
\begin{flushright}
Verabschiedet am 11.06.2019 in Bonn
\end{flushright}

\end{document}
