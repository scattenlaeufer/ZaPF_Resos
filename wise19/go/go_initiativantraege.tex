\documentclass[draft,10pt,oneside]{scrartcl}

% Sprache und Encodings
\usepackage[ngerman]{babel}

% Typographisch interessante Pakete
\usepackage{microtype} % Randkorrektur und andere Anpassungen

% References to Internet and within the document
\usepackage[pdftex,colorlinks=false,
pdftitle={Antrag zur Änderung der Geschäftsordnung für Plenen der ZaPF},
pdfauthor={Jörg Behrmann (FUB), Björn Guth (RWTH)},
pdfcreator={pdflatex},
pdfdisplaydoctitle=true]{hyperref}

% Absaetze nicht Einruecken
\setlength{\parindent}{0pt}
\setlength{\parskip}{2pt}

% Formatierung auf A4 anpassen
\usepackage{geometry}
\geometry{paper=a4paper,left=15mm,right=15mm,top=10mm,bottom=10mm}

\begin{document}

\section*{Antrag zur Änderung der Geschäftsordnung für Plenen der ZaPF}

\textbf{Antragsteller:} Jörg Behrmann (FUB), Björn Guth (RWTH)

\subsection*{Antrag}

Hiermit beantragen wir die Geschäftsordnung für Plenen der ZaPF wie folgend zu
ändern:

In 3.1 ersetze:
\begin{quote}
    \begin{enumerate}
        \setcounter{enumi}{2}
        \item Anträge, die nach dieser Frist eingereicht werden, sind
            Initiativanträge und müssen von mindestens zwei Personen aus
            verschiedenen Fachschaften getragen werden. Auch diese Anträge
            müssen dem Plenum in geeigneter Form vorgelegt werden.
    \end{enumerate}
\end{quote}
durch
\begin{quote}
    \begin{enumerate}
        \setcounter{enumi}{2}
        \item Anträge, die nach dieser Frist eingereicht werden, sind
            Initiativanträge und müssen von mindestens zwei Personen aus
            verschiedenen Fachschaften getragen werden.  Auch diese Anträge
            müssen dem Plenum in geeigneter Form vorgelegt werden.
            Initiativanträge werden am Ende der Liste der Anträge im
            Tagesordnungspunkt ``Anträge'' angehängt, so sie nicht mit einem
            anderen Antrag konkurrieren.  Sie früher zu behandeln bedarf eines
            Geschäftsordnungsantrages zur Änderung der Tagesordnung.
    \end{enumerate}
\end{quote}
Außerdem ergänze 3.1.7 durch
\begin{quote}
    Sie werden gleichzeitig behandelt.
\end{quote}

\end{document}
