\documentclass[DIV=calc]{scrartcl}
\usepackage[utf8]{inputenc}
\usepackage[T1]{fontenc}
\usepackage[ngerman]{babel}
\usepackage{graphicx}

\usepackage{fixltx2e}
\usepackage{ellipsis}
\usepackage[tracking=true]{microtype}

\usepackage{lmodern}                        % Ersatz fuer Computer Modern-Schriften
\usepackage{hfoldsty}

\usepackage{fourier}             % Schriftart
\usepackage[scaled=0.81]{helvet}     % Schriftart

\usepackage{url}
\usepackage{tocloft}             % Paket für Table of Contents

\usepackage{xcolor}
\definecolor{urlred}{HTML}{660000}

\usepackage{hyperref}
\hypersetup{
    colorlinks=true,    
    linkcolor=black,    % Farbe der internen Links (u.a. Table of Contents)
    urlcolor=black,    % Farbe der url-links
    citecolor=black} % Farbe der Literaturverzeichnis-Links

\usepackage{mdwlist}     % Änderung der Zeilenabstände bei itemize und enumerate

\parindent 0pt                 % Absatzeinrücken verhindern
\parskip 12pt                 % Absätze durch Lücke trennen

%\usepackage{titlesec}    % Abstand nach Überschriften neu definieren
%\titlespacing{\subsection}{0ex}{3ex}{-1ex}
%\titlespacing{\subsubsection}{0ex}{3ex}{-1ex}        

% \pagestyle{empty}
\setlength{\textheight}{23cm}
\usepackage{fancyhdr}
\pagestyle{fancy}
\cfoot{}
\lfoot{Zusammenkunft aller Physik-Fachschaften}
\rfoot{www.zapfev.de\\stapf@zapf.in}
\renewcommand{\headrulewidth}{0pt}
\renewcommand{\footrulewidth}{0.1pt}
\newcommand{\gen}{*innen}

\begin{document}
    \hspace{0.87\textwidth}
    \begin{minipage}{120pt}
        \vspace{-1.8cm}
        \includegraphics[width=80pt]{logo.pdf}
        \centering
        \small Zusammenkunft aller Physik-Fachschaften
    \end{minipage}
    \begin{center}
        \huge{Positionspapier zu Lernzielen für Grund- oder Anfängerpraktika der Physik} \\
        \normalsize
    \end{center}
    
    \vspace{1cm}    
\textbf{Die Zusammenkunft aller Physikfachschaften im deutschsprachigen Raum fordert die Vermittlung der unten aufgeführten Lernziele in Grund- oder Anfängerpraktika der Physik.}\\

\textbf{Zielsetzungen der Grund- oder Anfängerpraktika}\\

Praktika sind eine zentrale Lehrveranstaltungsform in naturwissenschaftlichen Fächern. Das Ziel dieser Grund- oder Anfängerpraktika ist die Vermittlung von spezifischen inhaltlichen sowie formellen Lernzielen und Schlüsselqualifikationen.\\
Diese Lernziele gelten dabei als zentrale Fähigkeiten von Absolvent\gen \ der Physik. Nach Grund- oder Anfängerpraktika in der Physik sollen die unten aufgeführten Lernziele vermittelt worden sein. Die Gestaltung und Vermittlung dieser Lernziele obliegt dabei der Universität.\\

\textbf{Lernziele für Grund- oder Anfängerpraktika in der Physik}\\

Um den Grundstein für das selbstständige wissenschaftliche Arbeiten zu legen, sollen Studierende im Grund- oder Anfängerpraktikum lernen, die Durchführung von Experimenten mit gegebener Aufgabenstellung eigenständig zu planen sowie Experimente korrekt aufzubauen. Bei der Durchführung soll der richtige Umgang mit den technischen Geräten vermittelt werden.\\

Während der Grund- oder Anfängerpraktika werden Studierende mit verschiedenen möglichen Gefahrensituationen konfrontiert. Der korrekte Umgang mit diesen Situationen stellt ein wichtiges Lernziel dar.
Dies beinhaltet auch die Vermeidung von Sicherheitsrisiken wie beispielsweise falscher Kleidung.\\

Für die Nachvollziehbarkeit eines Versuches sollen alle relevanten Informationen inklusive Messwerte in geeigneter Form, wie zum Beispiel in einem Laborbuch, festgehalten werden.\\
\\
Die Auswertung dieser Daten mit einem digitalen Fit-Programm sollte erlernt werden, wobei das Verständnis der verwendeten Methodik vorausgesetzt wird.\\
Nach der abgeschlossenen Auswertung sollen die Interpretation und Diskussion der Ergebnisse vermittelt werden, besonders im Hinblick auf Unsicherheiten und unter Berücksichtigung des physikalischen Kontextes. Außerdem sollen die Studierenden lernen, ihre im Grund- oder Anfängerpraktikum gewonnenen Ergebnisse schlüssig, bündig und übersichtlich auszuarbeiten und schriftlich darzustellen.\\
Dabei stellen die Abschätzung, Diskussion und der Einfluss von Fehlern auf die Ergebnisse einen zentralen Teil der eigentlichen Resultate dar. Hierbei soll sowohl die Messgenauigkeit abgeschätzt als auch ihr Einfluss durch eine Fehlerrechnung berücksichtigt werden. Die Herkunft der Fehler soll hierbei ebenfalls diskutiert und interpretiert werden.\\

Bei der Erstellung des Protokolls soll auf einen sensiblen Umgang mit Quellen inklusive deren korrektes Zitieren geachtet werden.\\
Beim Verfassen des Protokolls ist auf eine sorgfältige Formulierung und die korrekte äußere Form zu achten. Aufgrund der allgemeinen wissenschaftlichen Relevanz wird dabei dringlichst empfohlen, dass sich die Studierenden Grundkenntnisse in einem geeigneten Textsatzsystem (z.B. LaTeX) aneignen.\\

Ebenfalls ein zentraler Bestandteil der Grund- oder Anfängerpraktika ist der Transfer von theoretischem Wissen in die Praxis, sodass die Arbeit an Experimenten zu einem besseren Verständnis der zu Grunde liegenden Zusammenhänge und deren Vertiefung führt. So soll insbesondere der physikalische Erkenntnisgewinn am selbst durchgeführten Experiment erfahren werden, gerade auch zum Erlernen und Vertiefen einer Intuition für physikalische Zusammenhänge.\\
Außerdem sollen Absolvent\gen \ der Physik in der Lage sein, sowohl im Team als auch eigenständig organisiert zu arbeiten.\\

Bei Erfüllung der oben genannten Lernziele im Grund- oder Anfängerpraktikum ist der Grundstein für gutes wissenschaftliches Arbeiten gelegt.
    \vfill
    \begin{flushright}
        Verabschiedet am 28.05.2017 in Berlin
    \end{flushright}
    
\end{document}
%%% Local Variables:
%%% mode: latex
%%% TeX-master: t
%%% End:
