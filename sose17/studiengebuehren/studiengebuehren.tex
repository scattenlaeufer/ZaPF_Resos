\documentclass[DIV=calc]{scrartcl}
\usepackage[utf8]{inputenc}
\usepackage[T1]{fontenc}
\usepackage[ngerman]{babel}
\usepackage{graphicx}

\usepackage{fixltx2e}
\usepackage{ellipsis}
\usepackage[tracking=true]{microtype}

\usepackage{lmodern}                        % Ersatz fuer Computer Modern-Schriften
\usepackage{hfoldsty}

\usepackage{fourier}             % Schriftart
\usepackage[scaled=0.81]{helvet}     % Schriftart

\usepackage{url}
\usepackage{tocloft}             % Paket für Table of Contents

\usepackage{xcolor}
\definecolor{urlred}{HTML}{660000}

\usepackage{hyperref}
\hypersetup{
    colorlinks=true,    
    linkcolor=black,    % Farbe der internen Links (u.a. Table of Contents)
    urlcolor=black,    % Farbe der url-links
    citecolor=black} % Farbe der Literaturverzeichnis-Links

\usepackage{mdwlist}     % Änderung der Zeilenabstände bei itemize und enumerate

\parindent 0pt                 % Absatzeinrücken verhindern
\parskip 12pt                 % Absätze durch Lücke trennen

%\usepackage{titlesec}    % Abstand nach Überschriften neu definieren
%\titlespacing{\subsection}{0ex}{3ex}{-1ex}
%\titlespacing{\subsubsection}{0ex}{3ex}{-1ex}        

% \pagestyle{empty}
\setlength{\textheight}{23cm}
\usepackage{fancyhdr}
\pagestyle{fancy}
\cfoot{}
\lfoot{Zusammenkunft aller Physik-Fachschaften}
\rfoot{www.zapfev.de\\stapf@zapf.in}
\renewcommand{\headrulewidth}{0pt}
\renewcommand{\footrulewidth}{0.1pt}
\newcommand{\gen}{*innen}

\begin{document}
    \hspace{0.87\textwidth}
    \begin{minipage}{120pt}
        \vspace{-1.8cm}
        \includegraphics[width=80pt]{logo.pdf}
        \centering
        \small Zusammenkunft aller Physik-Fachschaften
    \end{minipage}
    \begin{center}
        \huge{Resolution gegen Studiengebühren} \\
        \normalsize
    \end{center}
    \vspace{1cm}

Die ZaPF lehnt Studiengebühren jeglicher Art ab. Darüber hinaus positioniert sie sich gegen alle weiteren Bildungsbarrieren.

 Wir stehen gegen Studiengebühren, wie sie beispielsweise in Baden-Württemberg eingeführt und in anderen Bundesländern noch z.B. für das Zweitstudium vorgesehen sind. Dies wendet sich auch gegen jüngste Bestrebungen in Bundesländern wie Nordrhein-Westfalen.

 Gesellschafts-, sozial und bildungspolitische Gründe sprechen gegen Studiengebühren und wurden schon an zahllosen Orten ausgiebig diskutiert, z.B. im Krefelder Aufruf des Aktionsbündnisses gegen Studiengebühren (ABS) \footnote{http://www.abs-bund.de/aktionsbuendnis/krefelder-aufruf}: " [Studiengebühren] lösen kein einziges Problem [..]."

 Statt ständiger Umstrukturierung und Neuorientierung in einer kurzfristigen Bildungspolitik sind stabile Rahmenbedingungen 
 für ein frei zugängliches Bildungssystem essentiell.
    
    \vfill
    \begin{flushright}
        Verabschiedet am 27.05.2017 in Berlin
    \end{flushright}
    
\end{document}

%%% Local Variables:
%%% mode: latex
%%% TeX-master: t
%%% End:
