\documentclass[DIV=calc]{scrartcl}
\usepackage[utf8]{inputenc}
\usepackage[T1]{fontenc}
\usepackage[ngerman]{babel}
\usepackage{graphicx}

\usepackage{fixltx2e}
\usepackage{ellipsis}
\usepackage[tracking=true]{microtype}

\usepackage{lmodern}                        % Ersatz fuer Computer Modern-Schriften
\usepackage{hfoldsty}

\usepackage{fourier}             % Schriftart
\usepackage[scaled=0.81]{helvet}     % Schriftart

\usepackage{url}
\usepackage{tocloft}             % Paket für Table of Contents

\usepackage{xcolor}
\definecolor{urlred}{HTML}{660000}

\usepackage{hyperref}
\hypersetup{
    colorlinks=true,    
    linkcolor=black,    % Farbe der internen Links (u.a. Table of Contents)
    urlcolor=black,    % Farbe der url-links
    citecolor=black} % Farbe der Literaturverzeichnis-Links

\usepackage{mdwlist}     % Änderung der Zeilenabstände bei itemize und enumerate

\parindent 0pt                 % Absatzeinrücken verhindern
\parskip 12pt                 % Absätze durch Lücke trennen

%\usepackage{titlesec}    % Abstand nach Überschriften neu definieren
%\titlespacing{\subsection}{0ex}{3ex}{-1ex}
%\titlespacing{\subsubsection}{0ex}{3ex}{-1ex}        

% \pagestyle{empty}
\setlength{\textheight}{23cm}
\usepackage{fancyhdr}
\pagestyle{fancy}
\cfoot{}
\lfoot{Zusammenkunft aller Physik-Fachschaften}
\rfoot{www.zapfev.de\\stapf@zapf.in}
\renewcommand{\headrulewidth}{0pt}
\renewcommand{\footrulewidth}{0.1pt}
\newcommand{\gen}{*innen}

\begin{document}
    \hspace{0.87\textwidth}
    \begin{minipage}{120pt}
        \vspace{-1.8cm}
        \includegraphics[width=50pt]{logo.pdf}
        \centering
        \small Zusammenkunft aller Physik-Fachschaften
    \end{minipage}
    \begin{center}
%        \huge{Pressemitteilung zum Verhalten der HRK} \\
        \Huge{Hochschulrektorenkonferenz blockiert Zusammenarbeit}\\
        \large{Studentischer Vertreter für Akkreditierungsrat abgelehnt}
        \normalsize
    \end{center}
Die Hochschulrektorenkonferenz (HRK) lehnt den vom Poolvernetzungstreffen (PVT) entsandten studentischen Kandidaten für den Akkreditierungsrat ohne Begründung ab und bestimmt einen eigenen Kandidaten ohne Rücksprache mit dem PVT. Dieser ist nicht von der Statusgruppe der Studierenden demokratisch legitimiert. Mehrere Versuche einer Kontaktaufnahme seitens des studentischen Akkreditierungspools blieben ohne Erfolg.

Als pooltragende Organisation verurteilt die Zusammenkunft aller deutschsprachigen Physikfachschaften (ZaPF) das Verhalten der HRK bezüglich der Ablehnung des demokratisch gewählten studentischen Vertreters im Akkreditierungsrat, sowie die Benennung eines durch die HRK ausgewählten und nicht von der Statusgruppe der Studierenden legitimierten Vertreters. 

Daher fordert die ZaPF die HRK nun auf, die demokratisch, durch das Poolvernetzungstreffen gewählten Vertreter\gen\ zu benennen! \glqq Außerdem soll die HRK zu einer konstruktiven und kommunikativen Zusammenarbeit mit dem studentischen Akkreditierungspool zurückkehren\grqq , fordert Peter Steinmüller, Sprecher der ZaPF.

An deutschen Hochschulen ist die Akkreditierung von Studiengängen durch die \glqq Stiftung zur Akkreditierung von Studiengängen in Deutschland\grqq\ geregelt und betrifft alle dort vertretenen Statusgruppen\footnote{Hochschulvertreter, Ländervertreter, Vertreter der Berufspraxis, Studierende und Internationale Vertreter}.

Der Akkreditierungsrat ist das zentrale Beschlussgremium dieser Stiftung, in welchem die Statusgruppe der Studierenden durch zwei Mitglieder aus dem studentischen Akkreditierungspools vertreten ist. Dieser besteht aus potentiellen studentischen Gutachter\gen\ für Akkreditierungsverfahren. Die beiden Mitglieder werden durch das oberste, beschlussfassende Organ des studentischen Akkreditierungspools, dem PVT, gewählt und für den Akkreditierungsrat vorgeschlagen. Der letzte Vorschlag des PVT wurde jedoch ohne eine Begründung abgelehnt.

Die ZaPF vertritt seit 1980 die Studierenden des Faches Physik in Deutschland, Österreich und der Schweiz. Als pooltragende Organisation entsendet sie Vertreter\gen\ in den studentischen Akkreditierungspool und wirkt an seiner Entwicklung mit.

Ansprechpartner: Peter Steinmüller (stapf@zapf.in)\\
Anzahl Wörter: 262\\
Anzahl Zeichen: 2279
\vfill
\end{document}
%%% Local Variables:
%%% mode: latex
%%% TeX-master: t
%%% End:
