\documentclass[DIV=calc]{scrartcl}
\usepackage[utf8]{inputenc}
\usepackage[T1]{fontenc}
\usepackage[ngerman]{babel}
\usepackage{graphicx}

\usepackage{fixltx2e}
\usepackage{ellipsis}
\usepackage[tracking=true]{microtype}

\usepackage{lmodern}      % Ersatz fuer Computer Modern-Schriften
\usepackage{hfoldsty}

\usepackage{fourier} 			% Schriftart
\usepackage[scaled=0.81]{helvet} 	% Schriftart

\usepackage{url}
\usepackage{tocloft} 			% Paket für Table of Contents

\usepackage{xcolor}
\definecolor{urlred}{HTML}{660000}

\usepackage{hyperref}
\hypersetup{
 colorlinks=true,	
 linkcolor=black,	% Farbe der internen Links (u.a. Table of Contents)
 urlcolor=black,	% Farbe der url-links
 citecolor=black} % Farbe der Literaturverzeichnis-Links

\usepackage{mdwlist} 	% Änderung der Zeilenabstände bei itemize und enumerate

\parindent 0pt 				% Absatzeinrücken verhindern
\parskip 12pt 				% Absätze durch Lücke trennen

%\usepackage{titlesec}	% Abstand nach Überschriften neu definieren
%\titlespacing{\subsection}{0ex}{3ex}{-1ex}
%\titlespacing{\subsubsection}{0ex}{3ex}{-1ex}		

% \pagestyle{empty}
\setlength{\textheight}{23cm}
\usepackage{fancyhdr}
\pagestyle{fancy}
\cfoot{}
\lfoot{Zusammenkunft aller Physik-Fachschaften}
\rfoot{www.zapfev.de\\stapf@zapf.in}
\renewcommand{\headrulewidth}{0pt}
\renewcommand{\footrulewidth}{0.1pt}


\begin{document}
\hspace{0.87\textwidth}
\begin{minipage}{120pt}
\vspace{-1.8cm}
\includegraphics[width=80pt]{logo.pdf}
\centering
\small Zusammenkunft aller Physik-Fachschaften
\end{minipage}
\begin{center}
\huge{Resolution der Zusammenkunft aller Physik-Fachschaften} \\
\normalsize
\end{center}

\vspace{1cm}
\section*{Kritik am Besetzungsverhalten der HRK in den Akkreditierungsrat}

Die „Zusammenkunft aller deutschsprachigen Physikfachschaften“ verurteilt die Entscheidung der Hochschulrektorenkonferenz (HRK), den vom Poolvernetzungstreffen gewählten studentischen Vertreter nicht in den Akkreditierungsrat zu entsenden. \\

Der Akkreditierungsrat ist die höchste Instanz der Akkreditierung in Deutschland, in dem die Studierenden nach dem  \glqq Gesetz zur Errichtung einer Stiftung \glqq Stiftung zur Akkreditierung von Studiengängen in Deutschland\grqq (\S 7 (2)) eine vertretene Statusgruppe darstellen. Der studentische Akkreditierungspool wird getragen von den Landeszusammenschlüssen der studierendenschaften, den Bundesfachschaftentagungen und dem freien zusammenschluss von studentInnenschaften. Eine Wahl durch sein oberstes beschlussfassendes Organ (Poolvernetzungstreffen) stellt damit die hochstmögliche demokratische Legitimation von studentischen VertreterInnen im Akkreditierungsrat dar. \\

Vor diesem Hintergrund kritisiert die ZaPF das folgende Verhalten der HRK:

\begin{enumerate}
	\item Die HRK lehnte den Vorschlag des Poolvernetzungstreffens für einen studentischen Vertreter ab.
	\item Sie begründete ihre Entscheidung zur Ablehnung nicht.
	\item Sie fragte ohne Rücksprache mit dem studentischen Akkreditierungspool einen eigenen studentischen Kandidaten an, der somit nicht von der Statusgruppe demokratisch legitimiert ist.
	\item Sie reagiert nicht auf wiederholte Versuche einer Kontaktaufnahme seitens des Koordinierungsausschusses des studentischen Akkreditierungspools.

\end{enumerate}
 
In der Vergangenheit wurden bereits zwei weitere Vorschläge des Poolvernetzungstreffens für studentische Vertreter*innen abgelehnt. Auch wenn die HRK das Recht zur Benennung innehat, befinden wir das wiederholte Ablehnen von den durch ihre Statusgruppe gewählten Kandidat*innen als undemokratisch.\\

Die Punkte 2. bis 4. wirken erschwerend. Wir kritisieren, dass dadurch die Mitbestimmung aller Statusgruppen untergraben und ihre Zusammenarbeit massiv erschwert wird. \\

Die ZaPF fordert die HRK auf, die von Studierenden selbst auf demokratischem Wege bestimmte studentische VertreterInnen zu benennen und zu einer konstruktiven und kommunikativen Zusammenarbeit zurückzukehren.

\vfill
\begin{flushright}
Verabschiedet am 27.05.2017 in Berlin
\end{flushright}




\end{document}