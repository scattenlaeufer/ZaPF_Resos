\documentclass[DIV=calc]{scrartcl}
\usepackage[utf8]{inputenc}
\usepackage[T1]{fontenc}
\usepackage[ngerman]{babel}
\usepackage{graphicx}

\usepackage{fixltx2e}
\usepackage{ellipsis}
\usepackage[tracking=true]{microtype}

\usepackage{lmodern}                        % Ersatz fuer Computer Modern-Schriften
\usepackage{hfoldsty}

\usepackage{fourier}             % Schriftart
\usepackage[scaled=0.81]{helvet}     % Schriftart

\usepackage{url}
\usepackage{tocloft}             % Paket für Table of Contents

\usepackage{xcolor}
\definecolor{urlred}{HTML}{660000}

\usepackage{hyperref}
\hypersetup{
    colorlinks=true,    
    linkcolor=black,    % Farbe der internen Links (u.a. Table of Contents)
    urlcolor=black,    % Farbe der url-links
    citecolor=black} % Farbe der Literaturverzeichnis-Links

\usepackage{mdwlist}     % Änderung der Zeilenabstände bei itemize und enumerate

\parindent 0pt                 % Absatzeinrücken verhindern
\parskip 12pt                 % Absätze durch Lücke trennen

%\usepackage{titlesec}    % Abstand nach Überschriften neu definieren
%\titlespacing{\subsection}{0ex}{3ex}{-1ex}
%\titlespacing{\subsubsection}{0ex}{3ex}{-1ex}        

% \pagestyle{empty}
\setlength{\textheight}{23cm}
\usepackage{fancyhdr}
\pagestyle{fancy}
\cfoot{}
\lfoot{Zusammenkunft aller Physik-Fachschaften}
\rfoot{www.zapfev.de\\stapf@zapf.in}
\renewcommand{\headrulewidth}{0pt}
\renewcommand{\footrulewidth}{0.1pt}
\newcommand{\gen}{*innen}

\begin{document}
    \hspace{0.87\textwidth}
    \begin{minipage}{120pt}
        \vspace{-1.8cm}
        \includegraphics[width=80pt]{logo.pdf}
        \centering
        \small Zusammenkunft aller Physik-Fachschaften
    \end{minipage}
    \begin{center}
        \huge{Positionspapier zum aktuellen Diskurs über den Mathematikkenntnisstand der
Studienanfänger\gen} \\
        \normalsize
    \end{center}
    \vspace{1cm}    
Die ZaPF begrüßt die aktuellen Entwicklungen innerhalb des öffentlichen Diskurses über den Mathematikkenntnisstand der Studienanfänger\gen\ in den MINT-Fächern.\\
Explizit verweisen wir hierbei auf den offenen Brief "Mathematikunterricht und Kompetenzorientierung"\footnote{\href{https://zapf.wiki/Datei:Offener_Brief_Mathematikunterricht_Kompetenzorientierung.pdf}{https://zapf.wiki/Datei:Offener\_Brief\_Mathematikunterricht\_Kompetenzorientierung.pdf}} vom 17.3.2017 sowie die beiden darauf folgenden Stellungnahmen\footnote{\href{https://zapf.wiki/Datei:Mathematiker-distanzieren-sich-vom-mathematiker-brandbrief.pdf}{https://zapf.wiki/Datei:Mathematiker-distanzieren-sich-vom-mathematiker-brandbrief.pdf}}\footnote{\href{https://zapf.wiki/Datei:Stellungnahme_DMV_GDM_MNU_20.04.2017.pdf}{https://zapf.wiki/Datei:Stellungnahme\_DMV\_GDM\_MNU\_20.04.2017.pdf}} zu diesem Thema.\\
Insbesondere schließen wir uns der Stellungnahme der DMV, GDM und MNU\footnote{Deutsche Mathematiker-Vereinigung, Gesellschaft für Didaktik der Mathematik und Verband zur Förderung des MINT-Unterrichts} in allen Punkten bis auf den beiden folgenden an:
\begin{enumerate}
\item Zur Thematik "`Taschenrechner im Schulunterricht"' verweisen wir auf die Stellungnahme zu unserer Resolution aus Dresden\footnote{\href{https://zapf.wiki/images/7/70/Taschenrechner_WiSe16.pdf}{https://zapf.wiki/images/7/70/Taschenrechner\_WiSe16.pdf}}.
\item Darüber hinaus schließen wir uns der Forderung nach einer bundesweit verbindlichen schriftlichen Mathematikprüfung im Abitur nicht an.
\end{enumerate}

\vfill
    \begin{flushright}
        Verabschiedet am 28.05.2017 in Berlin
    \end{flushright}
\end{document}
%%% Local Variables:
%%% mode: latex
%%% TeX-master: t
%%% End:
