\documentclass[DIV=calc]{scrartcl}
\usepackage[utf8]{inputenc}
\usepackage[T1]{fontenc}
\usepackage[ngerman]{babel}
\usepackage{graphicx}

\usepackage{fixltx2e}
\usepackage{ellipsis}
\usepackage[tracking=true]{microtype}

\usepackage{lmodern}                        % Ersatz fuer Computer Modern-Schriften
\usepackage{hfoldsty}

\usepackage{fourier}             % Schriftart
\usepackage[scaled=0.81]{helvet}     % Schriftart

\usepackage{url}
\usepackage{tocloft}             % Paket für Table of Contents

\usepackage{xcolor}
\definecolor{urlred}{HTML}{660000}

\usepackage{hyperref}
\hypersetup{
    colorlinks=true,    
    linkcolor=black,    % Farbe der internen Links (u.a. Table of Contents)
    urlcolor=black,    % Farbe der url-links
    citecolor=black} % Farbe der Literaturverzeichnis-Links

\usepackage{mdwlist}     % Änderung der Zeilenabstände bei itemize und enumerate

\parindent 0pt                 % Absatzeinrücken verhindern
\parskip 12pt                 % Absätze durch Lücke trennen

%\usepackage{titlesec}    % Abstand nach Überschriften neu definieren
%\titlespacing{\subsection}{0ex}{3ex}{-1ex}
%\titlespacing{\subsubsection}{0ex}{3ex}{-1ex}        

% \pagestyle{empty}
\setlength{\textheight}{23cm}
\usepackage{fancyhdr}
\pagestyle{fancy}
\cfoot{}
\lfoot{Zusammenkunft aller Physik-Fachschaften}
\rfoot{www.zapfev.de\\stapf@zapf.in}
\renewcommand{\headrulewidth}{0pt}
\renewcommand{\footrulewidth}{0.1pt}
\newcommand{\gen}{*innen}

\begin{document}
    \hspace{0.87\textwidth}
    \begin{minipage}{120pt}
        \vspace{-1.8cm}
        \includegraphics[width=80pt]{logo.pdf}
        \centering
        \small Zusammenkunft aller Physik-Fachschaften
    \end{minipage}
    \begin{center}
        \huge{Resolution zur studentischen Beteiligung bei Bauvorhaben} \\
        \normalsize
    \end{center}
    
    \vspace{1cm}    
Die ZaPF fordert, dass Studierende von Beginn an ein fester Bestandteil von Planungskommissionen für Neu- und Umbauten auf dem Hochschulgelände sind.

Oft berücksichtigen die Bedarfserhebungen und Planungen die Bedürfnisse der Lehre und des studentischen und
kulturellen Lebens auf dem Campus nicht in ausreichendem Maße, obwohl die Studierenden die größte Nutzer*innengruppe darstellen. Hochschulöffentliche Räumlichkeiten müssen von allen Statusgruppen gemeinsam geplant werden. Hierzu gehören Aufenthalts-, Arbeits- und Erholungsräume, die auch für Studierende zugänglich sind. Ebenso gehören auch großzügige Foren und Flure, die für informelle Begegnungen und akademischen Veranstaltungen genutzt werden können, dazu.

Bei der Erarbeitung und dem Beschluss von Nutzungskonzepten müssen Studierende von Beginn an beteiligt werden.    
    \vfill
    \begin{flushright}
        Verabschiedet am 28.05.2017 in Berlin
    \end{flushright}

    
\end{document}

%%% Local Variables:
%%% mode: latex
%%% TeX-master: t
%%% End:
