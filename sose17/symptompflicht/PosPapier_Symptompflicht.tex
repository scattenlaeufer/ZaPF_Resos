\documentclass[DIV=calc]{scrartcl}
\usepackage[utf8]{inputenc}
\usepackage[T1]{fontenc}
\usepackage[ngerman]{babel}
\usepackage{graphicx}

\usepackage{fixltx2e}
\usepackage{ellipsis}
\usepackage[tracking=true]{microtype}

\usepackage{lmodern}                        % Ersatz fuer Computer Modern-Schriften
\usepackage{hfoldsty}

\usepackage{fourier}             % Schriftart
\usepackage[scaled=0.81]{helvet}     % Schriftart

\usepackage{url}
\usepackage{tocloft}             % Paket für Table of Contents

\usepackage{xcolor}
\definecolor{urlred}{HTML}{660000}

\usepackage{hyperref}
\hypersetup{
    colorlinks=true,    
    linkcolor=black,    % Farbe der internen Links (u.a. Table of Contents)
    urlcolor=black,    % Farbe der url-links
    citecolor=black} % Farbe der Literaturverzeichnis-Links

\usepackage{mdwlist}     % Änderung der Zeilenabstände bei itemize und enumerate

\parindent 0pt                 % Absatzeinrücken verhindern
\parskip 12pt                 % Absätze durch Lücke trennen

%\usepackage{titlesec}    % Abstand nach Überschriften neu definieren
%\titlespacing{\subsection}{0ex}{3ex}{-1ex}
%\titlespacing{\subsubsection}{0ex}{3ex}{-1ex}        

% \pagestyle{empty}
\setlength{\textheight}{23cm}
\usepackage{fancyhdr}
\pagestyle{fancy}
\cfoot{}
\lfoot{Zusammenkunft aller Physik-Fachschaften}
\rfoot{www.zapfev.de\\stapf@zapf.in}
\renewcommand{\headrulewidth}{0pt}
\renewcommand{\footrulewidth}{0.1pt}
\newcommand{\gen}{*innen}

\begin{document}
    \hspace{0.87\textwidth}
    \begin{minipage}{120pt}
        \vspace{-1.8cm}
        \includegraphics[width=80pt]{logo.pdf}
        \centering
        \small Zusammenkunft aller Physik-Fachschaften
    \end{minipage}
    \begin{center}
        \huge{Positionspapier zur Symptompflicht auf Attesten} \\
        \normalsize
    \end{center}
    \vspace{1cm}    
Die Zusammenkunft aller Physikfachschaften (ZaPF) spricht sich gegen  die geforderte Angabe von Symptomen auf Attesten für die Prüfungsunfähigkeitsmeldung aus. 

An vielen Universitäten ist es erforderlich, für den Nachweis der Prüfungsunfähigkeit ein ärztliches Attest mit der Angabe von Symptomen einzureichen. Der Prüfungsausschuss entscheidet darüber, ob die Symptome im jeweiligen Fall eine Prüfungsunfähigkeit darstellen. 

Aus unserer Sicht sprechen mehrere Gründe gegen diese Regelung: 
\begin{itemize}
\item Studierende müssen Ärzt\gen\ "`freiwillig"' von der Schweigepflicht entbinden 
\item Die Mitglieder der Prüfungsausschüsse haben in der Regel keine Qualifikation, um über Leistungseinschränkungen durch die angegebenen Symptome zu entscheiden. 
\item Die Weitergabe und Speicherung solcher hochsensibler Daten birgt das Risiko, dass ungewollt Dritte Kenntnis darüber gelangen 
\end{itemize}

Wir fordern die Gesetzgeber daher dazu auf, ausschließlich folgendes Verfahren zu ermöglichen: 
Eine Arbeitsunfähigkeitsbescheinigung ist einer ärztlichen Prüfungsunfähigkeitsbescheinigung gleichzusetzen.
\vfill
    \begin{flushright}
        Verabschiedet am 28.05.2017 in Berlin
    \end{flushright}
\end{document}
%%% Local Variables:
%%% mode: latex
%%% TeX-master: t
%%% End:
