\documentclass[DIV=calc]{scrartcl}
\usepackage[utf8]{inputenc}
\usepackage[T1]{fontenc}
\usepackage[ngerman]{babel}
\usepackage{graphicx}

\usepackage{fixltx2e}
\usepackage{ellipsis}
\usepackage[tracking=true]{microtype}

\usepackage{lmodern}                        % Ersatz fuer Computer Modern-Schriften
\usepackage{hfoldsty}

\usepackage{fourier}             % Schriftart
\usepackage[scaled=0.81]{helvet}     % Schriftart

\usepackage{url}
\usepackage{tocloft}             % Paket für Table of Contents

\usepackage{xcolor}
\definecolor{urlred}{HTML}{660000}

\usepackage{hyperref}
\hypersetup{
    colorlinks=true,    
    linkcolor=black,    % Farbe der internen Links (u.a. Table of Contents)
    urlcolor=black,    % Farbe der url-links
    citecolor=black} % Farbe der Literaturverzeichnis-Links

\usepackage{mdwlist}     % Änderung der Zeilenabstände bei itemize und enumerate

\parindent 0pt                 % Absatzeinrücken verhindern
\parskip 12pt                 % Absätze durch Lücke trennen

%\usepackage{titlesec}    % Abstand nach Überschriften neu definieren
%\titlespacing{\subsection}{0ex}{3ex}{-1ex}
%\titlespacing{\subsubsection}{0ex}{3ex}{-1ex}        

% \pagestyle{empty}
\setlength{\textheight}{23cm}
\usepackage{fancyhdr}
\pagestyle{fancy}
\cfoot{}
\lfoot{Zusammenkunft aller Physik-Fachschaften}
\rfoot{www.zapfev.de\\stapf@zapf.in}
\renewcommand{\headrulewidth}{0pt}
\renewcommand{\footrulewidth}{0.1pt}
\newcommand{\gen}{*innen}

\begin{document}
    \hspace{0.87\textwidth}
    \begin{minipage}{120pt}
        \vspace{-1.8cm}
        \includegraphics[width=80pt]{logo.pdf}
        \centering
        \small Zusammenkunft aller Physik-Fachschaften
    \end{minipage}
    \begin{center}
        \huge{Handreichung zur Exzellenz-Strategie} \\
        \normalsize
    \end{center}
    
    \vspace{1cm}    
\textbf{Was ist die "`Exzelleninitiative I \& II"'?}\\\\
\begin{minipage}{0.05\textwidth}
~ % Einrückung
\end{minipage}
\begin{minipage}{0.95\textwidth}
Sie ist eine strukturierte Forschungsförderungsmaßnahme, welche Forschung auf hohem
internationalen Niveau ermöglichen soll. Ursprünglich war sie auch dazu gedacht, die Universitäten
untereinander weiter auszudifferenzieren. Dies ist nicht passiert, welches auch der Imboden-Bericht\footnotemark[1] bestätigt.\\
Dabei gab es bisher die folgenden Bereiche:
\begin{itemize}
\item Exzellenzcluster: Forschungsprojekte
\item Graduiertenschulen: strukturierte Promotionen
\item Universitäten mit Zukunftskonzepten: (\glq Exzellenzuniversität\grq)
\end{itemize}
\end{minipage}\vspace{2\baselineskip}\\
\textbf{Und was ist die Exzellenzstrategie?}\\\\
\begin{minipage}{0.05\textwidth}
~ % Einrückung
\end{minipage}
\begin{minipage}{0.95\textwidth}
Das ist die Fortsetzung der Exzellenzinitiative I \& II nach Evaluation des Instruments (Imboden-Bericht), bei der gerade die Antragsphase läuft.\\
Die möglichen Anträge laufen für:
\begin{itemize}
\item Exzellenzcluster
\item Exzellenzuniversitäten
\end{itemize}
Dabei sollen insgesamt 8 - 11 Exzellenzuniversitäten inklusive Verbünde von \mbox{2 - 3} Unis gefördert
werden.\\
Vorraussetzung sind mind. 2 bewilligte Cluster, bei Verbünden insgesamt mind. 3, wobei jede Uni
an einem Cluster beteiligt sein muss. Die jährlichen Mittel von rund 148 Millionen Euro sollen
folgendermaßen aufgeteilt werden: 10 bis 15 Millionen Euro für einzelne Universitäten und 15 bis
28 Millionen Euro für Universitätsverbünde. Die Entscheidungskriterien sind hierbei die bisherige
Leistungen, der Status Quo und zukünftige Planungsideen \footnotemark[2].\\
Nähere Informationen findet ihr auf den Seiten der DFG\footnotemark[3]\footnotemark[4].\\
Hierbei muss der Bereich Lehre ein integraler Aspekt jeder Bewerbung sein.
Denn je besser Forschung und Lehre aufeinander abgestimmt sind, desto nachhaltiger
ist auch die Forschungsstrategie. Aus diesem Grund sollten alle Statusgruppen am Prozess der
Bewerbung beteiligt werden. Da zumindest alle größeren Universitäten Antragsskizzen eingereicht
haben, fordern wir alle Fachschaften auf, sich aktiv in die Bemühungen ihrer Universitäten zur
Exzellenzstrategie einzubringen.
\end{minipage}\vspace{2\baselineskip}\\
\textbf{Im Folgenden haben wir hierfür einige Hinweise zusammengestellt}\\\\
\begin{minipage}{0.05\textwidth}
~ % Einrückung
\end{minipage}
\begin{minipage}{0.95\textwidth}
Wo bekomme ich meine Infos her?\\\\
    \begin{minipage}{0.05\textwidth}
        ~ % Einrückung
    \end{minipage}
    \begin{minipage}{0.95\textwidth}
    \begin{itemize}
    \item Fragt bei eurem AStA/ bei den studentischen Vertreter\gen\ im Senat o.ä. nach, ob es bereits Arbeitsgruppen zur Organisation der Antragsstellung gibt. In der Regel werden die Informationen darüber nicht an die Fachschaften verteilt.
    \item Gibt es bestimmte Personen im Präsidium, die die Anträge der Exzellenzstrategie koordinieren? Falls ja, sind diese auch sehr gute Ansprechpartner\gen.
    \item Welche Professor\gen\ an eurer Fakultät/ eurem Institut/ Fachbereich sind für die Cluster etc. verantwortlich?
    \item Gleichstellungsbeauftragte sollten in der Regel ebenfalls im Prozess eingebunden sein, da sie in dem Antrag Stellung beziehen müssen.
    \item Wenn eure Uni bereits an Exzellenzinitiativen  teilnimmt oder bereits einen/ mehrere Cluster hat, wurden sie auf jeden Fall dazu aufgefordert einen Antrag zu stellen. Fragt am besten bei den Sprecher\gen\ der jeweiligen Cluster/ Teilprojektleiter\gen\ nach, Informationen hierzu sollten (online) leicht zu finden sein.
    \end{itemize}
    \end{minipage}\vspace{2\baselineskip}\\
    Wo können sich Studierende während der Antragsphase einbringen?\\\\
    \begin{minipage}{0.05\textwidth}
        ~ % Einrückung
    \end{minipage}
    \begin{minipage}{0.95\textwidth}
    \begin{itemize}
    \item Versucht Studierende in Arbeitsgruppen zur Exzellenzinitiative beispielsweise über den Senat o.ä. mit einzubringen.
    \item Sollte es extra Arbeitsgruppen zu Forschung und Lehre, wissenschaftlichem Nachwuchs, Gleichstellung und Diversität u.ä. geben, dann arbeitet dort aktiv mit. Wenn nicht, dann schlagt sie vor.
    \item Bringt euch in die Arbeitsgruppen der Clusterantragssteller\gen\ ein.
    \item Bei Berufungen (die eventuell im Bereich der angestrebten
Exzellenzcluster liegen könnten) solltet ihr besonders darauf achten,
dass der Aspekt der Lehre ausreichend beachtet wird.
    \item Wie steht es um die Sprachkenntnisse der Kandidierenden?
    \end{itemize}
    \end{minipage}\\\\
\end{minipage}
\begin{minipage}{0.05\textwidth}
~
\end{minipage}
\begin{minipage}{0.95\textwidth}
    Wenn Cluster da sind\\\\
    \begin{minipage}{0.05\textwidth}
        ~ % Einrückung
    \end{minipage}
    \begin{minipage}{0.95\textwidth}
    \begin{itemize}
    \item Beobachtet die Entwicklungen der Cluster und setzt euch für eine akademische
Selbstverwaltung ein.
    \item Achtet bei Berufungen besonders darauf, dass der Aspekt der Lehre ausreichend beachtet wird.
    \item Achtet auch auf die Sprachkenntnisse der Kandidierenden.
    \item Schaut darauf, dass Module in der Forschungsrichtung angeboten werden.
    \item Es kann eine Vertiefungsrichtung passend zu Forschungsthemen und -methoden des Exzellenzclusters gebildet werden. Der Vorteil gegenüber eines Studiengangs ist, dass Vertiefungsrichtungen zügig ins Curriculum integriert, aber auch wieder aufgehoben werden können.
    \item Bringt euch bei der Einrichtung von Studiengängen ein.
    \item Partizipiert bei der Entwicklung von geeigneten Lehrveranstaltungskonzepten wie
    \begin{itemize}
    \item Lehrforschungsprojekten mit Forschungsthemen der Cluster
    \item Team-Praktika
    \item offene Werkstatt/Labor
    \item in denen Studierende eigenen Forschungsfragen nachgehen können
    \item Lernlabore/ Schülerlabore
    \item Angebote im Rahmen der wissenschaftlichen Weiterbildung z.B. durch Kontakt- oder Fortbildungsstudiengänge
    \end{itemize}
    \end{itemize}
    \end{minipage}
\end{minipage}
\footnotetext[1]{\href{http://www.gwk-bonn.de/fileadmin/Papers/Imboden-Bericht-2016.pdf}{http://www.gwk-bonn.de/fileadmin/Papers/Imboden-Bericht-2016.pdf}}
\footnotetext[2]{\href{https://bwsyncandshare.kit.edu/dl/fiTtVEq4PVan1g4nrZqn5DaL/Foerderkriterien_Exzellenzuniversitaeten.pdf}{https://bwsyncandshare.kit.edu/dl/fiTtVEq4PVan1g4nrZqn5DaL/\\Foerderkriterien-\_Exzellenzuniversitaeten.pdf}}
\footnotetext[3]{\href{http://www.dfg.de/foerderung/programme/exzellenzstrategie/index.html}{http://www.dfg.de/foerderung/programme/exzellenzstrategie/index.html}}
\footnotetext[4]{\href{http://www.dfg.de/download/pdf/foerderung/programme/exzellenzstrategie/zeitplan_exzellenzstrategie.pdf}{http://www.dfg.de/download/pdf/foerderung/programme/exzellenzstrategie/\\zeitplan\_exzellenzstrategie.pdf}}
    \vfill
    \begin{flushright}
        Verabschiedet am 28.05.2017 in Berlin
    \end{flushright}
\end{document}
%%% Local Variables:
%%% mode: latex
%%% TeX-master: t
%%% End:
