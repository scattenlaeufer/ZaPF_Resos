\documentclass[DIV=calc]{scrartcl}
\usepackage[utf8]{inputenc}
\usepackage[T1]{fontenc}
\usepackage[ngerman]{babel}
\usepackage{graphicx}

\usepackage{fixltx2e}
\usepackage{ellipsis}
\usepackage[tracking=true]{microtype}

\usepackage{lmodern}                        % Ersatz fuer Computer Modern-Schriften
\usepackage{hfoldsty}

\usepackage{fourier}             % Schriftart
\usepackage[scaled=0.81]{helvet}     % Schriftart

\usepackage{url}
\usepackage{tocloft}             % Paket für Table of Contents

\usepackage{xcolor}
\definecolor{urlred}{HTML}{660000}

\usepackage{hyperref}
\hypersetup{
    colorlinks=true,    
    linkcolor=black,    % Farbe der internen Links (u.a. Table of Contents)
    urlcolor=black,    % Farbe der url-links
    citecolor=black} % Farbe der Literaturverzeichnis-Links

\usepackage{mdwlist}     % Änderung der Zeilenabstände bei itemize und enumerate

\parindent 0pt                 % Absatzeinrücken verhindern
\parskip 12pt                 % Absätze durch Lücke trennen

%\usepackage{titlesec}    % Abstand nach Überschriften neu definieren
%\titlespacing{\subsection}{0ex}{3ex}{-1ex}
%\titlespacing{\subsubsection}{0ex}{3ex}{-1ex}        

% \pagestyle{empty}
\setlength{\textheight}{23cm}
\usepackage{fancyhdr}
\pagestyle{fancy}
\cfoot{}
\lfoot{Zusammenkunft aller Physik-Fachschaften}
\rfoot{www.zapfev.de\\stapf@zapf.in}
\renewcommand{\headrulewidth}{0pt}
\renewcommand{\footrulewidth}{0.1pt}
\newcommand{\gen}{*innen}

\begin{document}
    \hspace{0.87\textwidth}
    \begin{minipage}{120pt}
        \vspace{-1.8cm}
        \includegraphics[width=80pt]{logo.pdf}
        \centering
        \small Zusammenkunft aller Physik-Fachschaften
    \end{minipage}
    \begin{center}
        \huge{Positionspapier zur BaMa-Umfrage}\vspace{.25\baselineskip}\\ \large{Rahmenbedingungen für die Befragung der Physik-Studierenden des deutschsprachigen Raumes
(„BaMa-Umfrage“)} \\
        \normalsize
    \end{center}
    \vspace{1cm}    
Die 2010 und 2014 durchgeführte Umfrage unter den Physik-Studierenden in Deutschland ("`BaMaUmfrage"') soll in Zukunft weiter 
fortgeführt werden. Die nächste Befragung im Rahmen der BaMa-Umfrage soll im Sommersemester 2018 stattfinden. Das folgende Konzept soll den \textsc{LEUTE für HUMBUG} bei der weiteren Erarbeitung der kommenden BaMa-Umfrage als Richtlinie dienen:
\begin{enumerate}
\item Langfristig soll die Entwicklung von Studiengängen und die Veränderung der Studienzufriedenheit erhoben werden.
\item Die Umfrage richtet sich schwerpunktmäßig an Studierende der Physik und physiknaher Fächer, vor allem, aber nicht 
ausschließlich, an Bachelor und Master-Studierende.
\item Die Umfrage soll langfristig die Physikstudierenden der Länder Deutschland, Österreich und Schweiz berücksichtigen.
\item Die Umfrage beinhaltet Kernfragen, die über einen längeren Zeitraum betrachtet werden, sowie Zusatzfragen, die aus 
aktuellen Themen hervorgehen.
\item Die Befragung soll 20 bis 25 Fragen umfassen, davon sollen ca. 75\% als Kernfragen und der Rest als Zusatzfragen 
formuliert werden.
\item Die Fragen werden in deutscher und englischer Sprache formuliert.
\item Die Rohdaten sollen geeignet - unter Berücksichtigung einer angemessenen Anonymität bzw. des Datenschutzes - zur freien 
Verwendung veröffentlicht werden.
\item Die Befragung findet in digitaler Form statt.
\item Die Befragung der Studierenden soll alle 4 Jahre wiederholt werden .
\item Die Fachschaften werden regelmäßig, alle 1 bis 2 Jahre, über den Aufbau des Studiums befragt. Weiter beurteilen sie, zu 
welchem Teil Physik in den betreffenden Studiengängen vertreten ist.
\end{enumerate}
\vspace{-0.5\baselineskip}
    \begin{flushright}
        Verabschiedet am 28.05.2017 in Berlin
    \end{flushright}
\end{document}
%%% Local Variables:
%%% mode: latex
%%% TeX-master: t
%%% End:
