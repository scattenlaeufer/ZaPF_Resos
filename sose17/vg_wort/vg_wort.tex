\documentclass[DIV=calc]{scrartcl}
\usepackage[utf8]{inputenc}
\usepackage[T1]{fontenc}
\usepackage[ngerman]{babel}
\usepackage{graphicx}

\usepackage{fixltx2e}
\usepackage{ellipsis}
\usepackage[tracking=true]{microtype}

\usepackage{lmodern}                        % Ersatz fuer Computer Modern-Schriften
\usepackage{hfoldsty}

\usepackage{fourier}             % Schriftart
\usepackage[scaled=0.81]{helvet}     % Schriftart

\usepackage{url}
\usepackage{tocloft}             % Paket für Table of Contents

\usepackage{xcolor}
\definecolor{urlred}{HTML}{660000}

\usepackage{hyperref}
\hypersetup{
    colorlinks=true,    
    linkcolor=black,    % Farbe der internen Links (u.a. Table of Contents)
    urlcolor=black,    % Farbe der url-links
    citecolor=black} % Farbe der Literaturverzeichnis-Links

\usepackage{mdwlist}     % Änderung der Zeilenabstände bei itemize und enumerate

\parindent 0pt                 % Absatzeinrücken verhindern
\parskip 12pt                 % Absätze durch Lücke trennen

%\usepackage{titlesec}    % Abstand nach Überschriften neu definieren
%\titlespacing{\subsection}{0ex}{3ex}{-1ex}
%\titlespacing{\subsubsection}{0ex}{3ex}{-1ex}        

% \pagestyle{empty}
\setlength{\textheight}{23cm}
\usepackage{fancyhdr}
\pagestyle{fancy}
\cfoot{}
\lfoot{Zusammenkunft aller Physik-Fachschaften}
\rfoot{www.zapfev.de\\stapf@zapf.in}
\renewcommand{\headrulewidth}{0pt}
\renewcommand{\footrulewidth}{0.1pt}
\newcommand{\gen}{*innen}

\begin{document}
    \hspace{0.87\textwidth}
    \begin{minipage}{120pt}
        \vspace{-1.8cm}
        \includegraphics[width=80pt]{logo.pdf}
        \centering
        \small Zusammenkunft aller Physik-Fachschaften
    \end{minipage}
    \begin{center}
        \huge{Offener Brief zum Thema VG-Wort} \\
        \large{An den Petitionsausschuss des Deutschen Bundestages, die Bundestagsfraktionen, die Kultusministerkonferenz, die Hochschulrektorenkonferenz und Verwertungsgesellschaft WORT}
        \normalsize
    \end{center}
    \vspace{0.5cm}
    Sehr geehrte Damen und Herren,\\ 
\vspace{0.2cm}\\
Die Fachschaftentagung Maschinenbau und die Zusammenkunft aller Physikfachschaften haben den Novellierungsprozess des Urheberrechts aufmerksam verfolgt. Wir sind  davon überzeugt, dass der inzwischen vom Kabinett beschlossene  Regierungsentwurf eine Verbesserung für die Studierenden an deutschen  Hochschulen bringen wird. Der Entwurf ist zur aktuellen Lage ein  Fortschritt und als solcher zu honorieren. Er berücksichtigt die Pauschalabrechnung, wie von uns bereits in der Vergangenheit gefordert wurden (vgl. Petition der FaTaMa und weiterer Tagungen von September 2016 \footnote{\url{https://epetitionen.bundestag.de/petitionen/_2017/_02/_06/Petition_69880.nc.html}}) und dürfte somit zu einer Entschärfung des Konflikts zwischen  Hochschulen und der Verwertungsgesellschaft WORT (VG WORT) führen.\\
Trotzdem stellen wir fest, dass auch eine Abrechnung auf Basis von  Stichproben erfolgen kann. Da hier keine weiteren Informationen zur Durchführung  der Stichprobenerhebung vorliegen, lehnen wir dies vorerst ab, da an einzelnen Institutionen  ähnliche Belastungen wie durch Einzelfallabrechnungen herbeigeführt  werden können.
Wir betrachten mit Sorge die Möglichkeit, dass die  Gesetzesänderung nicht vor Ende des aktuellen Moratoriums, also bis Ende September, zum Vertrag zwischen Hochschulen und VG WORT zum Tragen  kommt. Aus diesem Grunde fordern wir die Verhandlungspartner auf, das  Moratorium bis zum Inkrafttreten des Gesetzes zu verlängern. In diesem  Zusammenhang sprechen sich die Fachschaftentagung Maschinenbau und die Zusammenkunft aller Physikfachschaften  für mehr Transparenz rund um den  Verhandlungsprozess aus. Auch ist eine Beteiligung aller betroffenen Statusgruppen  sinnvoll. Insbesondere sollten bundesweite Vertreter  der  Studierendenschaften als Teil der Abordnung der Hochschulen mit  einbezogen werden. 

Der selbstgesetzte Zeitrahmen der Gesetzesverabschiedung muss eingehalten werden. Im Gesetzgebungsprozess muss auf die Forderungen  seitens der Wissenschaft eingegangen werden. Dies gilt insbesonders für  die für jeden Studiengang absolut notwendige Arbeit mit wissenschaftlichen Texten und Publikationen. Besonders die digitale  Entwicklung in den letzten Jahren macht eine Reform des Urheberrechts  unumgänglich. Neben digitalen Semesterapparaten sind auch digitale Fernleihen ein Thema welches wissenschaftliche Arbeit und Lehre  vereinfachen würden. Wir fordern an dieser Stelle ebenfalls eine Stärkung von Open Access  Angeboten, welche mit einer nachhaltigen Digitalisierung des Hochschul-  und Bildungsbereichs im Allgemeinen einhergehen. Wir bitten den Gesetzgeber, sicherzustellen, dass das Gesetz der  Weiterentwicklung digitaler Lehrangebote, wie zum Beispiel  Vorlesungsaufzeichnungen, aber auch  Vorlesungsübertragungen in andere Hörsäle, nicht im Wege stehen. Hierbei muss die an vielen Hochschulen angespannte Raumauslastung angemessen berücksichtigt werden. Wir sehen insbesondere  §60a Abs (3) Punkt 1 diesbezüglich sehr kritisch. Hierbei muss unbedingt die Entwicklung der Lehre in den letzten Jahren berücksichtigt werden, was insbesondere Konzepte der digitalen Lehre wie Blended Learning, digitale Hochschulen, etc. mit einschließt. Es ist hochgradig zu bedauern, wenn die Bemühungen der Hochschulen und des Hochschulforums Digitalisierung durch eine Gesetztesänderung entwertet würden (vgl. Abschlussbericht des Hochschulforums Digitalisierung \footnote{\url{https://hochschulforumdigitalisierung.de/sites/default/files/dateien/HFD_Abschlussbericht_Kurzfassung.pdf}} ). Hier sind sämtliche  Akteure der Hochschullandschaft in der Verantwortung, die Entwicklung  voran zu treiben. Wir sehen den aktuellen Regierungsentwurf zum  Urheberrecht nicht in der Lage dazu den sich stetig verändernden Gegebenheiten der Hochschulbildung gerecht zu werden. Daher appellieren  wir an die Akteure sich bereits
 jetzt in einen Dialog zur Zukunft eines  Urheberrechts zu begeben, welches auch diesen Anforderungen gerecht  wird. Die Entwicklungen der vergangenen Monate haben gezeigt, dass eine Verschleppung solcher Reformen fatale Auswirkungen für die Lehre an Hochschulen haben können.
    \vfill
    \begin{flushright}
        Verabschiedet am 28.05.2017 in Berlin
    \end{flushright}
    
\end{document}

%%% Local Variables:
%%% mode: latex
%%% TeX-master: t
%%% End:
