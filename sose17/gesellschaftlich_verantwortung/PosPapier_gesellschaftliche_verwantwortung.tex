\documentclass[DIV=calc]{scrartcl}
\usepackage[utf8]{inputenc}
\usepackage[T1]{fontenc}
\usepackage[ngerman]{babel}
\usepackage{graphicx}

\usepackage{fixltx2e}
\usepackage{ellipsis}
\usepackage[tracking=true]{microtype}

\usepackage{lmodern}                        % Ersatz fuer Computer Modern-Schriften
\usepackage{hfoldsty}

\usepackage{fourier}             % Schriftart
\usepackage[scaled=0.81]{helvet}     % Schriftart

\usepackage{url}
\usepackage{tocloft}             % Paket für Table of Contents

\usepackage{xcolor}
\definecolor{urlred}{HTML}{660000}

\usepackage{hyperref}
\hypersetup{
    colorlinks=true,    
    linkcolor=black,    % Farbe der internen Links (u.a. Table of Contents)
    urlcolor=black,    % Farbe der url-links
    citecolor=black} % Farbe der Literaturverzeichnis-Links

\usepackage{mdwlist}     % Änderung der Zeilenabstände bei itemize und enumerate

\parindent 0pt                 % Absatzeinrücken verhindern
\parskip 12pt                 % Absätze durch Lücke trennen

%\usepackage{titlesec}    % Abstand nach Überschriften neu definieren
%\titlespacing{\subsection}{0ex}{3ex}{-1ex}
%\titlespacing{\subsubsection}{0ex}{3ex}{-1ex}        

% \pagestyle{empty}
\setlength{\textheight}{23cm}
\usepackage{fancyhdr}
\pagestyle{fancy}
\cfoot{}
\lfoot{Zusammenkunft aller Physik-Fachschaften}
\rfoot{www.zapfev.de\\stapf@zapf.in}
\renewcommand{\headrulewidth}{0pt}
\renewcommand{\footrulewidth}{0.1pt}
\newcommand{\gen}{*innen}

\begin{document}
    \hspace{0.87\textwidth}
    \begin{minipage}{120pt}
        \vspace{-1.8cm}
        \includegraphics[width=80pt]{logo.pdf}
        \centering
        \small Zusammenkunft aller Physik-Fachschaften
    \end{minipage}
    \begin{center}
        \huge{Positionspapier zu Gesellschaftliche Verantwortung der Hochschulen} \\
        \normalsize
    \end{center}
    
    \vspace{1cm}    
    
\textbf{Gesellschaftliche Verantwortung der Hochschulen}\\

Die ZaPF spricht sich dafür aus, dass „die Hochschulen (...) ihren Beitrag zu einer [gerechten,] nachhaltigen, friedlichen und demokratischen Welt“
entwickeln. Sie spricht sich weiterhin dafür aus, dass die Hochschulen sich „friedlichen Zielen“ verpflichten und „ihrer besonderen Verantwortung für
eine nachhaltige Entwicklung nach innen und außen“\footnote{Hochschulgesetz NRW, § 3 Abs. 6} nachkommen.\\

Hochschulen müssen in der Position sein, zu Aufklärung über Falschdarstellungen, Kriegsursachen und -profiteure beizutragen, sowie an – nicht
ergriffenen und noch zu entwickelnden – zivilen Möglichkeiten zum Beispiel zur Lösung von Ressourcenkonflikten zu forschen. Dieser Funktion können
Hochschulen nur nachkommen, wenn ihre Unabhängigkeit gewahrt ist.\\

Die ZaPF setzt sich gegen Kooperationsprojekte ein, die diesen Aufgaben im Wege stehen oder Rüstung, Kriegsvorbereitung oder -durchführung dienen.
    \vfill
    \begin{flushright}
        Verabschiedet am 28.05.2017 in Berlin
    \end{flushright}
    
\end{document}
%%% Local Variables:
%%% mode: latex
%%% TeX-master: t
%%% End:
