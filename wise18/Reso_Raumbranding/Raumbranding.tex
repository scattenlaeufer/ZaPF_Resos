\documentclass[DIV=calc]{scrartcl}
\usepackage[utf8]{inputenc}
\usepackage[T1]{fontenc}
\usepackage[ngerman]{babel}
\usepackage{graphicx}
\usepackage[draft, markup=underlined]{changes}
\usepackage{csquotes}

\usepackage{ulem}
%\usepackage[dvipsnames]{xcolor}
\usepackage{paralist}
\usepackage{fixltx2e}
%\usepackage{ellipsis}
\usepackage[tracking=true]{microtype}

\usepackage{lmodern}                        % Ersatz fuer Computer Modern-Schriften
%\usepackage{hfoldsty}

%\usepackage{fourier}             % Schriftart
\usepackage[scaled=0.81]{helvet}     % Schriftart

\usepackage{url}
\usepackage{tocloft}             % Paket für Table of Contents

\usepackage{xcolor}
\definecolor{urlred}{HTML}{660000}

\usepackage{hyperref}
\hypersetup{
    colorlinks=true,    
    linkcolor=black,    % Farbe der internen Links (u.a. Table of Contents)
    urlcolor=black,    % Farbe der url-links
    citecolor=black} % Farbe der Literaturverzeichnis-Links

\usepackage{mdwlist}     % Änderung der Zeilenabstände bei itemize und enumerate
\usepackage{draftwatermark} % Wasserzeichen ``Entwurf'' 
\SetWatermarkText{}

\parindent 0pt                 % Absatzeinrücken verhindern
\parskip 12pt                 % Absätze durch Lücke trennen

\setlength{\textheight}{23cm}
\usepackage{fancyhdr}
\pagestyle{fancy}
\fancyhead{} % clear all header fields
\cfoot{}
\lfoot{Zusammenkunft aller Physik-Fachschaften}
\rfoot{www.zapfev.de\\stapf@zapf.in}
\renewcommand{\headrulewidth}{0pt}
\renewcommand{\footrulewidth}{0.1pt}
\newcommand{\gen}{*innen}
\addto{\captionsngerman}{\renewcommand{\refname}{Quellen}}

%%%% Mit-TeXen Kommandoset
\usepackage[normalem]{ulem}
\usepackage{xcolor}

\newcommand{\replace}[2]{
    \sout{\textcolor{blue}{#1}}~\textcolor{blue}{#2}
}

\newcommand{\delete}[1]{
    \sout{\textcolor{red}{#1}}
}

\newcommand{\add}[1]{
    \textcolor{green}{#1}
}



\begin{document}
    \hspace{0.87\textwidth}
    \begin{minipage}{120pt}
        \vspace{-1.8cm}
        \includegraphics[width=80pt]{../../logo.pdf}
        \centering
        \small Zusammenkunft aller Physik-Fachschaften
    \end{minipage}
    \begin{center}
        \huge{Resolution der Zusammenkunft aller Physik-Fachschaften}\vspace{.25\baselineskip}\\
        \normalsize
    \end{center}
    \vspace{1cm}

\section*{Gegen außeruniversitäre Werbung in Lern- und Lehrräumen}
Die Zusammenkunft aller Physik Fachschaften (ZaPF) fordert die Unterlassung von Raumbranding\footnote{Hörsaal- und Raumbranding bedeutet in diesem Fall den Verkauf von Namensrechten von
Hörsälen und anderen Lehr- und Lernräumen. In konkreten Fällen kann dies das Anbringen von
Firmenlogos am und im betroffenen Raum und an der Rauminfrastruktur, sowie die Eintragung des
Namens ins Raumverwaltungssystem der Hochschule bedeuten.} und außeruniversitärer Werbung\footnote{Unter universitärer Werbung wird Werbung für direkt studien- und universitätsrelevante
Veranstaltungen und Ähnliches von nicht kommerziellen Einrichtungen verstanden,
außeruniversitäre Werbung ist Werbung, die nicht unter diese Einschränkung fällt.} in allen Lern- und Lehrräumen (z.B. Bibliotheken, Hörsäle,
Übungsräume, Praktikumsräume) bei Lehrbetrieb.
Sinn der Lehrveranstaltungen und des Lernbetriebs ist es, dass Studierende, unbeeinflusst von
Interessen Dritter, Fachinhalte erlernen und diskutieren, sowie Lehrende Lehrinhalte frei vermitteln
können. Raumbranding steht zu diesem Prozess im Widerspruch.
Auch ist Raumbranding insbesondere deshalb abzulehnen, da es eine sehr einseitige Form der
Werbung darstellt, der sich die Teilnehmenden von Lehrveranstaltungen nicht entziehen können.\\
Wir sehen die Entscheidung solcher Fragen als Grundsatzfrage an und fordern daher den offenen
und transparenten Diskurs in den legitimierten Vertretungen aller Statusgruppen der Universität.\vspace{2\baselineskip}\\~
\vfill
    \begin{flushright}
        Verabschiedet am 25.11.2018 in Würzburg
    \end{flushright}
\end{document}

