\documentclass[draft,10pt,oneside]{scrartcl}

% Sprache und Encodings
\usepackage[ngerman]{babel}
%\usepackage[T1]{fontenc}
\usepackage[utf8]{inputenc}

% Typographisch interessante Pakete
\usepackage{microtype} % Randkorrektur und andere Anpassungen

% References to Internet and within the document
\usepackage[pdftex,colorlinks=false,
pdftitle={Antrag zur Änderung der Geschäftsordnung für Plenen der ZaPF},
pdfauthor={Jörg Behrmann (FUB), Björn Guth (RWTH)},
pdfcreator={pdflatex},
pdfdisplaydoctitle=true]{hyperref}

% Absaetze nicht Einruecken
\setlength{\parindent}{0pt}
\setlength{\parskip}{2pt}

% Formatierung auf A4 anpassen
\usepackage{geometry}
\geometry{paper=a4paper,left=15mm,right=15mm,top=10mm,bottom=10mm}

\begin{document}

\section*{Antrag zur Änderung der Satzung der ZaPF}

\textbf{Antragsteller:} Jörg Behrmann (FUB), Björn Guth (RWTH)

\subsection*{Antrag}

Hiermit beantragen wir die Satzung der ZaPF wie folgend zu
ändern:

In §5 füge als letzten Absatz vor §5(a) ein:
\begin{quote}
    Für Amtszeiten sei ein Jahr definiert als die Zeit zwischen einer
    Sommer-ZaPF und der ihr nächsten nachfolgenden Sommer-ZaPF bzw. einer
    Winter-ZaPF und der ihr nächsten nachfolgenden Winter-ZaPF. Das Jahr
    beginnt mit dem Plenum in dem die Wahl für ein Organ turnusmäßig
    stattfindet und endet mit dem Plenum in dem die Wahl zur Neubesetzung der
    entsprechenden Plätze des Organs turnusmäßig stattfindet, spätestens jedoch
    mit dem Ende der Tagung.

\end{quote}

\subsection*{Begründung}
Schon immer wurden Wahlen und Neuwahlen zu Posten im ZaPF-Organen gemäß der der
Intention der Satzung vorgenommen. Diese Satzungsänderung passt den Text nun
auch der Intention an.

\end{document}
