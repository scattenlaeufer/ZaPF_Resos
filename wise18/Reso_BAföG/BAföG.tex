\documentclass[DIV=calc]{scrartcl}
\usepackage[utf8]{inputenc}
\usepackage[T1]{fontenc}
\usepackage[ngerman]{babel}
\usepackage{graphicx}
\usepackage[draft, markup=underlined]{changes}
\usepackage{csquotes}

\usepackage{ulem}
%\usepackage[dvipsnames]{xcolor}
\usepackage{paralist}
\usepackage{fixltx2e}
%\usepackage{ellipsis}
\usepackage[tracking=true]{microtype}

\usepackage{lmodern}                        % Ersatz fuer Computer Modern-Schriften
%\usepackage{hfoldsty}

%\usepackage{fourier}             % Schriftart
\usepackage[scaled=0.81]{helvet}     % Schriftart

\usepackage{url}
\usepackage{tocloft}             % Paket für Table of Contents

\usepackage{xcolor}
\definecolor{urlred}{HTML}{660000}

\usepackage{hyperref}
\hypersetup{
    colorlinks=true,    
    linkcolor=black,    % Farbe der internen Links (u.a. Table of Contents)
    urlcolor=black,    % Farbe der url-links
    citecolor=black} % Farbe der Literaturverzeichnis-Links

\usepackage{mdwlist}     % Änderung der Zeilenabstände bei itemize und enumerate
\usepackage{draftwatermark} % Wasserzeichen ``Entwurf'' 
\SetWatermarkText{}

\parindent 0pt                 % Absatzeinrücken verhindern
\parskip 12pt                 % Absätze durch Lücke trennen

\setlength{\textheight}{23cm}
\usepackage{fancyhdr}
\pagestyle{fancy}
\fancyhead{} % clear all header fields
\cfoot{}
\lfoot{Zusammenkunft aller Physik-Fachschaften}
\rfoot{www.zapfev.de\\stapf@zapf.in}
\renewcommand{\headrulewidth}{0pt}
\renewcommand{\footrulewidth}{0.1pt}
\newcommand{\gen}{*innen}
\addto{\captionsngerman}{\renewcommand{\refname}{Quellen}}

%%%% Mit-TeXen Kommandoset
\usepackage[normalem]{ulem}
\usepackage{soul}
\usepackage{xcolor}

\newcommand{\replace}[2]{
    \sout{#1} \textcolor{blue}{#2}
}

\newcommand{\delete}[1]{
    %\sout{\textcolor{red}{#1}}
    \sout{#1}
}

\definecolor{darkgreen}{RGB}{19,107,69}

\newcommand{\add}[1]{
    \textcolor{blue}{#1}
}



\begin{document}
    \hspace{0.87\textwidth}
    \begin{minipage}{120pt}
        \vspace{-1.8cm}
        \includegraphics[width=80pt]{../logo.pdf}
        \centering
        \small Zusammenkunft aller Physik-Fachschaften
    \end{minipage}
    \begin{center}
        \huge{Resolution der Zusammenkunft aller  Physik-Fachschaften}\vspace{.25\baselineskip}\\
        \normalsize
    \end{center}
    \vspace{1cm}

\section*{Zum BAföG}


Die Zusammenkunft aller deutschsprachigen Physik-Fachschaften (ZaPF) begrüßt die durch
Bundesbildungsministerin Anja Karliczek angestrebte Erhöhung des BAföG für 2019, jedoch geht
diese Anpassungen aus der Sicht der ZaPF nicht weit genug. Um allen Studierenden die
Finanzierung ihres Studiums zu ermöglichen, sprechen wir uns daher für eine Novellierung des
BAföGs aus. Dabei sollte, neben realistischen Bezügen und Freibeträgen, die generelle
Elternunabhängigkeit beschlossen werden. Auch eine Verminderung des bürokratischen Aufwandes
sollte als Teil dieser Umgestaltung angestrebt werden.
\vspace{0.5\baselineskip}\\
Die momentane Situation der Elternabhängigkeit reduziert die Berechnung alleine auf das
Einkommen der Eltern. Individuelle familiäre Faktoren und Probleme werden dabei nicht
berücksichtigt und blockieren dadurch die benötigte Unterstützung durch das BAföG. Das kann
dazu führen, dass Studierende diese Förderung nicht erhalten. Daher spricht sich die ZaPF für ein
generelles elternunabhängiges BAföG
und die jährliche dynamische Anpassung an die sich verändernden
Lebenshaltungskosten aus. 
\vspace{0.5\baselineskip}\\
Im bisherigen BAföG-System entsteht durch die Bürokratie für Studierende und Behörden viel
Aufwand. Eine Vereinfachung dieser, wie es auch das elternunabhängige BAföG erzielen würde,
kann zu einer erheblichen Entlastung der Verwaltung, von Sacharbeiter*Innen und Studierenden
führen. Insbesondere würde dies den Bearbeitungsprozess der Anträge beschleunigen und
Planungssicherheit für Studierende erhöht.
\vspace{0.5\baselineskip}\\
Für eine moderne Bildungsgesellschaft ist der zuverlässige Zugang zu Hochschulbildung
unerlässlich. Die finanzielle Sicherung durch das BAföG würde hierzu einen wichtigen Beitrag
leisten.
\vfill
    \begin{flushright}
        Verabschiedet am 25.11.2018 in Würzburg
    \end{flushright}
\end{document}

