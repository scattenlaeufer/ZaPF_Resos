Die ZaPF sieht die Bedeutung von Drittmitteln für die moderne Forschung an öffentlichen Einrichtungen. Wir halten Transparenz bei der Durchführung von wissenschaftlichen Tätigkeiten im Interesse Dritter für notwendig. Deshalb fordert die ZaPF, dass bei Drittmittelprojekten folgende Angaben jährlich veröffentlicht werden müssen:\\
\begin{enumerate}
\item Titel des Projekts \footnotemark[1]
\item Hochschule mit Organisationseinheit
\item Auftraggebende Personen mit Sparte/Handlungsfeld der Abteilung \footnotemark[2]
\item Projekt- und Vertragslaufzeit 
\item Gesamtsumme
\item Angaben der Geheimhaltungsvereinbarungen oder Publikationsbeschränkungen, u. a. Art, Dauer und Umfang
\end{enumerate}
Zusätzlich muss am Projektende folgendes veröffentlicht werden:
\begin{enumerate}
\item[7.] Abstract \footnotemark[1]
\end{enumerate}
\footnotetext[1]{In begründeten Fällen können zeitlich befristete Ausnahmen bis zu einer Höchstdauer von zwei Jahren zugelassen werden.}
\footnotetext[2]{Der Verwendungszweck der Forschungsergebnisse muss aus dem angegebenen Handlungsfeld hervorgehen.}
