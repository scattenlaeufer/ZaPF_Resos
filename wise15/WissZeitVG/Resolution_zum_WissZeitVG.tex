\documentclass[parskip, 10pt,oneside]{scrartcl}

% Sprache und Encodings
\usepackage[ngerman]{babel}
\usepackage[T1]{fontenc}
\usepackage[utf8]{inputenc}
\usepackage{epigraph}

% Typographisch interessante Pakete
\usepackage{microtype} % Randkorrektur und andere Anpassungen

% References to Internet and within the document
\usepackage[pdftex,colorlinks=false,
pdftitle={Resolution zur Novelle des Wissenschaftszeitvertragsgesetzes},
pdfauthor={Björn Guth (Aachen), Jörg Behrmann (FUB), Tobias Löffler (Düsseldorf) und Jan Naumann (FUB)},
pdfcreator={pdflatex},
pdfdisplaydoctitle=true]{hyperref}

% Absaetze nicht Einruecken
\setlength{\parindent}{0pt}
\setlength{\parskip}{5pt}

% Formatierung auf A4 anpassen
\usepackage{geometry}
\geometry{paper=a4paper,left=20mm,right=20mm,top=10mm,bottom=10mm}

\renewcommand*\dictumwidth{0.95\linewidth}

\hyphenation{anonymer ano-ny-mer}
\begin{document}

\section*{Resolution zur Novelle des Wissenschaftszeitvertragsgesetzes}

\textbf{Adressaten:} An den Ausschuss für Bildung, Forschung und Technikfolgenabschätzung des Deutschen Bundestages, den Ausschuss für Kulturfragen sowie den Ausschuss für Frauen und Jugend des Deutschen Bundesrates, die bildungspolitischen Sprecher der Fraktionen im Deutschen Bundestag, die für Forschung zuständigen Ministerien der Bundesländer, das für das WissZeitVG zuständige Referat im Bundesministerium für Bildung und Forschung sowie Bundesministerin Johanna Wanka

\textbf{Antragssteller:} Jörg (FUB), Tobi (Düsseldorf), Björn (Aachen), Jan (FUB)

\subsection*{Antrag:}
Die ZaPF mögen beschließen:
\begin{quote}
	Die ZaPF begrüßt die Novellierung des WissZeitVG. Leider löst auch die Novellierung bestehende Probleme nicht vollständig und fügt dabei einige neue Probleme hinzu. 
Unter anderem werden verwendete Begrifflichkeiten, wie ``Qualifizierung'' oder  ``wissenschaftliche Hilfstätigkeit'', nicht abschließend geklärt und auch die Novelle sieht noch immer keine verbindlichen Mindestlaufzeit für Zeitverträge vor, so dass Arbeitsverträge beliebig befristet werden können.

Wir fordern, dass die Befristung von Arbeitsverträgen folgenden Mindeststandards entspricht:
Die Befristung muss dem Ziel der Anstellung entsprechen. Ist eine, wie auch immer geartete, Qualifikation das Ziel der Anstellung, muss die Befristung es erlauben, dieses Ziel zu erreichen.
Falls die Anstellung durch Drittmittel finanziert ist, darf die Befristung die Laufzeit dieser nicht unterschreiten.

Da die befristete Anstellung von Menschen aufgrund zeitlich befristeter Mittel schon durch das Gesetz über Teilzeitarbeit und befristete Arbeitsverträge geregelt wird, könnten die Regelungen für die Wissenschaft auch durch das Hinzufügen eines neuen Sachgrundes ``Qualifikation'' im §14 eben dieses Gesetzes geregelt werden, so dass auf das WissZeitVG verzichtet werden kann.
Ein Vertrag über die Qualifizierung, z.B. in der Form einer Promotionsvereinbarung, sollte in jedem Fall vor Beginn des Arbeitsverhältnisses abgeschlossen werden und das Qualifizierungsziel und die Qualifizierungsdauer festlegen.

Außerhalb der Qualifikation und anderen Sachgründen, wie zeitlich befristeteten Projekten, sehen wir befristete Anstellungen kritisch und fordern eine erhöhte Grundfinanzierung von Hochschulen und Universitäten, um eine angemesse Anzahl von Festanstellungen zu sichern. Insbesondere müssen Daueraufgaben durch unbefristete Anstellungen abgedeckt sein.

Neu hinzukommen zum WissZeitVG soll eine Regelung, welche die Arbeit als wissenschaftliche Hilfskraft für maximal vier Jahre während des Studiums von der maximalen Befristungsdauer freistellt.
Einer Festlegung der maximalen Arbeitsdauer in befristeten Arbeitsverträgen für Studierende widersprechen wir unabhängig davon, ob Studierende als wissenschaftliche Hilfskraft oder als wissenschaftliche Mitarbeiter, wie manchmal im Fall von Masterstudierenden, angestellt sind. 
Viele Studierende sind zur Finanzierung ihres Studiums auf wissenschaftliche Hilfskraftstellen angewiesen. Schon die aktuelle vorgeschlagene Lösung von vier Jahren unterschreitet die fünfjährige Regelstudienzeit eines akkreditierbaren Studiengangs aus Bachelor und Master. Damit wäre vor allem, wenn die Regelstudienzeit überschritten wird, die Finanzierungsmöglichkeit in der Endphase des Studiums kritisch, in welcher eine Arbeitsmöglichkeit bei Erreichen der Fristdauer entfällt. Dies betrifft gerade Studierende, welche ansonsten keine andere Finanzierungsmoglichkeit haben. Insbesondere in Studiengängen mit hoher Regelstudienzeit bzw. Studiengängen, die in Teilzeit absolviert, werden oder bei einem Studiengangswechsel kann sich eine Befristung so katastrophal auswirken.

Die Neuregelung fügt außerdem das Problem hinzu, dass vor Beginn des Studiums durchgeführte wissenschaftliche Hilfstätigkeiten auf die maximale Befristungsdauer angerechnet werden müssen, was z.B. fur Laboranten, die sich erst später fur ein Studium entscheiden, ein Problem darstellen würde.
Auch diese Anstellungen sind von der maximalen Befristungsdauer freizustellen, um auch für diese Gruppe die gleichen Chancen zu gewährleisten.

Eine begrüßenswerte Änderung der Novellierung und notwendige Verbesserung des Gesetzes ist der Vorschlag zur Verlängerung der maximal zulässigen Befristungsdauer für Eltern und Menschen mit Behinderung oder chronischen Erkrankungen. Deswegen fordern wir, diese Regelungen analog auch in den neuen Paragraphen für studentische Hilfskräfte zu übernehmen. 
Darüber hinaus ist es für uns zwingend erforderlich, Eltern, Menschen mit Behinderung oder chronischen Erkrakungen und Menschen mit pflegebedürftigen Angehörigen durch einen Rechtsanspruch auf Verlängerung befristeter Arbeitsverträge zu unterstützen.

Nicht zuletzt wollen wir Tarifverträge als wertvolles Werkzeug zur gemeinsamen Gestaltung von Arbeitsbedingungen gestärkt sehen. Daher fordern wir, dass auch im wissenschaftlichem Bereich eine volle Tariffreiheit gewährt wird, wofür die Streichung des Satzes §1, Abs. 1, Satz 2 notwendig ist.

\end{quote}



\end{document}
