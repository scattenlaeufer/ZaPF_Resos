\documentclass[DIV=calc]{scrartcl}
\usepackage[utf8]{inputenc}
\usepackage[T1]{fontenc}
\usepackage[ngerman]{babel}
\usepackage{graphicx}

\usepackage{fixltx2e}
\usepackage{ellipsis}
\usepackage[tracking=true]{microtype}

\usepackage{lmodern}                        % Ersatz fuer Computer Modern-Schriften
\usepackage{hfoldsty}

\usepackage{fourier} 			% Schriftart
\usepackage[scaled=0.81]{helvet} 	% Schriftart

\usepackage{url}
\usepackage{tocloft} 			% Paket für Table of Contents

\usepackage{xcolor}
\definecolor{urlred}{HTML}{660000}

\usepackage{geometry}
\geometry{left=3cm,right=3cm}

\usepackage{hyperref}
\hypersetup{
  colorlinks=true,	
  linkcolor=black,	% Farbe der internen Links (u.a. Table of Contents)
  urlcolor=black,	% Farbe der url-links
  citecolor=black} % Farbe der Literaturverzeichnis-Links

\usepackage{mdwlist} 	% Änderung der Zeilenabstände bei itemize und enumerate

\parindent 0pt 				% Absatzeinrücken verhindern
\parskip 12pt 				% Absätze durch Lücke trennen

\usepackage{titlesec}	% Abstand nach Überschriften neu definieren
\titlespacing{\subsection}{0ex}{3ex}{-1ex}
\titlespacing{\subsubsection}{0ex}{3ex}{-1ex}		

% \pagestyle{empty}
\setlength{\textheight}{23cm}
\usepackage{fancyhdr}
\pagestyle{fancy}
\cfoot{}
\lfoot{Zusammenkunft aller Physik-Fachschaften}
\rfoot{www.zapfev.de\\stapf@googlegroups.de}
\renewcommand{\headrulewidth}{0pt}
\renewcommand{\footrulewidth}{0.1pt}


\begin{document}
\hspace{0.87\textwidth}
\begin{minipage}{120pt}
\vspace{-1.8cm}
\includegraphics[width=80pt]{logo.pdf}
\centering
\small Zusammenkunft aller Physik-Fachschaften
\end{minipage}
\begin{center}
\huge{Positionspapier der Zusammenkunft aller Physik-Fachschaften zu Lernzentren} \\
\normalsize
\end{center}

Die ZaPF spricht sich für die Einrichtung bzw. Etablierung von Lernzentren an Physikfachbereichen aus.
Unter einem Lernzentrum verstehen wir dauerhaft zur Verfügung stehende Räumlichkeiten, die allen Studierenden offenstehen. Die Ausstattung soll Gruppenarbeit, sowie individuelle tutorielle Betreuung ermöglichen.

Die ZaPF empfiehlt die Ausgestaltung des Lernzentrums wie folgt:

Das Lernzentrum soll auf Studierende im Bachelor und Lehramt (Grundstudium) ausgerichtet sein. Der Schwerpunkt soll dabei auf der Hilfe zum Selbst- und Gruppenlernprozess liegen.
Gerade jüngere Studierende, die eine sehr verschulte Lehrform gewohnt sind, sollen so dazu angehalten werden, sich intensiv in Gruppen oder alleine mit dem Vorlesungsstoff  auseinandersetzen. 
Das schließt die Betreuung durch fachlich und didaktisch geschulte Tutoren ein. Diese sollen in der Lage sein, die besonderen Ansprüche, die durch das große inhaltliche Spektrum und individuelle Anforderungen entstehen, zu erfüllen. Der Fokus liegt auf Vermittlung der Methodik, die zur Problemlösung nötig ist. Die Tutoren brauchen demnach Praxiserfahrung oder eine Schulung im Rahmen eines Fachtutoren-Workshops.
Des Weiteren ist es wichtig, Zugang zu Literatur und Internet zu gewährleisten. Eine Grundausstattung an Lehrbüchern, die zur Präsenznutzung vorliegen, ist dazu sehr wichtig.
Zur Etablierung dieses Konzepts ist eine dauerhaft sichergestellte Finanzierung und freier Zugang zu den Räumlichkeiten essentiell notwendig.

Wünschenswert wäre eine gute Kommunikation zwischen den Verantwortlichen der Übungen der Grundvorlesungen und den Tutoren des Lernzentrums, um sich über Schwerpunkte und Inhalte der Vorlesungen und Probleme der Studierenden auszutauschen.
Ferner ist die Eingliederung des Lernzentrums in das E-Learning-System ein wichtiger Zusatz. Dadurch können Studierende bspw. Fragen in einem Forum stellen, auf die sich die Tutoren bereits im Vorfeld vorbereiten können. Allgemein erhöht sich dadurch die Erreichbarkeit des Zentrums.

Darüber hinaus sehen wir die Möglichkeit ein solches Zentrum interdisziplinär aufzubauen. Gerade bei kleineren Fachbereichen kann man so regelmäßige Öffnungszeiten und eine gesicherte Finanzierung gewährleisten. 
Zur Sicherung der Qualität empfehlen wir eine Evaluation des Zentrums und der Tutoren. 

Wir sehen in der Errichtung eines solchen Zentrums die Chance, die Studienqualität und Betreuung erkennbar zu erhöhen, den Einstieg ins Studium zu erleichtern und den Ehrgeiz und die Motivation über dessen gesamten Verlauf hoch zu halten.




\vfill
\begin{flushright}
Verabschiedet am 23.11.2013 in Bremen
\end{flushright}




\end{document}
