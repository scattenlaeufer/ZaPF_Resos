\documentclass[draft,12pt,oneside]{scrartcl}

% Sprache und Encodings
\usepackage[ngerman]{babel}
\usepackage[T1]{fontenc}
\usepackage[utf8]{inputenc}

% Typographisch interessante Pakete
\usepackage{microtype} % Randkorrektur und andere Anpassungen

% References to Internet and within the document
\usepackage[pdftex,colorlinks=false,
pdftitle={Antrag zur Änderung der Geschäftsordnung für Plenen der ZaPF},
pdfauthor={Jörg Behrmann (FUB), Benjamin Dummer (HUB), Zafer El-Mokdad (Potsdam), Björn Guth (RWTH Aachen)},
pdfcreator={pdflatex},
pdfdisplaydoctitle=true]{hyperref}

% Absaetze nicht Einruecken
\setlength{\parindent}{0pt}
\setlength{\parskip}{2pt}

% Formatierung auf A4 anpassen
\usepackage{geometry}
\geometry{paper=a4paper,left=20mm,right=20mm,top=10mm,bottom=10mm}

\begin{document}

\section*{Antrag zur Änderung der Geschäftsordnung für Plenen der ZaPF}

\textbf{Antragsteller:} Jörg Behrmann (FUB), Björn Guth (RWTH Aachen)

\subsection*{Antrag}

Hiermit beantragen wir die vorliegende Entwurfsfassung für die Geschäftsordnung für Plenen der ZaPF als neue
Geschäftsordnung für Plenen der ZaPF zu bestätigen.

\subsection*{Begründung}

Dieser Antrag auf Änderung der Geschäftsordnung dient der Abrundung der im
letzten Semester vorgenommenen Änderungen der Geschäftsordnung und löst einige
kosmetische Probleme sowie Dinge deren Regelung vergessen wurde, darüber hinaus
wird mit der Schließung der Liste potentieller Geschäftsordnungsanträge das
Missbrauchspotential der Geschäftsordnung gesenkt.

\subsection*{Die Änderungen im Überblick}

Die inhaltlichen Änderungen sind:
\begin{enumerate}
\item Schließung der Liste möglicher Geschäftsordnungsanträge und Hinzufügen der
  Geschäftsordnungsanträge auf namentliche Abstimmung und
  Verfahrensvorschlag.
  Diese Änderungen decken alle bekannten Geschäftsordnungsanträge ab und
  senken das Missbrauchspotential durch Hinzufügen willkürlicher, neuer
  Geschäftsordnungsanträge.
\item Die Antragsfristen zur Änderung der Geschäftsordnung auf dem
  Zwischenplenum wird festgeschrieben.
\item Das automatische Inkrafttreten von Geschäftsordnungsänderungen zum
  nächsten Plenum wird festgeschrieben.
\item Regelung dessen, was das Plenum ohne Beschlussfähigkeit tun kann. Dies
  spiegelt den Status Quo wider.
\item Änderung der Geschäftsordnungsanträge zur Neuwahl von Protokoll und
  Sitzungsleitung die eine Neuwahl mehrerer Protokollanten oder Mitglieder der
  Sitzungsleitung erlaubt.
\end{enumerate}

Die redaktionellen Änderungen sind:
\begin{enumerate}
\item Einheitliche Schreibung von Redeleitung als Sitzungsleitung.
\item Einheitliche Schreibung von Redeliste.
\end{enumerate}

\end{document}
