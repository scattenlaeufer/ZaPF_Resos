\documentclass[draft,12pt,oneside]{scrartcl}

% Sprache und Encodings
\usepackage[ngerman]{babel}
\usepackage[T1]{fontenc}
\usepackage[utf8]{inputenc}

% Typographisch interessante Pakete
\usepackage{microtype} % Randkorrektur und andere Anpassungen

% References to Internet and within the document
\usepackage[pdftex,colorlinks=false,
pdftitle={Antrag zur Änderung der Geschäftsordnung für Plenen der ZaPF},
pdfauthor={Jörg Behrmann (FUB), Björn Guth (RWTH Aachen)},
pdfcreator={pdflatex},
pdfdisplaydoctitle=true]{hyperref}

% Absaetze nicht Einruecken
\setlength{\parindent}{0pt}
\setlength{\parskip}{2pt}

% Formatierung auf A4 anpassen
\usepackage{geometry}
\geometry{paper=a4paper,left=15mm,right=15mm,top=5mm,bottom=5mm}

\begin{document}

\section*{Antrag zur Änderung der Geschäftsordnung für Plenen der ZaPF}

\textbf{Antragsteller:} Jörg Behrmann (FUB), Björn Guth (RWTH Aachen)

\subsection*{Antrag}

Hiermit beantragen wir die folgenden Änderungen an der Geschäftsordnung für Plenen der ZaPF durchzuführen:
\begin{enumerate}
	\item In §3.2 ersetze
		\begin{quote}
			Geschäftsordnungsanträge sind insbesondere Anträge
		\end{quote}
		durch
		\begin{quote}
			Geschäftsordnungsanträge sind folgende Anträge
		\end{quote}
		Außerdem füge der Liste der Geschäftsordnungsanträge
		\begin{itemize}
			\item Verfahrensvorschlag
			\item namentliche Abstimmung
		\end{itemize}
		hinzu.
	\item in §3.1 ersetze in Punkt 3
		\begin{quote}
			Zur Abstimmung im Abschlussplenum müssen Anträge zur Änderung der Geschäftsordnung spätestens um 15:00 Uhr am Tag vor dem Abschlussplenum bekanntgegeben werden.
		\end{quote}
		durch
		\begin{quote}
			Zur Abstimmung im Zwischen- oder Abschlussplenum müssen Anträge zur Änderung der Geschäftsordnung spätestens um 15:00 Uhr am Tag vor dem Zwischen- oder Abschlussplenum bekanntgegeben werden.
		\end{quote}
	\item In §3.1 füge in Punkt 3
		\begin{quote}
			Die Änderung der Geschäftsordnung tritt automatisch zum nächsten Plenum in Kraft.
		\end{quote}
		als letzten Satz hinzu.
	\item In §4 füge
		\begin{quote}
			Die Beschlussfähigkeit ist ausschließlich für Abstimmungen und Personenwahlen notwendig entsprechend dieser Geschäftsordnung. Nur das Plenum betreffende Abstimmungen können ohne Beschlussfähigkeit durchgeführt werden, dies betrifft insbesondere die Wahl der Redeleitung und der Protokollanten, sowie das Sitzungsende.
		\end{quote}
		zwischen dem ersten und dem zweiten Absatz als neuen Absatz ein.
	\item Ersetze die Geschäftsordnungsanträge
		\begin{itemize}
			\item Neuwahl der Redeleitung unter Benennung eines Gegenkandidaten
			\item Neuwahl des Protokollanten unter Benennung eines Gegenkandidaten
		\end{itemize}
		durch
		\begin{itemize}
			\item Neuwahl der Redeleitung unter Benennung eines oder mehrerer Gegenkandidaten
			\item Neuwahl des oder der Protokollanten unter Benennung eines oder mehrerer Gegenkandidaten
		\end{itemize}
	\item Redaktionelle Änderungen:
		\begin{enumerate}
			\item Vereinheitlichung der Schreibung von Redeleitung zu Sitzungsleitung.
			\item Vereinheitlichung der Schreibung von Rednerliste zu Redeliste.
			\item Ersetze
				\begin{itemize}
					\item Neuwahl des Protokollanten unter Benennung eines oder mehrerer Gegenkandidaten
				\end{itemize}
				durch
				\begin{itemize}
					\item Neuwahl des oder der Protokollanten unter Benennung eines oder mehrerer Gegenkandidaten
				\end{itemize}
		\end{enumerate}
\end{enumerate}

\subsection*{Begründung}

Dieser Antrag auf Änderung der Geschäftsordnung dient der Abrundung der im
letzten Semester vorgenommenen Änderungen der Geschäftsordnung und löst einige
kosmetische Probleme sowie Dinge deren Regelung vergessen wurde, darüber hinaus
wird mit der Schließung der Liste potentieller Geschäftsordnungsanträge das
Missbrauchspotential der Geschäftsordnung gesenkt.

\subsection*{Die Änderungen im Überblick}

Die inhaltlichen Änderungen sind:
\begin{enumerate}
\item Schließung der Liste möglicher Geschäftsordnungsanträge und Hinzufügen der
  Geschäftsordnungsanträge auf namentliche Abstimmung und
  Verfahrensvorschlag.
  Diese Änderungen decken alle bekannten Geschäftsordnungsanträge ab und
  senken das Missbrauchspotential durch Hinzufügen willkürlicher, neuer
  Geschäftsordnungsanträge.
\item Die Antragsfristen zur Änderung der Geschäftsordnung auf dem
  Zwischenplenum wird festgeschrieben.
\item Das automatische Inkrafttreten von Geschäftsordnungsänderungen zum
  nächsten Plenum wird festgeschrieben.
\item Regelung dessen, was das Plenum ohne Beschlussfähigkeit tun kann. Dies
  spiegelt den Status Quo wieder.
\item Änderung der Geschäftsordnungsanträge zur Neuwahl von Protokoll und
  Sitzungsleitung die eine Neuwahl mehrerer Protokollanten oder Mitglieder der
  Sitzungsleitung erlaubt.
\end{enumerate}

\end{document}
