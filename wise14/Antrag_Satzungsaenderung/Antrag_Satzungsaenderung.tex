\documentclass[a4,12pt,oneside]{scrartcl}

% Sprache und Encodings
\usepackage[ngerman]{babel}
\usepackage[T1]{fontenc}
\usepackage[utf8]{inputenc}

% Typographisch interessante Pakete
\usepackage{microtype} % Randkorrektur und andere Anpassungen

% References to Internet and within the document !!!always last package!!!
\usepackage[pdftex,colorlinks=false,
pdftitle={Antrag zur Änderung der Satzung der ZaPF},
pdfauthor={Jörg Behrmann (FUB), Björn Guth (RWTH Aachen)},
pdfcreator={pdflatex},
pdfdisplaydoctitle=true]{hyperref}

% Absaetze nicht Einruecken
\setlength{\parindent}{0pt}
\setlength{\parskip}{2pt}


\begin{document}

\section*{Antrag zur Änderung der Satzung der ZaPF}

\textbf{Antragsteller:} Jörg Behrmann (FUB), Björn Guth (RWTH Aachen)

\subsection*{Antrag}

In §5.2 nach
\begin{quote}
	Sollten ein oder mehrere Posten im StAPF vakant sein, muss im Abschlussplenum der darauf folgenden ZaPF eine Nachbesetzung durchgeführt werden.
\end{quote}
füge
\begin{quote}
	Die nachbesetzte Person bleibt für die Restdauer der Wahlperiode des ausgeschiedenen Mitgliedes im Amt.
\end{quote}
ein.

\subsection*{Begründung}

Diese Änderung regelt die Amtszeit von nachgewählten StAPF-Mitgliedern, die
derzeit unklar ist.
Die Regelung, dass nachgewählte StAPF-Mitglieder nur die Restlaufzeit des
StAPF-Mitgliedes, das sie ersetzen, wahrnehmen, sichert, dass die versetzte Wahl
der StAPF-Mitglieder laut Satzung problemlos erhalten bleibt.

\end{document}
