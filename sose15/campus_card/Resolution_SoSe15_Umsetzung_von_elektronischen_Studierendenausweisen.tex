\documentclass[DIV=calc]{scrartcl}
\usepackage[utf8]{inputenc}
\usepackage[T1]{fontenc}
\usepackage[ngerman]{babel}
\usepackage{graphicx}

\usepackage{fixltx2e}
\usepackage{ellipsis}
\usepackage[tracking=true]{microtype}

\usepackage{lmodern}                        % Ersatz fuer Computer Modern-Schriften
\usepackage{hfoldsty}

\usepackage{fourier} 			% Schriftart
\usepackage[scaled=0.81]{helvet} 	% Schriftart

\usepackage{url}
\usepackage{tocloft} 			% Paket für Table of Contents

\usepackage{xcolor}
\definecolor{urlred}{HTML}{660000}

\usepackage{hyperref}
\hypersetup{
  colorlinks=true,	
  linkcolor=black,	% Farbe der internen Links (u.a. Table of Contents)
  urlcolor=black,	% Farbe der url-links
  citecolor=black} % Farbe der Literaturverzeichnis-Links

\usepackage{mdwlist} 	% Änderung der Zeilenabstände bei itemize und enumerate

\parindent 0pt 				% Absatzeinrücken verhindern
\parskip 12pt 				% Absätze durch Lücke trennen

\usepackage{titlesec}	% Abstand nach Überschriften neu definieren
\titlespacing{\subsection}{0ex}{3ex}{-1ex}
\titlespacing{\subsubsection}{0ex}{3ex}{-1ex}		

% \pagestyle{empty}
\setlength{\textheight}{23cm}
\usepackage{fancyhdr}
\pagestyle{fancy}
\cfoot{}
\lfoot{Zusammenkunft aller Physik-Fachschaften}
\rfoot{www.zapfev.de\\stapf@googlegroups.de}
\renewcommand{\headrulewidth}{0pt}
\renewcommand{\footrulewidth}{0.1pt}


\begin{document}
\hspace{0.87\textwidth}
\begin{minipage}{120pt}
\vspace{-1.8cm}
\includegraphics[width=80pt]{../../logo.pdf}
\centering
\small Zusammenkunft aller Physik-Fachschaften
\end{minipage}
\begin{center}
\huge{Stellungnahme der Zusammenkunft aller Physik-Fachschaften} \\
\normalsize
\end{center}

\vspace{1cm}
\section*{Umsetzung von elektronischen Studierendenausweisen}

In vielen Hochschulen wurden in den letzten Jahren elektronische Studierendenausweise eingeführt
bzw. sind diese in Planung. Grundsätzlich begrüßen wir Vereinfachungen des Studierendenalltags,
folgende Punkte müssen jedoch beachtet werden:

\begin{enumerate}
\itemsep1pt\parskip0pt\parsep0pt
    \item{\textbf{Chipkarte:} Falls in der Karte elektronische Aufzeichnungsmethoden verwendet werden, so sollen folgende Standards gelten:
	\begin{itemize}
	\itemsep1pt\parskip0pt\parsep0pt

	    \item{Auf dem Chip dürfen nur die nötigsten Daten für die angebotenen Funktionalitäten liegen.}
	    \item{Alle Daten der einzelnen Funktionen müssen separat gespeichert, verschlüsselt und verarbeitet werden.}
	    \item{Für die Chipkarte müssen aktuelle und sichere Verfahren verwendet werden. Insbesondere eine nicht authorisierte Datenauslesung muss verhindert werden.}
	    \item{Bei Bekanntwerden von Sicherheitsmängeln müssen die Betroffenen umgehend informiert und das Problem behoben werden.}
	\end{itemize}
    }
    \item{\textbf{Matrikelnummer:} Wenn die Matrikelnummer nicht zwingend auf dem Studierendenausweis für
    organisatorische Abläufe benötigt wird, bevorzugen wir, dass sie dort nicht aufgedruckt wird.}
    \item{\textbf{Lichtbild:} Ein verpflichtendes Lichtbild für die Studierendenausweise soll nur nach Absprache
    mit der Zustimmung der Studierendenvertretung eingeführt werden. Es muss für alle Studierenden
    die Option geben einen Ausweis ohne Lichtbild zu benutzen.}
    \item{\textbf{Bezahlfunktion/Mensa-Karte:} Sollte eine Bezahlfunktion für Mensa, Kopierer, Drucker oder
    ähnliches verwendet werden, soll diese anonym eingerichtet werden. Insbesondere ist uns wichtig,
    dass keine Daten außer den für das Bezahlsystem nötigen aufgezeichnet werden. Die Menge an
    notwendigen persönlichen Daten soll so gering wie möglich gehalten werden.}
    \item{\textbf{Anwesenheitskontrolle:} Wir sind gegen jegliche Möglichkeit zur Kontrolle der Anwesenheit
    einzelner, namentlicher Personen mit den Funktionen des Studierendenausweises. Dies gilt
    sowohl für Lehrveranstaltungen als auch für Tätigkeiten als studentische Hilfskraft. Anonyme
    Kontrollen von sicherheitskritischen Funktionen wie Schließberechtigungen sollen auf das Nötigste
    reduziert sein.}
    \item{\textbf{Prüfungsverwaltung:} Wir lehnen eine Verknüpfung des elektronischen Studierendenausweises und
    der Prüfungsverwaltung (An- und Abmeldung, Noteneinsicht) ab. Die Prüfungsverwaltung sollte
    unabhängig von den elektronischen Funktionen des Studierendenausweises durchführbar sein.
    Damit ist diese unabhängig von einem physischen Medium möglich, welches die Ausfallsicherheit
    erheblich erhöht. Dadurch ist die umständliche Implemetierung einer digitalen Signaturfunktion
    auf der Karte nicht notwendig.}
    \item{\textbf{Semesterticket:} Der Nachweis der Beförderungsberechtigung sollte in den Studierendenausweis
    integriert werden. Dabei sollen wenn möglich nur Methoden verwendet werden, welche das
    Verfolgen einzelner Personen verhindern.}
    \item{\textbf{Bibliothek:} Grundsätzlich befürworten wir eine Nutzung des Studierendenausweis auch als
    Bibliotheksausweis.}
\end{enumerate}

\vfill
\begin{flushright}
Verabschiedet am 31.05.2015 in Aachen
\end{flushright}




\end{document}
