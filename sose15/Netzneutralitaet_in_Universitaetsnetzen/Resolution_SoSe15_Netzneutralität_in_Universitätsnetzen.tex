\documentclass[DIV=calc]{scrartcl}
\usepackage[utf8]{inputenc}
\usepackage[T1]{fontenc}
\usepackage[ngerman]{babel}
\usepackage{graphicx}

\usepackage{fixltx2e}
\usepackage{ellipsis}
\usepackage[tracking=true]{microtype}

\usepackage{lmodern}                        % Ersatz fuer Computer Modern-Schriften
\usepackage{hfoldsty}

\usepackage{fourier} 			% Schriftart
\usepackage[scaled=0.81]{helvet} 	% Schriftart

\usepackage{url}
\usepackage{tocloft} 			% Paket für Table of Contents

\usepackage{xcolor}
\definecolor{urlred}{HTML}{660000}

\usepackage{hyperref}
\hypersetup{
  colorlinks=true,	
  linkcolor=black,	% Farbe der internen Links (u.a. Table of Contents)
  urlcolor=black,	% Farbe der url-links
  citecolor=black} % Farbe der Literaturverzeichnis-Links

\usepackage{mdwlist} 	% Änderung der Zeilenabstände bei itemize und enumerate

\parindent 0pt 				% Absatzeinrücken verhindern
\parskip 12pt 				% Absätze durch Lücke trennen

\usepackage{titlesec}	% Abstand nach Überschriften neu definieren
\titlespacing{\subsection}{0ex}{3ex}{-1ex}
\titlespacing{\subsubsection}{0ex}{3ex}{-1ex}		

% \pagestyle{empty}
\setlength{\textheight}{23cm}
\usepackage{fancyhdr}
\pagestyle{fancy}
\cfoot{}
\lfoot{Zusammenkunft aller Physik-Fachschaften}
\rfoot{www.zapfev.de\\stapf@googlegroups.de}
\renewcommand{\headrulewidth}{0pt}
\renewcommand{\footrulewidth}{0.1pt}


\begin{document}
\hspace{0.87\textwidth}
\begin{minipage}{120pt}
\vspace{-1.8cm}
\includegraphics[width=80pt]{../../logo.pdf}
\centering
\small Zusammenkunft aller Physik-Fachschaften
\end{minipage}
\begin{center}
\huge{Stellungnahme der Zusammenkunft aller Physik-Fachschaften, der Konferenz
der Informatikfachschaften und der Konfrenz der Mathematikfachschaften} \\
\normalsize
\end{center}

\vspace{1cm}
\section*{Resolution zur qualitativen Umsetzung von  eduroam und anderen
  hochschulöffentlichen Netzwerken an allen Hochschulen}

  Die zeichnenden Bundesfachschaftentagungen begrüßen das weit verbreitete
  Angebot von eduroam an deutschsprachigen Hochschulen und halten die
  Qualitätssicherung des eduroam-Netzwerkes für die Arbeit an Hochschulen für
  unerlässlich.

  Wir halten folgende Punkte für besonders kritisch und möchten daher auf diese
  explizit hinweisen:

  \begin{enumerate}
  \item Wir fordern die Einhaltung der eduroam Policy Service Definition,
    festgelegt von der GÉANT Association, in der Version 2.8 vom Juli 2012, da
    in der Vergangenheit von einigen Hochschulen einige Empfehlungen sowie
    Forderungen hierin nicht beachtet wurden.

    Herausheben wollen wir dabei
    \begin{enumerate}
    \item  die Einhaltung der in Abschnitt 6.3.3, Unterpunkt ``Network'',
      aufgeführten Liste der unbedingt anzubietenden Ports. Leider wurden wir
      auf zahlreiche Verstöße gegen diesen Punkt aufmerksam gemacht.

      Wir unterstützen darüber hinaus die Empfehlung, keine
      bzw. möglichst wenige Portrestriktionen vorzunehmen, sowie keine
      Anwendungs- und Abfangproxies zu verwenden.

    \item die Einhaltung der in Abschnitt 6.3.2 festgelegten Unterstützung
      anonymer Authentifizierung. Wir bitten, diese Unterstützung auch in den
      entsprechenden Anleitungen zu dokumentieren.
    \end{enumerate}
  \item Falls Portrestriktionen unumgänglich sind, sollten diese öffentlich
    zugänglich dokumentiert und begründet werden, sowohl für ein- als auch für
    ausgehende Beschränkungen.

    Wir bitten die GÉANT Association, dies in die eduroam Policy Service Definition
    als ``MUST''-Anforderung aufzunehmen.

  \item Aufgrund der herausragenden Bedeutung des eduroam-Netzes für die
    wissenschaftliche Gemeinschaft fordern wir eine ausreichende Ausstattung mit
    personellen und finanziellen Mitteln zur Aufrechterhaltung, zur Verbesserung
    und zum Ausbau des Netzwerkes durch die betreibenden Hochschulen.
  \end{enumerate}

  Wir bitten diese Hinweise analog für andere hochschulöffentliche Netze zu
  beherzigen.

\vfill
\begin{flushright}
Verabschiedet am 17.11.2013 in Wien
\end{flushright}




\end{document}
