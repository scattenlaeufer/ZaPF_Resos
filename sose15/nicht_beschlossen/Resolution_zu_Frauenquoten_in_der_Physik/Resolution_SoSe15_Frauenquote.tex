\documentclass[DIV=calc]{scrartcl}
\usepackage[utf8]{inputenc}
\usepackage[T1]{fontenc}
\usepackage[ngerman]{babel}
\usepackage{graphicx}

\usepackage{fixltx2e}
\usepackage{ellipsis}
\usepackage[tracking=true]{microtype}

\usepackage{lmodern}                        % Ersatz fuer Computer Modern-Schriften
\usepackage{hfoldsty}

\usepackage{fourier} 			% Schriftart
\usepackage[scaled=0.81]{helvet} 	% Schriftart

\usepackage{url}
\usepackage{tocloft} 			% Paket für Table of Contents

\usepackage{xcolor}
\definecolor{urlred}{HTML}{660000}

\usepackage{hyperref}
\hypersetup{
  colorlinks=true,	
  linkcolor=black,	% Farbe der internen Links (u.a. Table of Contents)
  urlcolor=black,	% Farbe der url-links
  citecolor=black} % Farbe der Literaturverzeichnis-Links

\usepackage{mdwlist} 	% Änderung der Zeilenabstände bei itemize und enumerate

\parindent 0pt 				% Absatzeinrücken verhindern
\parskip 12pt 				% Absätze durch Lücke trennen

\usepackage{titlesec}	% Abstand nach Überschriften neu definieren
\titlespacing{\subsection}{0ex}{3ex}{-1ex}
\titlespacing{\subsubsection}{0ex}{3ex}{-1ex}		

% \pagestyle{empty}
\setlength{\textheight}{23cm}
\usepackage{fancyhdr}
\pagestyle{fancy}
\cfoot{}
\lfoot{Zusammenkunft aller Physik-Fachschaften}
\rfoot{www.zapfev.de\\stapf@googlegroups.de}
\renewcommand{\headrulewidth}{0pt}
\renewcommand{\footrulewidth}{0.1pt}


\begin{document}
\hspace{0.87\textwidth}
\begin{minipage}{120pt}
\vspace{-1.8cm}
\includegraphics[width=80pt]{../../logo.pdf}
\centering
\small Zusammenkunft aller Physik-Fachschaften
\end{minipage}
\begin{center}
\huge{Stellungnahme der Zusammenkunft aller Physik-Fachschaften} \\
\normalsize
\end{center}

\vspace{1cm}
\section*{Frauenquote in der Physik}

Diese Resolution orientiert sich an dem §11c des Hochschulzukunftsgesetzes NRW (HZG).

Die Intention dieses Paragraphen ist die geschlechtergerechte Besetzung von akademischen
Gremien wie Senat, Fakultätsrat und Berufungskommission.

Dem Ziel einer geschlechtergerechten Verteilung stehen wir positiv gegenüber, kritisieren
jedoch die im HZG genannten Regelungen, dieses zu gewährleisten. Sie sind unseres
Erachtens nach diesem Ziel nicht förderlich, sondern sogar schädlich.

Im folgenden werden wir den entsprechenden Paragraphen erläutern, unsere Kritik
darlegen und einen Gegenvorschlag unterbreiten, der mit dem §3 des Hochschulrahmengesetzes (HRG)
in Einklang steht.

\begin{quotation}
	\textbf{§3 Gleichberechtigung von Frauen und Männern}\\
	Die Hochschulen fördern die tatsächliche Durchsetzung der Gleichberechtigung von Frauen und Männern
	und wirken auf die Beseitigung bestehender Nachteile hin. Die Aufgaben und Mitwirkungsrechte der
	Frauen- und Gleichstellungsbeauftragten der Hochschulen regelt das Landesrecht.

	\hfill \textit{Hochschulrahmengesetz}
\end{quotation}

Das Ziel dieses Paragraphen ist, wie zuvor bereits erwähnt, die Gleichstellung der
Geschlechter in akademischen Gremien. Versucht wird dies über die absolute Geschlechterparität
umzusetzen. Dieser Ansatz kann Probleme verursachen, da er im Widerspruch
zur freien Entfaltung der einzelnen Person stehen kann. Insbesondere sehen wir einen
Konflikt zwischen dem auf Angehörigen der Minderheit lastenden Druck ein hochschulpolitisches
Amt auszuüben\footnote{Dies gilt gerade bei Fächern mit deutlicher Überrepräsentation eines Geschlechtes.}
und dem Fortschritt des Studiums, der Forschung oder dem Lehrauftrag.

Unter Umständen wird die aktive Partizipation in Gremien den Interessierten mit der
Begründung untersagt, dass sie nicht dem richtigen/gewünschten Geschlecht angehören.
Im selben Zuge werden weniger motivierte Personen in ein Gremium entsandt. Dieser Sachverhalt
fördert nicht die Effizienz und Zuverlässigkeit der akademischen Selbstverwaltung, kann also
nicht im Sinne des Gesetzgebers sein.

Die geschlechtliche Minderheit kann sich durch diese Regelungen auf Grund von überproportionaler
Mehrarbeit beeinträchtigt oder bevormundet fühlen. Im Gegensatz dazu fühlt sich die
jeweils andere Partei benachteiligt, da sie unteranteilig Vertreter stellt\footnote{Beispielsweise
habe eine Physik-Fakultät 50 Dozenten, davon 9 Frauen. Bekanntlich engagieren sich nur bis zu 10\%
beteiligter Personen, so dass eine Frau alle anfallenden Ämter bestreiten müsste;
während im schlimmsten Fall engagierte Männer sich nicht einbringen können.}. Diese mögliche
beidseitige Unzufriedenheit widerspricht dem Ziel der Gleichstellung als Ausdruck sozialer
Gerechtigkeit.

Die Formulierung des Gesetzes ist zusätzlich ungenau gewählt. Gesprochen wird in einigen Teilen von
\glqq{}Geschlechterparität\grqq{} aber in dem Bereich, in dem eine Ausnahmeregelung für mangelnde
\glqq{}Geschlechterparität\grqq{} definiert wird, wird von einer \glqq{}... mindestens
dem [tatsächlichen] Frauenanteil...\grqq{} entsprechenden Quote gesprochen (Geschlechterverteilung
in der Hochschullehrergruppe, wenn ansonsten Parität vorliegt). Dies ermöglicht
ausschließlich weiblichen Hochschullehrerinnen, auf Vergabe eines Platzes zu klagen.

Falls in einem aus verschiedenen Statusgruppen gebildeten Gremium der akademischen
Selbstverwaltung alle Statusgruppen außer der Gruppe der Hochschullehrenden Geschlechterparität
erreichen, steht es nach §11c HZG der Gruppe der Professorinnen und Professoren frei, nur
den prozentualen Anteil des weiblichen Geschlechts als Quote anzunehmen\footnote{Dies bedeutet,
dass die Gruppe der Hochschullehrenden sich von der Paritätsbedingung befreien kann, falls die
anderen Statusgruppen sie erfüllen (oder explizite Argumente existieren).}. Anderen Statusgruppen
ist dieser Weg verschlossen. Diese Sonderstellung der Gruppe der Professorinnen und Professoren
steht in keinerlei Einklang mit der Intention von Gremienarbeit auf Augenhöhe zwischen den
Statusgruppen.

Sollte in einem Gremium eine geschlechterparitätische Besetzung nicht möglich sein, muss es für
diesen Fall \glqq{}sachlich begründete\grqq{} Argumente geben. Sollte eine sachliche
Begründung zwar von der Universität als ausreichend, aber vom Land als nicht ausreichend
klassifiziert werden, so besteht die Möglichkeit, dass das jeweilige Gremium durch
rechtliche Einsprüche aufgelöst und handlungsunfähig gemacht werden kann. Dieser Einspruch
kann auch unter der Annahme einer böswilligen Absicht künstlich herbeigeführt
werden. Um dies zu umgehen sollten einige Ausnahmen beispielhaft definiert werden um
eine Vergleichbarkeit der sachlichen Begründungen zu erhalten.

Da für das Aufstellen der Wahllisten die selben Regelungen gelten, treten obige Probleme
dort analog auf.

Eine mögliche Lösung für die obigen Problem wäre, die Besetzung nicht nach Parität
sondern nach Verhältnissen zu bestimmen. Aus jeder Statusgruppe sollten die Anteile der
Vertreterinnen und Vertreter, entsprechend der tatsächlichen Verhältnisse der Mitglieder der Statusgruppe
bestimmt werden. Kein mindest, kein maximal als Präfix sondern das einfache Verhältnis.

Generell ist es wichtig zu beachten, dass keine Gruppe unbegründete Privilegien oder
Zusatzlasten durch eine übertriebene Reglementierung bekommen sollte.

Gleichstellung ist ein wichtiges Thema und ihre Einrichtung eine der zentralen Aufgaben
der kommenden Jahre. Allerdings ist auf Grund der oben geführten Argumentation
die im HZG verwendete Methode der Parität nicht sinnvoll.

Die ZaPF fordert hiermit alle gesetzgebenden Autoritäten auf, sich über dieses Problem
Gedanken zu machen und unnötige Privilegien und Bevorzugungen zu verhindern.



\vfill
\begin{flushright}
Verabschiedet am 31.05.2015 in Aachen
\end{flushright}




\end{document}
