\documentclass[DIV=calc]{scrartcl}
\usepackage[utf8]{inputenc}
\usepackage[T1]{fontenc}
\usepackage[ngerman]{babel}
\usepackage{graphicx}

\usepackage{fixltx2e}
\usepackage{ellipsis}
\usepackage[tracking=true]{microtype}

\usepackage{lmodern}                        % Ersatz fuer Computer Modern-Schriften
\usepackage{hfoldsty}

\usepackage{fourier} 			% Schriftart
\usepackage[scaled=0.81]{helvet} 	% Schriftart

\usepackage{url}
\usepackage{tocloft} 			% Paket für Table of Contents

\usepackage{xcolor}
\definecolor{urlred}{HTML}{660000}

\usepackage{hyperref}
\hypersetup{
  colorlinks=true,	
  linkcolor=black,	% Farbe der internen Links (u.a. Table of Contents)
  urlcolor=black,	% Farbe der url-links
  citecolor=black} % Farbe der Literaturverzeichnis-Links

\usepackage{mdwlist} 	% Änderung der Zeilenabstände bei itemize und enumerate

\parindent 0pt 				% Absatzeinrücken verhindern
\parskip 12pt 				% Absätze durch Lücke trennen

\usepackage{titlesec}	% Abstand nach Überschriften neu definieren
\titlespacing{\subsection}{0ex}{3ex}{-1ex}
\titlespacing{\subsubsection}{0ex}{3ex}{-1ex}		

% \pagestyle{empty}
\setlength{\textheight}{22,8cm}
\usepackage{fancyhdr}
\pagestyle{fancy}
\cfoot{}
\lfoot{Zusammenkunft aller Physik-Fachschaften\\Konferenz der deutschsprachigen Mathematikfachschaften}
\rfoot{zapfev.de\\die-koma.org}
\renewcommand{\headrulewidth}{0pt}
\renewcommand{\footrulewidth}{0.1pt}

\hyphenation{}

\begin{document}
\begin{minipage}{120pt}
\vspace{-2cm}
\includegraphics[width=100pt]{komaLogo.png}
\centering
\small Konferenz der deutschsprachigen Mathematikfachschaften
\end{minipage}
%\hspace{0.5\textwidth}
\hfill
\begin{minipage}{120pt}
\vspace{-2cm}
\includegraphics[width=80pt]{zapf_logo.pdf}
\centering
\small Zusammenkunft aller Physik-Fachschaften
\end{minipage}
\begin{center}
\huge{Positionspapier der Zusammenkunft aller Physik-Fachschaften und der Konferenz der deutschsprachigen Mathematikfachschaften} \\
\normalsize
\end{center}

%\vspace{1cm}
\section*{Empfehlungen für einen guten Übungsbetrieb}
%\vspace{1cm}
Übungen sind eine zentrale Lehrveranstaltungsform in mathematisch"=naturwissenschaftlichen Fächern. Hier wird die Fähigkeit, analytisch und systematisch Problemstellungen anzugehen und zu lösen, geschult. Diese Kompetenz gilt als eine der zentralen Fähigkeiten von Absolventinnen und Absolventen mathematisch-naturwissenschaftlicher Fächer.

\subsection*{Organisatorische Rahmenbedingungen}
Übungen bieten die Möglichkeit, Inhalte einer Vorlesung zu vertiefen und zu festigen. Daher sollten Übungsinhalte gut mit der Vorlesung abgestimmt sein. Ein intensives Betreuungsverhältnis bildet die Grundlage für eine erfolgreiche Übung, daher sollten die Übungsgruppen eine angemesse Größe haben.

Für die Übungen sollten Aufgaben in Form von Übungsblättern zur Bearbeitung außerhalb der Übungstermine zur Verfügung stehen. Darüberhinaus ermöglichen Präsenzaufgaben ein breites Spektrum von Herangehensweisen und Lösungsansätzen anzubieten und Verständnisprobleme aufzudecken.
Die Übungsblätter sollten zur Korrektur eingereicht werden können. Die korrigierten Lösungen sollten den Studierenden schnellstmöglich zur Verfügung gestellt werden, damit aufgetretene Fehler und Defizite nachgearbeitet werden können.

Zu den Übungsblättern sollten zeitnah Lösungsskizzen zur Verfügung gestellt werden, um die Nachbereitung der Übungen und die Vorbereitung auf Prüfungen zu unterstützen.
Ebenso sollten die Übungsblätter Aufgaben beinhalten, die in Stil und Niveau auf die Prüfung vorbereiten. Dies sorgt für Transparenz in Bezug auf die Prüfungsanforderungen.

Des Weiteren empfehlen wir, den Übungsgruppenleiter*innen eine didaktische Schulung anzubieten, um sie auf ihre Aufgaben vorzubereiten.

\subsection*{Struktur der Übungsaufgaben}
Auf einem Übungsblatt sollten Aufgaben mit unterschiedlichem Schwierigkeitsgrad bzw. Arbeitsaufwand vorhanden sein.
Das Konzept von zusätzlichen Aufgaben auf freiwilliger Basis stellt eine weitere Vertiefung oder Erweiterung des in der Vorlesung behandelten Stoffes dar. Die zum Lösen der regulären Aufgaben benötigten Inhalte sollten in der Vorlesung behandelt worden sein.\footnote{Die ZaPF möchte hierbei klarstellen, dass dies Transferleistungen nicht ausschließt. \textit{(Diese Fußnote wurde auf dem Abschlussplenum der ZaPF hinzugefügt und konnte somit nicht vom Abschlussplenum der KoMa abgestimmt werden.)}}

\emph{Beispiele für Best-Practice:}
\begin{itemize}
	\item Bei Bedarf können Literaturverweise zur Thematik einer Übungsaufgabe angegeben werden, um z.~B. bei Problemen oder weitergehendem Interesse, die Möglichkeit zum Selbststudium anzubieten.
	\item Praktikabel ist weiterhin am Ende der jeweiligen Übungsaufgabe das Lernziel zu formulieren, um eine Selbstreflexion der Studierenden verstärkt zu ermöglichen.
\end{itemize}

\subsection*{Übungsablauf}
Im prinzipiellen Ablauf einer Übung sollte ausreichend Zeit für fachliche und aufgabenbezogene Diskussionen eingeplant sein. Die aufgeworfenen Fragen in fachliche Diskussionsbahnen zu lenken, gehört unseres Erachtens in das Aufgabengebiet von Übungsgruppenleiter*innen genauso, wie die Moderation und die Motivation zu entsprechenden Diskussionen (z.~B. mithilfe von Kontroll- bzw. Verständnisfragen). Dies fördert den Dialog zwischen Studierenden und Übungsgruppenleiter*innen und schafft ein kollegiales Verhältnis, welches der Atmosphäre und dem Lernklima zugutekommt. Eine solche Beziehung ermöglicht zudem eine konstruktive, gegenseitige Kritikbehandlung. In diesem Zusammenhang kann individueller auf gehäufte Fehler der Übungsgruppe eingegangen werden.

Zusätzlich zu den regulären Übungen sind offene und durch Tutor*innen betreute Übungsräume (z.~B. Übungsflure/-labore, Lernzentren) eine sinnvolle Ergänzung. In diesen Zeiträumen stehen die Tutor*innen den Studierenden bei deren Problemen, Fragen und Verständnisschwierigkeiten helfend zur Seite.

\vspace{1cm}
Diese Empfehlungen sind als Anregung zur Auseinandersetzung mit und Weiterentwicklung von vorhandenen Übungskonzepten zu verstehen.
%\pagebreak
%\renewcommand{\headrulewidth}{0.1pt}
%\lhead{Positionspapier von ZaPF \& KoMa: \emph{Empfehlungen f. einen guten Übungsbetrieb}}
%\rhead{\thepage}

\vfill
\begin{flushright}
Verabschiedet durch die KoMa am 30.~Mai 2015 in Aachen\\
Verabschiedet durch die ZaPF am 31.~Mai 2015 in Aachen
\end{flushright}

\end{document}
