\documentclass[12pt,oneside]{scrartcl}

% Sprache und Encodings
\usepackage[ngerman]{babel}
\usepackage[T1]{fontenc}
\usepackage[utf8]{inputenc}

% Typographisch interessante Pakete
\usepackage{microtype} % Randkorrektur und andere Anpassungen

% References to Internet and within the document
%\usepackage[pdftex,colorlinks=false,
%pdftitle={Resolution für mehr Lebensqualität},
%pdfauthor={Björn Guth (Aachen), Jörg Behrmann (FUB), Wolfgang Bauer (Alter Sack) und der Rest des Git-Workshops},
%pdfcreator={pdflatex},
%pdfdisplaydoctitle=true]{hyperref}

% Absaetze nicht Einruecken
\setlength{\parindent}{0pt}
\setlength{\parskip}{2pt}

% Formatierung auf A4 anpassen
\usepackage{geometry}
\geometry{paper=a4paper,left=20mm,right=20mm,top=10mm,bottom=10mm}

\begin{document}

\section*{Resolution für mehr Lebensqualität}

\textbf{Adressaten:} Zukünftige ZaPFen ausrichtende Fachschaften.

\subsection*{Antrag:}
Die ZaPF möge beschließen:
\begin{quote}
Die ZaPF spricht ihr wohlwollendes Interesse an ewigen Frühstücken
aussprechen aus. Weiterhin wäre es wünschenswert, die Möglichkeit der Einreichungen von
Anträgen über Git zu realisieren. Darüber hinaus soll in Zukunft auch Tee
gleichberechtigt mit Kaffee innerhalb der Arbeitskreise verteilt werden.
\end{quote}

\subsection*{Begründung:}
Viele ZaPFika besuchen andere BuFaTas, darunter KIF und KOMA.
KIF und KOMA zeichnen sich unter anderem durch die Tradition des ewigen
Frühstücks aus, dabei wird die ganze Tagung über ein Frühstücksbuffet
angeboten und ständig befüllt gehalten.

Das Managen der auf einer ZaPF entstehenden Anträgen hat sowohl für Teilnehmer,
als auch für die ausführende Fachschaft einige Vorteile. Zum einen wird so das
Einreichen von Beschlüssen siginifikant vereinfacht, da dies über ein Pull-Request
getan werden kann, zum anderen hat das Tagungsbüro auf diese Weise immer Zugriff
auf die aktuellste Version des Antrags. Es sei hier explizit darauf hingewiesen,
dass dies nicht den aktuellen Modus des Aushängens von Anträgen vor dem Tagungsbüro
ersetzen soll, sondern nur die Wege, wie ein Antrag zum Tagungsbüro kommt,
erweitert.

Einige ZaPFika mögen trotz des großen Schlafmangels keinen Kaffee. Um diesen
trotzdem die Möglichkeit eines Heißgetränkes nicht zu nehmen, sollte auch heißes
Wasser und Teebeutel verteilt werden.

\vspace{1cm}
\textbf{Verfasser:} Björn Guth (Aachen), Jörg Behrmann (FUB), Wolfgang Bauer (Alter Sack)
und der Rest des Git-Workshops

\end{document}
