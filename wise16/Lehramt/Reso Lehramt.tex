\documentclass[DIV=calc]{scrartcl}
\usepackage[utf8]{inputenc}
\usepackage[T1]{fontenc}
\usepackage[ngerman]{babel}
\usepackage{graphicx}

\usepackage{fixltx2e}
\usepackage{ellipsis}
\usepackage[tracking=true]{microtype}

\usepackage{lmodern}      % Ersatz fuer Computer Modern-Schriften
\usepackage{hfoldsty}

\usepackage{fourier} 			% Schriftart
\usepackage[scaled=0.81]{helvet} 	% Schriftart

\usepackage{url}
\usepackage{tocloft} 			% Paket für Table of Contents

\usepackage{xcolor}
\definecolor{urlred}{HTML}{660000}

\usepackage{hyperref}
\hypersetup{
 colorlinks=true,	
 linkcolor=black,	% Farbe der internen Links (u.a. Table of Contents)
 urlcolor=black,	% Farbe der url-links
 citecolor=black} % Farbe der Literaturverzeichnis-Links

\usepackage{mdwlist} 	% Änderung der Zeilenabstände bei itemize und enumerate

\parindent 0pt 				% Absatzeinrücken verhindern
\parskip 12pt 				% Absätze durch Lücke trennen

%\usepackage{titlesec}	% Abstand nach Überschriften neu definieren
%\titlespacing{\subsection}{0ex}{3ex}{-1ex}
%\titlespacing{\subsubsection}{0ex}{3ex}{-1ex}		

% \pagestyle{empty}
\setlength{\textheight}{23cm}
\usepackage{fancyhdr}
\pagestyle{fancy}
\cfoot{}
\lfoot{Zusammenkunft aller Physik-Fachschaften}
\rfoot{www.zapfev.de\\stapf@zapf.in}
\renewcommand{\headrulewidth}{0pt}
\renewcommand{\footrulewidth}{0.1pt}


\begin{document}
\hspace{0.87\textwidth}
\begin{minipage}{120pt}
\vspace{-1.8cm}
\includegraphics[width=80pt]{logo.pdf}
\centering
\small Zusammenkunft aller Physik-Fachschaften
\end{minipage}
\begin{center}
\huge{Resolution der Zusammenkunft aller Physik-Fachschaften} \\
\normalsize
\end{center}

\vspace{1cm}
%\section*{Reso Lehramt (hat keinen Titel)}

Ohne die Empfehlung der ZaPF und der jDPG zur Ausgestaltung der Lehramtstudiengänge im Fach Physik (verabschiedet am 16.05.2010 in Frankfurt)\footnote{\url{http://www.zapfev.de/resolutionen/sose10/Lehramtstellungnahme.pdf}}, dass an jeder Universität, die Physiklehrerinnen und -lehrer ausbildet, mindestens eine Fachdidaktikprofessur existieren soll, in Frage zu stellen, korrigiert die ZaPF ihre Stellungnahme zu Fachdidaktikprofessuren (verabschiedet am 17.11.2013 in Wien)\footnote{\url{http://www.zapfev.de/resolutionen/wise13/Reso_WiSe13_Fachdidaktikprofessuren.pdf}} um folgende Punkte:

Das bestehende Verhältnis zwischen der Vermittlung der Anwendung und der Weiterentwicklung der Fachdidaktik soll zu Gunsten der Vermittlung an die Lehramtstudierenden angepasst werden.

Für die Berufung als Hochschullehrerin oder Hochschullehrer der Fachdidaktik sieht die ZaPF eine abgeschlossene Promotion als unabdingbar. Diese soll im fachdidaktischen Bereich erfolgt sein. Darüber hinaus schätzt die ZaPF eine angemessene Praxiserfahrung als notwendig ein. Diese soll ca. 5 Jahre betragen und kann selbstverantwortlichen Unterricht an der Schule, das Ableisten des Vorbereitungsdienstes, Tätigkeit im Schulbuchverlag etc. umfassen.

Um einen aktuellen Praxisbezug zu gewährleisten und die fachdidaktische Forschung am konkreten Fall zu evaluieren, empfiehlt die ZaPF, dass Fachdidaktikprofessorinnen und -professoren einen Teil ihrer Arbeit als Schulunterricht einbringen.

\vfill
\begin{flushright}
Verabschiedet am 13.11.2016 in Dresden
\end{flushright}




\end{document}
