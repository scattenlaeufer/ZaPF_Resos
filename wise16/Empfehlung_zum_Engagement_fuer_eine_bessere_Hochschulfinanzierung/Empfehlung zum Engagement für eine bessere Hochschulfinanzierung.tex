\documentclass[DIV=calc]{scrartcl}
\usepackage[utf8]{inputenc}
\usepackage[T1]{fontenc}
\usepackage[ngerman]{babel}
\usepackage{graphicx}

\usepackage{fixltx2e}
\usepackage{ellipsis}
\usepackage[tracking=true]{microtype}

\usepackage{lmodern}      % Ersatz fuer Computer Modern-Schriften
\usepackage{hfoldsty}

\usepackage{fourier} 			% Schriftart
\usepackage[scaled=0.81]{helvet} 	% Schriftart

\usepackage{url}
\usepackage{tocloft} 			% Paket für Table of Contents

\usepackage{xcolor}
\definecolor{urlred}{HTML}{660000}

\usepackage{hyperref}
\hypersetup{
 colorlinks=true,	
 linkcolor=black,	% Farbe der internen Links (u.a. Table of Contents)
 urlcolor=black,	% Farbe der url-links
 citecolor=black} % Farbe der Literaturverzeichnis-Links

\usepackage{mdwlist} 	% Änderung der Zeilenabstände bei itemize und enumerate

\parindent 0pt 				% Absatzeinrücken verhindern
\parskip 12pt 				% Absätze durch Lücke trennen

%\usepackage{titlesec}	% Abstand nach Überschriften neu definieren
%\titlespacing{\subsection}{0ex}{3ex}{-1ex}
%\titlespacing{\subsubsection}{0ex}{3ex}{-1ex}		

% \pagestyle{empty}
\setlength{\textheight}{23cm}
\usepackage{fancyhdr}
\pagestyle{fancy}
\cfoot{}
\lfoot{Zusammenkunft aller Physik-Fachschaften}
\rfoot{www.zapfev.de\\stapf@zapf.in}
\renewcommand{\headrulewidth}{0pt}
\renewcommand{\footrulewidth}{0.1pt}


\begin{document}
\hspace{0.87\textwidth}
\begin{minipage}{120pt}
\vspace{-1.8cm}
\includegraphics[width=80pt]{logo.pdf}
\centering
\small Zusammenkunft aller Physik-Fachschaften
\end{minipage}
\begin{center}
\huge{Resolution der Zusammenkunft aller Physik-Fachschaften} \\
\normalsize
\end{center}

\vspace{1cm}
\section*{Empfehlung zum Engagement für eine bessere Hochschulfinanzierung}

Die Hochschulen in Deutschland leiden schon seit einigen Jahren unter einer starken Unterfinanzierung. Fast jährlich finden Demonstrationen wegen mangelnder Finanzierung sowohl von Seiten der Studierenden als auch von Seiten der Hochschulen statt. 

Abgeordnete und Ministerien erachten Themen allerdings am ehesten als wichtig, wenn möglichst viele einzelne Anfragen bei ihnen eingehen. Denn obwohl Demonstrationen eine große Öffentlichkeit herstellen, können viele kleine Aktionen und wiederholte Anfragen auch eine ähnliche Wirkung beziehungsweise ein Bewusstsein bei den Entscheidungsträger*innen erzielen. Dies wird hauptsächlich durch die stetige Wiederholung von vielen Anfragen erreicht. Wir zeigen damit, dass uns dieses Thema sehr wichtig ist. 

Um die Situation an den Hochschulen zu verbessern, schlägt die Zusammenkunft aller Physikfachschaften deshalb folgende Handlungsmöglichkeiten vor: 
\begin{itemize}
\item Einen öffentlichen (Foto)Wettbewerb über die schlimmsten Ecken der Uni, Postkartenaktion 
\item Eine Mängelliste (überfüllte Hörsäle, zu wenig Seminare, kaputte Bibliothek, keine Arbeitsplätze, Barrierefreiheit, kurze Öffnungszeiten durch fehlendes Personal,...) aufstellen 
\item Die/den örtlichen MdL/MdB einladen und über die Probleme reden 
\item Lokale Zeitungen mit den Problemen anschreiben 
\item Das zuständige Wissenschaftsministerium anschreiben 
\item Das Rektorat/die Studierendenvertretung anschreiben und bitten, das Gleiche zu machen 
\end{itemize}
Die Zusammenkunft aller Physikfachschaften bittet darum die Handlungsvorschläge an möglichst viele Fachschaften und Studierendenvertretungen zu versenden und begrüßt es, wenn unser Aufruf von möglichst vielen verschiedenen Stellen umgesetzt wird. 

\vfill
\begin{flushright}
Verabschiedet am 13.11.2016 in Dresden
\end{flushright}




\end{document}
