\documentclass[DIV=calc]{scrartcl}
\usepackage[utf8]{inputenc}
\usepackage[T1]{fontenc}
\usepackage[ngerman]{babel}
\usepackage{graphicx}

\usepackage{fixltx2e}
\usepackage{ellipsis}
\usepackage[tracking=true]{microtype}

\usepackage{lmodern}                        % Ersatz fuer Computer Modern-Schriften
\usepackage{hfoldsty}

\usepackage{fourier}             % Schriftart
\usepackage[scaled=0.81]{helvet}     % Schriftart

\usepackage{url}
\usepackage{tocloft}             % Paket für Table of Contents

\usepackage{xcolor}
\definecolor{urlred}{HTML}{660000}

\usepackage{hyperref}
\hypersetup{
    colorlinks=true,    
    linkcolor=black,    % Farbe der internen Links (u.a. Table of Contents)
    urlcolor=black,    % Farbe der url-links
    citecolor=black} % Farbe der Literaturverzeichnis-Links

\usepackage{mdwlist}     % Änderung der Zeilenabstände bei itemize und enumerate

\parindent 0pt                 % Absatzeinrücken verhindern
\parskip 12pt                 % Absätze durch Lücke trennen

%\usepackage{titlesec}    % Abstand nach Ãœberschriften neu definieren
%\titlespacing{\subsection}{0ex}{3ex}{-1ex}
%\titlespacing{\subsubsection}{0ex}{3ex}{-1ex}        

% \pagestyle{empty}
\setlength{\textheight}{23cm}
\usepackage{fancyhdr}
\pagestyle{fancy}
\cfoot{}
\lfoot{Zusammenkunft aller Physik-Fachschaften}
\rfoot{www.zapfev.de\\stapf@zapf.in}
\renewcommand{\headrulewidth}{0pt}
\renewcommand{\footrulewidth}{0.1pt}
\newcommand{\gen}{*innen}

\begin{document}
    \hspace{0.87\textwidth}
    \begin{minipage}{120pt}
        \vspace{-1.8cm}
        \includegraphics[width=80pt]{logo.pdf}
        \centering
        \small Zusammenkunft aller Physik-Fachschaften
    \end{minipage}
    \begin{center}
        \huge{Offener Brief der Zusammenkunft aller Physik-Fachschaften} \\
        \normalsize
    \end{center}
    
    \vspace{1cm}
    \section*{Offener Brief zum Rahmenvertrag zwischen der VG Wort und der Kultusministerkonferenz}
    
   Die Kultusministerkonferenz hat zusammen mit der VG Wort einen Rahmenvertrag\footnote{ \url{http://www.bibliotheksverband.de/fileadmin/user_upload/DBV/vereinbarungen/Rahmenvertrag __54c_181213_unterschrieben.pdf}} über die Vergütung für die Benutzung von urheberrechtlich geschützten Texten zu Unterrichtszwecken abgeschlossen. Dieser gilt ab dem 01.01.2017 und ersetzt das bis dahin geltende pauschale Abrechnungsverfahren. Die Hochschulen haben einzeln die Möglichkeit, diesem Rahmenvertrag beizutreten und jede Benutzung von geschützten Texten einzeln abzurechnen oder komplett auf sie zu verzichten. \\   
Die Zusammenkunft aller deutschsprachigen Physikfachschaften empfiehlt den Hochschulen, dem Rahmenvertrag nicht beizutreten. \\
Wie das Pilotprojekt der Universität Osnabrück\footnote{ \url{https://www.virtuos.uni-osnabrueck.de/forschung/projekte/pilotprojekt_zum_52a_urhg.html}} zeigt, hat die Einzelabrechnung erhebliche Nachteile gegenüber einer kaum mit Verwaltungsaufwand verbundenen Pauschalabrechnung. Nicht nur, dass die Verwaltungskosten für die Einzelabrechnung die reinen Lizenzgebühren um das Fünffache übersteigen, auch die Qualität von Studium und Lehre nimmt spürbar ab. Ganz abzusehen von dem enormen Mehraufwand für die Lehrenden. \\
Gerade zu Zeiten der fortschreitenden Digitalisierung der Gesellschaft und der starken Förderung des eLearnings ist es uns unbegreiflich, wie die VG Wort eine Einzelabrechnung über die Nutzung von urheberrechtlich geschütztem Material fordern kann. Dass dies auch anders geht, zeigen alle anderen Verwertungsgesellschaften. Diese haben auch weiterhin Pauschalverträge mit den Bundesländern abgeschlossen.
Wir befürworten eine gerechte Bezahlung der Autoren, jedoch sollte dies nicht zum Preis eines deutlich erhöhten Verwaltungsaufwands geschehen, denn dieser fährt zu einer erheblichen Verschlechterung der Lehre.  \\
Wir rufen deshalb die VG Wort auf, wie bisher auch, eine pauschale Abrechnung zu ermöglichen!
    
    \vfill
    \begin{flushright}
        Verabschiedet am 12.11.2016 in Dresden
    \end{flushright}
    
    
    
    
\end{document}
