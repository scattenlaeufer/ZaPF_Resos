\documentclass[draft,10pt,oneside]{scrartcl}

% Sprache und Encodings
\usepackage[ngerman]{babel}
\usepackage[T1]{fontenc}
\usepackage[utf8]{inputenc}

% Typographisch interessante Pakete
\usepackage{microtype} % Randkorrektur und andere Anpassungen

% References to Internet and within the document
\usepackage[pdftex,colorlinks=false,
pdftitle={Antrag zur Änderung der Geschäftsordnung für Plenen der ZaPF},
pdfauthor={Jörg Behrmann (FUB), Björn Guth (RWTH)},
pdfcreator={pdflatex},
pdfdisplaydoctitle=true]{hyperref}

% Absaetze nicht Einruecken
\setlength{\parindent}{0pt}
\setlength{\parskip}{2pt}

% Formatierung auf A4 anpassen
\usepackage{geometry}
\geometry{paper=a4paper,left=15mm,right=15mm,top=10mm,bottom=10mm}

\begin{document}

\section*{Antrag zur Änderung der Geschäftsordnung für Plenen der ZaPF}

\textbf{Antragsteller:} Jörg Behrmann (FUB), Björn Guth (RWTH)

\subsection*{Antrag}

Hiermit beantragen wir die Geschäftsordnung für Plenen der ZaPF wie folgend zu
ändern:

In 2.7 ersetze
\begin{quote}
	Auf einer vorherigen ZaPF durch einen GO-Antrag auf \glqq{}Schließung der
	Redeliste und Verweisung in eine Arbeitsgruppe mit Recht auf ein
	Meinungsbild im Plenum\grqq{} vertagten Anträge sollen priorisiert
	behandelt werden.
\end{quote}
durch
\begin{quote}
	Auf einer vorherigen ZaPF durch einen GO-Antrag auf \glqq{}Schließung der
	Redeliste und Verweisung in eine Arbeitsgruppe mit Recht auf ein
	Meinungsbild im Plenum\grqq{} vertagte Anträge sowie solche, die wegen
	mangelnder Beschlussfähigkeit, nicht mehr behandelt werden konnten, sollen
	priorisiert behandelt werden.
\end{quote}

\subsection*{Begründung}
Diese Änderung fügt auch passiv vertagte Anträge zur Priorisierung für das
nächste Planum hinzu.

\end{document}
