\documentclass[DIV=calc]{scrartcl}
\usepackage[utf8]{inputenc}
\usepackage[T1]{fontenc}
\usepackage[ngerman]{babel}
\usepackage{graphicx}

\usepackage{fixltx2e}
\usepackage{ellipsis}
\usepackage[tracking=true]{microtype}

\usepackage{lmodern}      % Ersatz fuer Computer Modern-Schriften
\usepackage{hfoldsty}

\usepackage{fourier} 			% Schriftart
\usepackage[scaled=0.81]{helvet} 	% Schriftart

\usepackage{url}
\usepackage{tocloft} 			% Paket für Table of Contents

\usepackage{xcolor}
\definecolor{urlred}{HTML}{660000}

\usepackage{hyperref}
\hypersetup{
 colorlinks=true,	
 linkcolor=black,	% Farbe der internen Links (u.a. Table of Contents)
 urlcolor=black,	% Farbe der url-links
 citecolor=black} % Farbe der Literaturverzeichnis-Links

\usepackage{mdwlist} 	% Änderung der Zeilenabstände bei itemize und enumerate

\parindent 0pt 				% Absatzeinrücken verhindern
\parskip 12pt 				% Absätze durch Lücke trennen

%\usepackage{titlesec}	% Abstand nach Überschriften neu definieren
%\titlespacing{\subsection}{0ex}{3ex}{-1ex}
%\titlespacing{\subsubsection}{0ex}{3ex}{-1ex}		

% \pagestyle{empty}
\setlength{\textheight}{23cm}
\usepackage{fancyhdr}
\pagestyle{fancy}
\cfoot{}
\lfoot{Zusammenkunft aller Physik-Fachschaften}
\rfoot{www.zapfev.de\\stapf@zapf.in}
\renewcommand{\headrulewidth}{0pt}
\renewcommand{\footrulewidth}{0.1pt}


\begin{document}
\hspace{0.87\textwidth}
\begin{minipage}{120pt}
\vspace{-1.8cm}
\includegraphics[width=80pt]{logo.pdf}
\centering
\small Zusammenkunft aller Physik-Fachschaften
\end{minipage}
\begin{center}
\huge{Resolution der Zusammenkunft aller Physik-Fachschaften} \\
\normalsize
\end{center}

\vspace{1cm}
\section*{Resolution zur Verwendung von Taschenrechnern in der Schule}
Die ZaPF schließt sich der Resolution der KoMa (siehe unten) mit folgendem Zusatz an:

Der vereinzelte, gezielte Einsatz von fortgeschrittenen Taschenrechnern oder sinnvollerweise
entsprechenden Computerprogrammen liegt jedoch in der didaktischen Diskretion der einzelnen
Lehrkräfte.

\subsection*{Resolution der KoMa:}
\begin{addmargin}[1em]{2em}
„In den verschiedenen Bundesländern ist durch die Landesbildungsministerien der
Einsatz von Taschenrechnern in Schulunterricht und Zentralabitur vorgeschrieben. Die Wahl
des jeweiligen Systems und Gerätemodells bleibt den Schulen überlassen. Wir
unterscheiden im Folgenden verschiedene Taschenrechner anhand ihrer Funktionalität und
trennen dabei zwischen einfachen und fortschrittlichen Taschenrechnern1.
Wir fordern die Landesbildungsministerien dazu auf, bei zentralen Abiturprüfungen die
Nutzung von fortschrittlichen Taschenrechnern nicht verpflichtend, sondern für Schüler
optional anzubieten, um die Nutzung und Abhängigkeit im Unterricht zu verringern und
dem momentanen Trend der stärkeren Nutzung entgegen zu wirken. Die wachsende
Diskrepanz zwischen Schulmathematik und Hochschulmathematik soll somit reduziert
werden. Weiter soll die Anschaffung der Taschenrechner im Sinne der Lehrmittelfreiheit den
Schulen obliegen.

Grundsätzlich sind wir der Meinung, dass diese Geräte nicht pauschal aus dem
Unterricht ausgeschlossen werden sollten: Sie können unterstützend und zur
Veranschaulichung von Sachverhalten vom Lehrer eingesetzt werden, um den Schülern
einen sinnvollen Umgang mit fortschrittlichen Taschenrechnern beizubringen, deren
Medienkompetenz zu stärken und Abwechslung in die Schulmethodik zu bringen.
Die Bedienung von fortschrittlichen Taschenrechnern eignet sich jedoch nicht, um das
grundlegende Erlernen von mathematischen Methoden zu ersetzen; fortschrittliche
Taschenrechner sollten daher frühestens in der Oberstufe verwendet werden.
Unsere Forderungen erachten wir aus folgenden Gründen als notwendig: Aufgrund der
Nutzung verschiedener Modelle mit unterschiedlichem Funktionsumfang entstehen von
Schule zu Schule Unterschiede in der Behandlung des Lehrstoffes. Dies wirkt somit dem
ursprünglichen Ziel des Zentralabiturs, der landesweiten Vergleichbarkeit, entgegen. Zudem
sorgt die Nutzung unterschiedlicher Geräte dafür, dass an verschiedenen Schulen
unterschiedlicher kognitiver Aufwand für die gestellten Aufgaben verlangt wird. Zusätzlich
benötigen Lehrer regelmäßige Fortbildungen; viele Lehrbücher sind auf spezifische Geräte
ausgelegt. Dies kann die Effizienz des Unterrichts verringern.

Des Weiteren wird an den Schulen in unregelmäßigen Abständen ein Wechsel auf
neuere Geräte durchgeführt. In der Praxis lassen Lehrer nur die von der Schule
vorgegebenen Geräte zu, um nicht jedes Modell auf seinen Funktionsumfang prüfen zu
müssen. Dies kann insbesondere in einkommensschwachen Familien oder Familien mit
mehreren Kindern zu finanziellen Problemen führen, da die bereits vorhandenen Geräte
nicht wiederverwendet werden können, sondern weitere zum Teil teure Modelle mit nahezu
identischem Funktionsumfang bestellt werden müssen.

Darüber hinaus sehen wir in der momentan steigenden Nutzung von fortschrittlichen
Taschenrechnern das Problem, dass dies den Fokus vom Lernen der mathematischen
Prinzipien hinweg bewegt. Grafisches Lösen wird häufig dem analytischen Weg
vorgezogen. Der Rechenweg gerät in den Hintergrund und wird von den Schülern nicht
weiter durchdacht, wodurch das problemlösende Denken nicht mehr in ausreichendemUmfang gefördert und gefordert wird. Dies sorgt dafür, dass das allgemeine Verständnis der
Mathematik nachlässt, wodurch der Einstieg in ein mathematisch geprägtes Studium extrem
erschwert wird.

Um zukünftige Studenten optimal auf den bereits sehr großen Sprung zum Studium
vorzubereiten, ist es erforderlich, dass die an den Hochschulen vorausgesetzten
Kompetenzen möglichst häufig geübt und intensiv im schulischen Kontext vermittelt
werden. Diese Kompetenzen nur mittels fortschrittlicher Taschenrechner anwenden zu
können ist für ein Hochschulstudium nicht ausreichend. Die eigene Erarbeitung einer
Lösung wird gefordert, in Konsequenz sind in den meisten Prüfungen an Hochschulen keine
fortschrittlichen Taschenrechner zugelassen; oftmals sind nicht einmal einfache
Taschenrechner erlaubt. Um eine hohe Qualifikation künftiger Studenten mathematisch
geprägter Fächer zu gewährleisten und auch den Umstieg an eine andere Schule nicht
zusätzlich zu erschweren, fordern wir die Umsetzung obig genannter Aspekte.
1 Unter einfachen Taschenrechnern verstehen wir solche, die nur die Grundrechenarten
und Prozentrechnung sowie elementare Funktionen beherrschen. Zu den fortschrittlichen
Taschenrechnern zählen wir:

- Wissenschaftliche Taschenrechner (WTR), unter denen wir Taschenrechner verstehen, die
über Standardberechnungen hinaus komplexere numerische Berechnungen wie z.B.
Nullstellenbestimmung, Matrizenrechnung etc. beherrschen.

- Grafikfähige Taschenrechner (GTR), unter denen wir Taschenrechner verstehen, welche
Funktionen, Daten, Folgen etc. visuell darstellen und mit diesen Darstellungen arbeiten
können.

- Computer-Algebra-Systeme (CAS), unter denen wir Taschenrechner verstehen, die
analytische Methoden z.B. zur Umformung von Termen, Lösung von Gleichungen,
Bestimmung von Ableitungen und Integralen etc. beherrschen.“
\end{addmargin}




\vfill
\begin{flushright}
Verabschiedet am 13.11.2016 in Dresden
\end{flushright}




\end{document}
