\documentclass[DIV=calc]{scrartcl}
\usepackage[utf8]{inputenc}
\usepackage[T1]{fontenc}
\usepackage[ngerman]{babel}
\usepackage{graphicx}

\usepackage{fixltx2e}
\usepackage{ellipsis}
\usepackage[tracking=true]{microtype}

\usepackage{lmodern}      % Ersatz fuer Computer Modern-Schriften
\usepackage{hfoldsty}

\usepackage{fourier} 			% Schriftart
\usepackage[scaled=0.81]{helvet} 	% Schriftart

\usepackage{url}
\usepackage{tocloft} 			% Paket für Table of Contents

\usepackage{xcolor}
\definecolor{urlred}{HTML}{660000}

\usepackage{hyperref}
\hypersetup{
 colorlinks=true,	
 linkcolor=black,	% Farbe der internen Links (u.a. Table of Contents)
 urlcolor=black,	% Farbe der url-links
 citecolor=black} % Farbe der Literaturverzeichnis-Links

\usepackage{mdwlist} 	% Änderung der Zeilenabstände bei itemize und enumerate

\parindent 0pt 				% Absatzeinrücken verhindern
\parskip 12pt 				% Absätze durch Lücke trennen

%\usepackage{titlesec}	% Abstand nach Überschriften neu definieren
%\titlespacing{\subsection}{0ex}{3ex}{-1ex}
%\titlespacing{\subsubsection}{0ex}{3ex}{-1ex}		

% \pagestyle{empty}
\setlength{\textheight}{23cm}
\usepackage{fancyhdr}
\pagestyle{fancy}
\cfoot{}
\lfoot{Zusammenkunft aller Physik-Fachschaften}
\rfoot{www.zapfev.de\\stapf@zapf.in}
\renewcommand{\headrulewidth}{0pt}
\renewcommand{\footrulewidth}{0.1pt}


\begin{document}
\hspace{0.87\textwidth}
\begin{minipage}{120pt}
\vspace{-1.8cm}
\includegraphics[width=80pt]{logo.pdf}
\centering
\small Zusammenkunft aller Physik-Fachschaften
\end{minipage}
\begin{center}
\huge{Resolution der Zusammenkunft aller Physik-Fachschaften} \\
\normalsize
\end{center}

\vspace{1cm}
\section*{Resolution gegen Zugangs- und Zulassungbeschränkung}

Aufgabe der Hochschulen sollte es sein, jedem Menschen die Möglichkeit zu Bildung zu eröffnen und sie nicht vorzuselektieren. Die dafür notwendigen Kapazitäten sind gegebenenfalls aufzubauen. 

In der Physik herrscht weder in der Vergangenheit noch derzeit ein Mangel an Studienplätzen.\footnote{Die Studiengangsdatenbank der Webseite www.hochschulkompass.de der Hochschulrektorenkonferenz ermöglicht für (fast) alle Studiengänge in Deutschland die Abfrage, ob sie zulassungsbeschränkt sind und welche Zugangsvoraussetzungen zu erfüllen sind. Die Grenznoten der Zulassungsverfahren der letzten Semester lassen sich auf den Hochschulwebseiten recherchieren, die privat betriebene Webseite \url{https://www.nc-werte.info/studiengang/physik/} pflegt eine Übersichtstabelle, in der diese Informationen zusammen getragen sind. Eine stichprobenartige Kontrolle ergab, dass diese Übersicht verlässlich ist.} Dennoch gibt es an einigen Hochschulen (größtenteils formale) Zulassungsbeschränkungen und bei den meisten Physik-Masterstudiengängen zusätzlich Zugangsbeschränkungen, vor allem in Form von Grenznoten. Dies führt dazu, dass das Studium den Charakter eines Privilegs bzw. einer Belohnung, statt der eines Rechts bekommt und hat vor allem folgende Wirkungen: 
\begin{itemize}
\item Die Existenz von Zugangs- und Zulassungsbeschränkungen erzieht in die falsche Richtung: 
\begin{itemize}
\item Lernen nach Notenkalkül und Konkurrenz zwischen Schüler*innen bzw. Kommiliton*innen werden durch Zulassungs- und Zugangsbeschränkungen heraufbeschworen, obwohl sie nicht in das Selbstverständnis einer kooperativen Wissenschaft passen.\footnote{Diese Konkurrenz erschwert Kooperation oder legt nahe, nur strategisch zu kooperieren. Dies gilt auch dann, wenn die Zulassungsbeschränkungen faktisch nicht greifen, weil nur wenige abschätzen können, ob die Beschränkungen auch zukünftig nicht greifen.}
\item Zugangs- und Zulassungsbeschränkungen legen nahe, sich als auserwählt auf Grund besonders guter Erfüllung normierter Erwartungen anzusehen. Dies fördert Arroganz sowie ein angepasstes statt kritisch-hinterfragendes Studium. Dies ist kontraproduktiv dafür, dass Wissenschaft von allen im Interesse aller betrieben wird. Es ist zudem einschüchternd für Studierende ohne akademischen Hintergrund oder mit weniger geradlinigem Lebenslauf. 
\end{itemize}
\item Es gibt Universitäten, an denen es üblich ist, dass relativ viele Studierende ihre Zugangsberechtigung zum Masterstudium über Ausnahmeregelungen erhalten, obwohl sie die Grenznote nicht erfüllen.\footnote{\url{https://zapf.wiki/SoSe16_AK_NC#Berichte}} Die damit verbundene willkürliche Entscheidung über die Möglichkeit, sein Recht auf Bildung wahrzunehmen, ist unvertretbar und verstärkt die negative Erziehungswirkung immens. 
\end{itemize}
\textbf{Die ZaPF spricht sich gegen Zugangsbeschränkungen über das Abitur und dessen Äquivalente hinaus sowie gegen Zulassungsbeschränkungen jeder Form für die Physik-Bachelor-Studiengänge aus.}

\textbf{Die ZaPF spricht sich gegen Zugangsbeschränkungen über einen inhaltlich passenden (Bachelor-)Abschluss hinaus sowie gegen Zulassungsbeschränkungen jeder Form für die Physik-Master-Studiengänge aus. Insbesondere spricht sich die ZaPF gegen Grenznoten aus.}

Die verbreiteten Zugangsbeschränkungen im Master sind oft eine falsche Antwort auf das real existierende Problem, dass sich vor allem Bachelor-Absolvent*innen aus dem Ausland vielfach mit Voraussetzungen und Erwartungen bewerben, die nicht zu den Studiengängen passen. Grenznoten können dieses Problem aber nicht lösen, allein schon weil das Problem Bewerber*innen mit guten und schlechten Noten gleichermaßen betrifft. Auch (gut gemeinte) Willkür ist hier fehl am Platz. Die ZaPF empfiehlt stattdessen: 

\begin{itemize}
\item Klare Darstellung des gesamten Inhalts inklusive inhaltlicher Schwerpunkte, des Lehrkonzeptes und der Sprachvoraussetzungen 
\item Transparenz der zu erfüllenden Anforderungen an die vorangegangenen Studienleistungen 
\item Möglichkeit nicht erfüllte Anforderungen während des Masterstudiums nachzuholen 
\item Informatives, individuelles Beratungsgespräch vor der Einschreibung 
\end{itemize}


\vfill
\begin{flushright}
Verabschiedet am 13.11.2016 in Dresden
\end{flushright}




\end{document}
