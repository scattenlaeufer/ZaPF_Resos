\documentclass[draft,10pt,oneside]{scrartcl}

% Sprache und Encodings
\usepackage[ngerman]{babel}
\usepackage[T1]{fontenc}
\usepackage[utf8]{inputenc}

% Typographisch interessante Pakete
\usepackage{microtype} % Randkorrektur und andere Anpassungen

% References to Internet and within the document
\usepackage[pdftex,colorlinks=false,
pdftitle={Antrag zur Änderung der Geschäftsordnung für Plenen der ZaPF},
pdfauthor={Jörg Behrmann (FUB)},
pdfcreator={pdflatex},
pdfdisplaydoctitle=true]{hyperref}

% Absaetze nicht Einruecken
\setlength{\parindent}{0pt}
\setlength{\parskip}{2pt}

% Formatierung auf A4 anpassen
\usepackage{geometry}
\geometry{paper=a4paper,left=15mm,right=15mm,top=10mm,bottom=10mm}

\begin{document}

\section*{Antrag zur Verschiebung aus der Satzung der ZaPF in die
		Geschäftsordnung für Plenen der ZaPF}

\textbf{Antragsteller:} Jörg Behrmann (FUB), Björn Guth (RWTH)

\subsection*{Antrag}

Hiermit beantragen wir die Satzung der ZaPF wie folgendt zu ändern:

In §5 ersetze
\begin{quote}
	Die Mitgliedschaft im StAPF, dem Kommunikationsgremium oder dem TOPF endet
	mit Ablauf der Amtszeit, Ableben des Amtsinhabers oder der Amtsinhaberin,
	Niederlegung des Amtes oder Abwahl mit Zweidrittelmehrheit durch das
	Plenum.
\end{quote}
durch
\begin{quote}
	Die Mitgliedschaft im StAPF, dem Kommunikationsgremium oder dem TOPF endet
	mit Ablauf der Amtszeit, Ableben des Amtsinhabers oder der Amtsinhaberin,
	Niederlegung des Amtes oder Abwahl durch das Plenum.
\end{quote}
außerdem ersetze den darauf folgenden Satz durch
\begin{quote}
	Das Verfahren der Abwahl regelt die Geschäftsordnung für Plenen der ZaPF.
\end{quote}
\vspace{0.5cm}
Des weiteren beantragen wir die Geschäftsordnung für Plenen der ZaPF wie folgt
zu ändern:

In 4.2.9 füge
\begin{quote}
	Der Antrag auf Abwahl ist bis spätestens 15 Uhr am Vortag der ausrichtenden
	Fachschaft anzukündigen.
\end{quote}
als zweiten Satz ein.

Streiche den darauf folgenden Satz.

\subsection*{Begründung}
Dies Behebt einen Fehler, der uns bei der Einführung dieser Regelung
unterlaufen ist. Hierbei handelt es sich um ein Verfahren im Plenum der ZaPF
und hat daher in der Satzung nichts zu suchen, sondern gehört in die
Geschäftsordnung für Plenen der ZaPF.

\end{document}
