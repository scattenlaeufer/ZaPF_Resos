\documentclass[DIV=calc]{scrartcl}
\usepackage[utf8]{inputenc}
\usepackage[T1]{fontenc}
\usepackage[ngerman]{babel}
\usepackage{graphicx}

\usepackage{fixltx2e}
\usepackage{ellipsis}
\usepackage[tracking=true]{microtype}

\usepackage{lmodern}      % Ersatz fuer Computer Modern-Schriften
\usepackage{hfoldsty}

\usepackage{fourier}             % Schriftart
\usepackage[scaled=0.81]{helvet}     % Schriftart

\usepackage{url}
\usepackage{tocloft}             % Paket für Table of Contents

\usepackage{xcolor}
\definecolor{urlred}{HTML}{660000}

\usepackage{hyperref}
\hypersetup{
 colorlinks=true,    
 linkcolor=black,    % Farbe der internen Links (u.a. Table of Contents)
 urlcolor=black,    % Farbe der url-links
 citecolor=black} % Farbe der Literaturverzeichnis-Links

\usepackage{mdwlist}     % Änderung der Zeilenabstände bei itemize und enumerate

\parindent 0pt                 % Absatzeinrücken verhindern
\parskip 12pt                 % Absätze durch Lücke trennen

\usepackage{titlesec}    % Abstand nach Überschriften neu definieren
\titlespacing{\subsection}{0ex}{3ex}{-1ex}
\titlespacing{\subsubsection}{0ex}{3ex}{-1ex}        

% \pagestyle{empty}
\setlength{\textheight}{23cm}
\usepackage{fancyhdr}
\pagestyle{fancy}
\cfoot{}
\lfoot{Zusammenkunft aller Physik-Fachschaften}
\rfoot{www.zapfev.de\\stapf@zapf.in}
\renewcommand{\headrulewidth}{0pt}
\renewcommand{\footrulewidth}{0.1pt}


\begin{document}
\hspace{0.87\textwidth}
\begin{minipage}{120pt}
\vspace{-1.8cm}
\includegraphics[width=80pt]{logo.pdf}
\centering
\small Zusammenkunft aller Physik-Fachschaften
\end{minipage}
\begin{center}
%\huge{Resolution der Zusammenkunft aller Physik-Fachschaften} \\
\normalsize
\end{center}

\vspace{1cm}
\section*{Solidaritätsbekundung mit den Wissenschaftler*innen in der Türkei}

Seit dem Putschversuch in der Türkei werden Wissenschaftler*innen in der Türkei
systematisch drangsaliert und in ihrer Wissenschafts- und Reisefreiheit
eingeschränkt.

In dem von der Erdogan-Regierung kreierten Klima ist keine freie Meinungsäußerung
mehr möglich, da willkürliche Haft und sogar Folter befürchtet werden
müssen. Aus diesem Grund ist auch keine freie Forschung mehr möglich.

Die ZaPF erklärt sich mit den Wissenschaftler*innen in der Türkei solidarisch und
fordert die Bundesregierung auf, auf eine Verbesserung dieser Situation
hinzuwirken, so dass alle Menschen in der Türkei wieder frei von Repression
leben und arbeiten können.

Wir fordern weiterhin alle weiteren Empfänger dieser Resolution auf, sich ebenso
mit den Forscher*innen in der Türkei zu solidarisieren.
\vfill
\begin{flushright}
Verabschiedet am 13.11.2016 in Dresden
\end{flushright}




\end{document}


%%% Local Variables:
%%% mode: latex
%%% TeX-master: t
%%% End:
