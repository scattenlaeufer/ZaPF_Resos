\documentclass[DIV=calc]{scrartcl}
\usepackage[utf8]{inputenc}
\usepackage[T1]{fontenc}
\usepackage[ngerman]{babel}
\usepackage{graphicx}

\usepackage{ulem}
\usepackage[dvipsnames]{xcolor}
\usepackage{paralist}
\usepackage{fixltx2e}
\usepackage{ellipsis}
\usepackage[tracking=true]{microtype}

\usepackage{lmodern}                        % Ersatz fuer Computer Modern-Schriften
\usepackage{hfoldsty}

\usepackage{fourier}             % Schriftart
\usepackage[scaled=0.81]{helvet}     % Schriftart

\usepackage{url}
\usepackage{tocloft}             % Paket f�r Table of Contents

\usepackage{xcolor}
\definecolor{urlred}{HTML}{660000}

\usepackage{hyperref}
\hypersetup{
    colorlinks=true,    
    linkcolor=black,    % Farbe der internen Links (u.a. Table of Contents)
    urlcolor=black,    % Farbe der url-links
    citecolor=black} % Farbe der Literaturverzeichnis-Links

\usepackage{mdwlist}     % �nderung der Zeilenabst�nde bei itemize und enumerate
%\usepackage{draftwatermark} % Wasserzeichen ``Entwurf'' 
%\SetWatermarkText{Entwurf}

\usepackage{blindtext}
\parindent 0pt                 % Absatzeinr�cken verhindern
\parskip 12pt                 % Abs�tze durch L�cke trennen

%\usepackage{titlesec}    % Abstand nach �berschriften neu definieren
%\titlespacing{\subsection}{0ex}{3ex}{-1ex}
%\titlespacing{\subsubsection}{0ex}{3ex}{-1ex}        

% \pagestyle{empty}
\setlength{\textheight}{23cm}
\usepackage{fancyhdr}
\pagestyle{fancy}
\cfoot{}
\lfoot{Zusammenkunft aller Physik-Fachschaften}
\rfoot{www.zapfev.de\\stapf@zapf.in}
\renewcommand{\headrulewidth}{0pt}
\renewcommand{\footrulewidth}{0.1pt}
\newcommand{\gen}{*innen}

\begin{document}
    \hspace{0.87\textwidth}
    \begin{minipage}{120pt}
        \vspace{-1.8cm}
        \includegraphics[width=80pt]{../../logo.pdf}
        \centering
        \small Zusammenkunft aller Physik-Fachschaften
    \end{minipage}
    \begin{center}
        \huge{Resolution der Zusammenkunft aller Physikfachschaften}\vspace{.25\baselineskip}\\
        \normalsize
    \end{center}
    \vspace{1cm} 
    
%\textbf{Antragsteller}: Jan Geisel-Brinck (Uni Köln), Stefan Brackertz (Uni Köln)\\\\
%\textbf{Adressaten}: Physik-Fachschaften des deutschsprachigen Raumes, DPG, KFP, Wissenschaftsministerien der Bundesländer und Österreich und der Schweiz\\\\
\section*{Zu Zwangsexmatrikulation} 
       
Die ZaPF spricht sich gegen sämtliche Regelungen in Studienordnungen aus, welche den Fokus des Studiums von der Aneignung von Wissen und persönlicher Entwicklung hin zu der Verhinderung der eigenen Exmatrikulation verschieben. Insbesondere fordern wir, solche Regelungen aufzuheben oder abzuändern,  die auf eine Zwangsexmatrikulation hinaus laufen können (z.B. die Begrenzung der Anzahl von Prüfungsversuchen).

Studierende durch drohende Zwangsexmatrikulationen unter Druck zu setzen ist in unseren Augen unangemessen; es ersetzt selbstverantwortliches und selbstbestimmtes durch prüfungsorientiertes Studieren  und behindert damit die freie Entfaltung.

Zudem stellt es eine Erleichterung für alle Beteiligten dar, wenn Dozierende nicht vor der Entscheidung stehen, Studierende z.B. in ihrem letzten Prüfungsversuch ggf. entweder trotz fraglicher Leistungen bestehen zu lassen oder ihnen für den Rest des Lebens Chancen zu nehmen.

Ein erzwungenes Studienende ist nicht als Akt der Fürsorge zu verstehen. Stattdessen gilt es, wenn
Studierende wiederholt durch Prüfungen fallen, die zu Grunde liegenden Probleme beispielsweise im Rahmen von Beratungen zu analysieren und kooperativ zu lösen. Auch ermöglicht dies, Probleme, die nicht in der Schuld der Studierenden liegen, zu erkennen, und ist eine Voraussetzung, um systematische, über den Einzelfall hinaus gehende Lösungen zu entwickeln.
\vspace{-0.5\baselineskip}
    \begin{flushright}
        Verabschiedet am 1.11.2017 in Siegen
    \end{flushright}
\end{document}
%%% Local Variables:
%%% mode: latex
%%% TeX-master: t
%%% End: