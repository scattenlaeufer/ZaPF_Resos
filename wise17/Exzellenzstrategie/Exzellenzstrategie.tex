\documentclass[DIV=calc]{scrartcl}
\usepackage[utf8]{inputenc}
\usepackage[T1]{fontenc}
\usepackage[ngerman]{babel}
\usepackage{graphicx}

\usepackage{ulem}
\usepackage[dvipsnames]{xcolor}

\usepackage{fixltx2e}
\usepackage{ellipsis}
\usepackage[tracking=true]{microtype}

\usepackage{lmodern}                        % Ersatz fuer Computer Modern-Schriften
\usepackage{hfoldsty}

\usepackage{fourier}             % Schriftart
\usepackage[scaled=0.81]{helvet}     % Schriftart

\usepackage{url}
\usepackage{tocloft}             % Paket f�r Table of Contents

\usepackage{xcolor}
\definecolor{urlred}{HTML}{660000}

\usepackage{hyperref}
\hypersetup{
    colorlinks=true,    
    linkcolor=black,    % Farbe der internen Links (u.a. Table of Contents)
    urlcolor=black,    % Farbe der url-links
    citecolor=black} % Farbe der Literaturverzeichnis-Links

\usepackage{mdwlist}     % �nderung der Zeilenabst�nde bei itemize und enumerate
%\usepackage{draftwatermark} % Wasserzeichen ``Entwurf'' 
%\SetWatermarkText{Entwurf}

\usepackage{blindtext}
\parindent 0pt                 % Absatzeinr�cken verhindern
\parskip 12pt                 % Abs�tze durch L�cke trennen

%\usepackage{titlesec}    % Abstand nach �berschriften neu definieren
%\titlespacing{\subsection}{0ex}{3ex}{-1ex}
%\titlespacing{\subsubsection}{0ex}{3ex}{-1ex}        

% \pagestyle{empty}
\setlength{\textheight}{23cm}
\usepackage{fancyhdr}
\pagestyle{fancy}
\cfoot{}
\lfoot{Zusammenkunft aller Physik-Fachschaften}
\rfoot{www.zapfev.de\\stapf@zapf.in}
\renewcommand{\headrulewidth}{0pt}
\renewcommand{\footrulewidth}{0.1pt}
\newcommand{\gen}{*innen}

\begin{document}
    \hspace{0.87\textwidth}
    \begin{minipage}{120pt}
        \vspace{-1.8cm}
        \includegraphics[width=80pt]{../../logo.pdf}
        \centering
        \small Zusammenkunft aller Physik-Fachschaften
    \end{minipage}
    \begin{center}
        \huge{Resolution der Zusammenkunft der Physik Fachschaften}\vspace{.25\baselineskip}\\
        \normalsize
    \end{center}
    \vspace{1cm}    %\textbf{Antragstellerin}: Victoria Schemenz (KIT)\\
%\textbf{Adressaten}:
%DFG, Forschungs- und Wissenschaftsminister der Länder und des Bundes, KMK, GWK, HRK, Landesrektorenkonferenz, LandesAStenKonferenzen

\textbf{Zur Exzellenzstrategie}

Bezugnehmend auf die Ausschreibung der DFG zu den Exzellenzuniversitäten fordert die ZaPF, dass bei der Auswahl die Aspekte der exzellenten Lehre eine elementare Rolle spielen.
Die Universitäten sollen zudem in ihren Anträgen explizit angeben, wie sie ein Ungleichgewicht von Lehre und Forschung verhindern. 

\textbf{Begründung}

Die Ausbildungsfunktion der Universität sollte von zentraler Bedeutung sein und ist essentieller Bestandteil einer erfolgreichen Forschungsuniversität im internationalen Wettbewerb. 
Das Erfolgsmodell der Einheit von Lehre und Forschung nach dem humboldtschen Prinzip wird durch die einseitige Förderung der Forschung durch die Exzellenzstrategie aufgehoben \footnote{siehe hierzu auch Imboden-Bericht 2016 Kapitel 3.3.}. Ein solches Ungleichgewicht sollte in jedem Fall verhindert werden. Profitieren die Studierenden nicht von der neuesten (geförderten) Forschung, so bleiben die Probleme der studentischen Ausbildung im direkten Vergleich mit internationalen Spitzenuniversitäten bestehen - dem eigentlichen Ziel der Exzellenzstrategie.
\vspace{-0.5\baselineskip}
    \begin{flushright}
        Verabschiedet am 1.11.2017 in Siegen
    \end{flushright}
\end{document}
%%% Local Variables:
%%% mode: latex
%%% TeX-master: t
%%% End: