\documentclass[draft,10pt,oneside]{scrartcl}

% Sprache und Encodings
\usepackage[ngerman]{babel}
\usepackage[T1]{fontenc}
\usepackage[utf8]{inputenc}

% Typographisch interessante Pakete
\usepackage{microtype} % Randkorrektur und andere Anpassungen

% References to Internet and within the document
\usepackage[pdftex,colorlinks=false,
pdftitle={Antrag zur Änderung der Geschäftsordnung für Plenen der ZaPF},
pdfauthor={Jörg Behrmann (FUB), Björn Guth (RWTH)},
pdfcreator={pdflatex},
pdfdisplaydoctitle=true]{hyperref}

% Absaetze nicht Einruecken
\setlength{\parindent}{0pt}
\setlength{\parskip}{2pt}

% Formatierung auf A4 anpassen
\usepackage{geometry}
\geometry{paper=a4paper,left=15mm,right=15mm,top=10mm,bottom=10mm}

\begin{document}

\section*{Antrag zur Änderung der Geschäftsordnung für Plenen der ZaPF}

\textbf{Antragsteller:} Jörg Behrmann (FUB), Björn Guth (RWTH)

\subsection*{Antrag}

Hiermit beantragen wir die Geschäftsordnung für Plenen der ZaPF wie folgend zu
ändern:

In 4.1 füge als neuen Punkt 7 ein:
\begin{quote}
	\textit{Konkurriende Anträge} sind einander widersprechende Anträge zur
	selben Sache.
\end{quote}
Korrigere die nachfolgende Nummerierung dem entsprechen und füge als neuen
Anhang an:
\begin{quote}
	\subsubsection*{Konkurrierende Anhänge}
	Konkurriende Anträge entfallen üblicherweise in eine von zwei Kategorien:
	\begin{enumerate}
		\item Verschiedene Änderungsanträge, die die selbe Textstelle ändern
			wollen.
		\item Verschiede inhaltliche Beschlussfassungen zur selben Sache.
	\end{enumerate}
\end{quote}

\subsection*{Begründung}
Bisher definiert die Geschäftsordnung nicht, was genau als konkurrierende
Anträge behandelt werden muss. Dies führte in der Vergangenheit vor allem bei
konkurrierenden Änderungsanträgen zu unnötigen Mehrfachabstimmungen und
Zeitverzögerungen im Plenum.

\end{document}
