\documentclass[DIV=calc]{scrartcl}
\usepackage[utf8]{inputenc}
\usepackage[T1]{fontenc}
\usepackage[ngerman]{babel}
\usepackage{graphicx}

\usepackage{ulem}
\usepackage[dvipsnames]{xcolor}

\usepackage{fixltx2e}
\usepackage{ellipsis}
\usepackage[tracking=true]{microtype}

\usepackage{lmodern}                        % Ersatz fuer Computer Modern-Schriften
\usepackage{hfoldsty}

\usepackage{fourier}             % Schriftart
\usepackage[scaled=0.81]{helvet}     % Schriftart

\usepackage{url}
\usepackage{tocloft}             % Paket f�r Table of Contents

\usepackage{xcolor}
\definecolor{urlred}{HTML}{660000}

\usepackage{hyperref}
\hypersetup{
    colorlinks=true,    
    linkcolor=black,    % Farbe der internen Links (u.a. Table of Contents)
    urlcolor=black,    % Farbe der url-links
    citecolor=black} % Farbe der Literaturverzeichnis-Links

\usepackage{mdwlist}     % �nderung der Zeilenabst�nde bei itemize und enumerate
%\usepackage{draftwatermark} % Wasserzeichen ``Entwurf'' 
%\SetWatermarkText{Entwurf}

\usepackage{blindtext}
\parindent 0pt                 % Absatzeinr�cken verhindern
\parskip 12pt                 % Abs�tze durch L�cke trennen

%\usepackage{titlesec}    % Abstand nach �berschriften neu definieren
%\titlespacing{\subsection}{0ex}{3ex}{-1ex}
%\titlespacing{\subsubsection}{0ex}{3ex}{-1ex}        

% \pagestyle{empty}
\setlength{\textheight}{23cm}
\usepackage{fancyhdr}
\pagestyle{fancy}
\cfoot{}
\lfoot{Zusammenkunft aller Physik-Fachschaften}
\rfoot{www.zapfev.de\\stapf@zapf.in}
\renewcommand{\headrulewidth}{0pt}
\renewcommand{\footrulewidth}{0.1pt}
\newcommand{\gen}{*innen}

\begin{document}
    \hspace{0.87\textwidth}
    \begin{minipage}{120pt}
        \vspace{-1.8cm}
        \includegraphics[width=80pt]{../../logo.pdf}
        \centering
        \small Zusammenkunft aller Physik-Fachschaften
    \end{minipage}
    \begin{center}
        \huge{Resolution der Zusammenkunft aller Physikfachschaften}\vspace{.25\baselineskip}\\
        \normalsize
    \end{center}
    \vspace{0cm}   

%\textbf{Antragsteller}: Stefan Brackertz (Uni K�ln), Fabian Freyer (FU Berlin)\\
%\textbf{Adressaten}: Fraktionen des Landtages NRW, Landesregierung NRW, Physik-Fachschaften\\

\section*{Zur Hochschulpolitik in Nordrhein-Westfalen}

Angesichts des Koalitionsvertrages der neuen NRW-Landesregierung nimmt die ZaPF wie folgt Stellung:

1. Die ZaPF fordert die Landesregierung auf, nicht dem Beispiel von Baden-Württem\-berg zu folgen, die Stellungnahmen der Hochschulen zu berücksichtigen und keine Studiengebühren - egal in welcher Form - einzuführen.

2. Die ZaPF fordert die Landesregierung auf, die Zivilklausel\footnote{siehe Resolution zur Zivilklausel:\\ \url{https://zapfev.de/resolutionen/sose17/gesellschaftlich_verantwortung/PosPapier_gesellschaftliche_verwantwortung.pdf}} nicht aus dem Hochschulgesetz zu streichen.
An der Drittmittelorientierung festzuhalten und gleichzeitig die "bürokratische (...) Bevormundung\grqq\footnote{Rede der Ministerin Pfeiffer-Poensgen am 27.9.2017}, \grqq zu einer nachhaltigen, friedlichen und demokratischen Welt\grqq\footnote{NRW-Hochschulgesetz} beitragen zu sollen, aufzuheben, bedeutet nicht mehr "Freiheit\grqq\footnote{Rede der Ministerin Pfeiffer-Poensgen am 27.9.2017} für die Hochschulen, sondern einen erhöhten Druck, auch inhumanen Vorhaben zuzuarbeiten.
Dies wurde zuletzt auch an der Entscheidung der RWTH-Aachen\footnote{siehe Pressemitteilung der RWTH Aachen:\\ \url{http://www.rwth-aachen.de/cms/root/Die-RWTH/Aktuell/Pressemitteilungen/September-2017/~oktv/Statement-der-RWTH-Aachen-zur-Machbarkei/}} deutlich, ein Drittmittelprojekt kurz vor Beendigung abzubrechen, bei dem es um eine Machbarkeitsstudie für ein Werk für Militärfahrzeuge in der Türkei ging.

3. Die ZaPF fordert die Landesregierung auf, an der gesetzlichen Verankerungen eines "Kodex  gute Arbeit\grqq ~festzuhalten und diesen weiterzuentwickeln.
Prekäre Arbeitsbedingungen sind den Kolleginnen und Kollegen weder zumutbar, noch tragen sie dazu bei, dass die Hochschulen ihren Aufgaben besser nachkommen können.
\vspace{-0.5\baselineskip}
    \begin{flushright}
        Verabschiedet am 1.11.2017 in Siegen
    \end{flushright}
\end{document}
%%% Local Variables:
%%% mode: latex
%%% TeX-master: t
%%% End: