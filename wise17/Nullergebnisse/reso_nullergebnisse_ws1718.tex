\documentclass[DIV=calc]{scrartcl}
\usepackage[utf8]{inputenc}
\usepackage[T1]{fontenc}
\usepackage[ngerman]{babel}
\usepackage{graphicx}

\usepackage{fixltx2e}
\usepackage{ellipsis}
\usepackage[tracking=true]{microtype}

\usepackage{lmodern}      % Ersatz fuer Computer Modern-Schriften
\usepackage{hfoldsty}

\usepackage{fourier} 			% Schriftart
\usepackage[scaled=0.81]{helvet} 	% Schriftart

\usepackage{csquotes}

\usepackage{url}
\usepackage{tocloft} 			% Paket für Table of Contents

\usepackage{xcolor}
\definecolor{urlred}{HTML}{660000}

\usepackage{hyperref}
\hypersetup{
 colorlinks=true,	
 linkcolor=black,	% Farbe der internen Links (u.a. Table of Contents)
 urlcolor=black,	% Farbe der url-links
 citecolor=black} % Farbe der Literaturverzeichnis-Links

\usepackage{mdwlist} 	% Änderung der Zeilenabstände bei itemize und enumerate

\parindent 0pt 				% Absatzeinrücken verhindern
\parskip 12pt 				% Absätze durch Lücke trennen

%\usepackage{titlesec}	% Abstand nach Überschriften neu definieren
%\titlespacing{\subsection}{0ex}{3ex}{-1ex}
%\titlespacing{\subsubsection}{0ex}{3ex}{-1ex}		

% \pagestyle{empty}
\setlength{\textheight}{23cm}
\usepackage{fancyhdr}
\pagestyle{fancy}
\cfoot{}
\lfoot{Zusammenkunft aller Physik-Fachschaften}
\rfoot{www.zapfev.de\\stapf@zapf.in}
\renewcommand{\headrulewidth}{0pt}
\renewcommand{\footrulewidth}{0.1pt}


\begin{document}
\hspace{0.87\textwidth}
\begin{minipage}{120pt}
\vspace{-1.8cm}
\includegraphics[width=80pt]{../../logo.pdf}
\centering
\small Zusammenkunft aller Physik-Fachschaften
\end{minipage}
\begin{center}
\huge{Resolution der Zusammenkunft aller Physik-Fachschaften} \\
\normalsize
\end{center}

\vspace{1.2cm}
\section*{Zum Umgang mit Nullergebnissen}
%\textbf{Adressaten:} DFG, HRK, KFP, BMBF, DPG
%
%
%\textbf{Antragsteller:} Jan Luca Naumann (FUB), Andreas Drotloff (Würzburg)
Die ZaPF sieht Nullergebnisse\footnotemark~als natürliche Begleiter ordentlicher Forschung. Allerdings sind sie keine Abfallprodukte, sondern haben an sich einen wissenschaftlichen Wert, den es zu schützen und zu wahren gilt. Sie stellen zwar keine endgültige Behandlung eines Themas dar, können für zukünftige Arbeiten aber eine wertvolle Hilfestellung bieten. Es soll darauf hingewirkt werden, dass sie als Folge von gründlicher Arbeit gesehen werden.

Dazu gehört insbesondere, dass Nullergebnisse der wissenschaftlichen Gemeinschaft in angemessener Weise zur Verfügung gestellt werden. Dadurch können Wissenschaftler*innen von den Erfahrungen anderer profitieren, und beispielsweise vermeiden, den selben nicht zielführenden Weg einzuschlagen. Das spart Ressourcen und liegt damit im Interesse aller am Prozess Beteiligten.

Bei der Planung und Vorbereitung von wissenschaftlichen Projekten soll der Umgang mit Nullergebnissen thematisiert werden. Die Erstellung von Konzepten entsprechend der Art des Projekts sollten in die Statuten der fördernden Gesellschaften aufgenommen werden. Dadurch lässt sich eine Veröffentlichung von Nullergebnissen langfristig als Teil der wissenschaftlichen Praxis etablieren. 

\footnotetext{\noindent
Als Nullergebnis definiert die ZaPF ein Resultat von wissenschaftlicher Arbeit, das eines der folgenden Kriterien erfüllt:
\begin{itemize}
\item Falsifizierung der ursprünglichen Arbeitshypothese
\item Mehrdeutiges oder nicht beweiskräftiges Ergebnis
\item Nicht zielführendes Ergebnis auf dem Weg zur einer Veröffentlichung (\enquote{Trial and Error})
\end{itemize}
Entscheidend ist, dass bei der Erlangung der Ergebnisse ordentliche wissenschaftliche Standards beachtet wurden. 
} 

\pagebreak
Um die oben genannten Ziele zu erreichen, spricht sich die ZaPF für die Umsetzung der folgenden Maßnahmen aus:
\begin{itemize}
\item Aufnahme von Informationen über Nullergebnisse des Projekts im Anhang von zugehörigen Veröffentlichungen. Dies ermöglicht es, sich bei der Recherche zu einem Thema neben dem Stand der Forschung direkt auch über Probleme bei der Umsetzung und Beobachtungen im gröberen Kontext zu informieren.
\item Aufbau einer institutsübergreifenden Infrastruktur zur Speicherung und Austausch von Daten, die unabhängig vom Stand der Aufbereitung nach Abschluss eines Projektes einen Wert für die wissenschaftliche Gemeinschaft haben können.
\end{itemize}
\vfill
\begin{flushright}
Verabschiedet am 31.10.2017 in Siegen
\end{flushright}



\end{document}