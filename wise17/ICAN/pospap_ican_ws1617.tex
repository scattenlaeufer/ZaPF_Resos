\documentclass[DIV=calc]{scrartcl}
\usepackage[utf8]{inputenc}
\usepackage[T1]{fontenc}
\usepackage[ngerman]{babel}
\usepackage{graphicx}

\usepackage{ulem}
\usepackage[dvipsnames]{xcolor}

\usepackage{fixltx2e}
\usepackage{ellipsis}
\usepackage[tracking=true]{microtype}

\usepackage{lmodern}                        % Ersatz fuer Computer Modern-Schriften
\usepackage{hfoldsty}

\usepackage{fourier}             % Schriftart
\usepackage[scaled=0.81]{helvet}     % Schriftart

\usepackage{url}
\usepackage{tocloft}             % Paket für Table of Contents

\usepackage{xcolor}
\definecolor{urlred}{HTML}{660000}

\usepackage{hyperref}
\hypersetup{
    colorlinks=true,    
    linkcolor=black,    % Farbe der internen Links (u.a. Table of Contents)
    urlcolor=black,    % Farbe der url-links
    citecolor=black} % Farbe der Literaturverzeichnis-Links

\usepackage{mdwlist}     % Änderung der Zeilenabstände bei itemize und enumerate
%\usepackage{draftwatermark} % Wasserzeichen ``Entwurf'' 
%\SetWatermarkText{Entwurf}
%
\usepackage{blindtext}
\parindent 0pt                 % Absatzeinrücken verhindern
\parskip 12pt                 % Absätze durch Lücke trennen

%\usepackage{titlesec}    % Abstand nach Überschriften neu definieren
%\titlespacing{\subsection}{0ex}{3ex}{-1ex}
%\titlespacing{\subsubsection}{0ex}{3ex}{-1ex}        

% \pagestyle{empty}
\setlength{\textheight}{23cm}
\usepackage{fancyhdr}
\pagestyle{fancy}
\cfoot{}
\lfoot{Zusammenkunft aller Physik-Fachschaften}
\rfoot{www.zapfev.de\\stapf@zapf.in}
\renewcommand{\headrulewidth}{0pt}
\renewcommand{\footrulewidth}{0.1pt}
\newcommand{\gen}{*innen}

\begin{document}
    \hspace{0.87\textwidth}
    \begin{minipage}{120pt}
        \vspace{-1.8cm}
        \includegraphics[width=80pt]{../../logo.pdf}
        \centering
        \small Zusammenkunft aller Physik-Fachschaften
    \end{minipage}
    \begin{center}
        \huge{Positionspapier der Zusammenkunft aller Physikfachschaften}\vspace{.25\baselineskip}\\
        \normalsize
    \end{center}
    \vspace{1cm} 
    %\textbf{Antragsstellende}: Lisa Zinta (Uni Rostock), Sven Stroteich (Uni Greifswald)\\\\
    %\textbf{Adressaten}:  ICAN, Fraktionen des Bundestages, Physik-Fachschaften
%       
\section*{Friedensnobelpreis für Atomwaffenverbotsinitiative}
Die ZaPF gratuliert ICAN zum Friedensnobelpreis.

Der Atomwaffenverbotsvertrag, dem die UNO-Vollversammlung am 7.7.2017 zugestimmt hat und der maßgeblich von ICAN voran gebracht wurde, geht auf das jahrzehntelange Engagement von Wissenschaftler*innen zurück:

Der erste Entwurf des Vertrages war Teil eines Konzeptes für eine Nuklearwaffenkonvention, die das International Network of Engineers and Scientists Against Proliferation (INESAP) 1995 vorgeschlagen hatte.\footnote{\texttt{http://www.inesap.org/what-inesap/achievements-and-activities}} INESAP ist eine Gruppe von mehr als 50 Wissenschaftler*innen aus 17 Ländern, die 1993 an der TU Darmstadt initiiert wurde.

In seiner Arbeit knüpfte INESAP an ein Konzept für eine atomwaffenfreie Welt\footnote{\texttt{http://www.inesap.org/projects}} an, das der Physiker Joseph Rotblat maßgeblich mit erarbeitet hatte. Rotblat war 1944 aus dem Manhattan-Projekt ausgestiegen und jüngster Unterzeichner des Russell-Einstein-Manifests geworden; 1957 hatte er die Pugwash-Bewegung mitbegründet.

Das Zustandekommen des Atomwaffenverbotsvertrages zeigt auf, wie bedeutsam und notwendig es ist, wenn Wissenschaftler*innen mit ihrer Arbeit Verantwortung für eine friedliche Entwicklung der Welt übernehmen.

Wir als Physikstudierende halten es für wichtig, sich mit diesem Thema im Zuge der politischen Bildung zu befassen, um das staatsbürgerliche Verantwortungsbewusstsein der Studierenden auf der Grundlage der verfassungsmäßigen Grundordnung zu fördern. Insbesondere rufen wir dazu auf, dass sich Studierende auch mit den gesellschaftlichen Implikationen wissenschaftlicher Forschung, wie sie gerade hier gegeben sind, beschäftigen.

\vspace{-0.5\baselineskip}
    \begin{flushright}
        Verabschiedet am 1.11.2017 in Siegen
    \end{flushright}
\end{document}
%%% Local Variables:
%%% mode: latex
%%% TeX-master: t
%%% End:
