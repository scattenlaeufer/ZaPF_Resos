\documentclass[DIV=calc]{scrartcl}
\usepackage[utf8]{inputenc}
\usepackage[T1]{fontenc}
\usepackage[ngerman]{babel}
\usepackage{graphicx}

\usepackage{ulem}
\usepackage[dvipsnames]{xcolor}
\usepackage{fixltx2e}
\usepackage{ellipsis}
\usepackage[tracking=true]{microtype}

\usepackage{lmodern}                        % Ersatz fuer Computer Modern-Schriften
\usepackage{hfoldsty}

\usepackage{fourier}             % Schriftart
\usepackage[scaled=0.81]{helvet}     % Schriftart

\usepackage{url}
\usepackage{tocloft}             % Paket f�r Table of Contents

\usepackage{xcolor}
\definecolor{urlred}{HTML}{660000}

\usepackage{hyperref}
\hypersetup{
    colorlinks=true,    
    linkcolor=black,    % Farbe der internen Links (u.a. Table of Contents)
    urlcolor=black,    % Farbe der url-links
    citecolor=black} % Farbe der Literaturverzeichnis-Links

\usepackage{mdwlist}     % �nderung der Zeilenabst�nde bei itemize und enumerate
%\usepackage{draftwatermark} % Wasserzeichen ``Entwurf'' 
%\SetWatermarkText{Entwurf}

\usepackage{blindtext}
\parindent 0pt                 % Absatzeinr�cken verhindern
\parskip 12pt                 % Abs�tze durch L�cke trennen

%\usepackage{titlesec}    % Abstand nach �berschriften neu definieren
%\titlespacing{\subsection}{0ex}{3ex}{-1ex}
%\titlespacing{\subsubsection}{0ex}{3ex}{-1ex}        

% \pagestyle{empty}
\setlength{\textheight}{23cm}
\usepackage{fancyhdr}
\pagestyle{fancy}
\cfoot{}
\lfoot{Zusammenkunft aller Physik-Fachschaften}
\rfoot{www.zapfev.de\\stapf@zapf.in}
\renewcommand{\headrulewidth}{0pt}
\renewcommand{\footrulewidth}{0.1pt}
\newcommand{\gen}{*innen}

\begin{document}
    \hspace{0.87\textwidth}
    \begin{minipage}{120pt}
        \vspace{-1.8cm}
        \includegraphics[width=80pt]{../../logo.pdf}
        \centering
        \small Zusammenkunft aller Physik-Fachschaften
    \end{minipage}
    \begin{center}
        \huge{Resolution  der Zusammenkunft aller Physik-Fachschaften}\vspace{.25\baselineskip}\\
        \normalsize
    \end{center}
    \vspace{1cm}      
%\textbf{Antragsteller}: Christian (Oldenburg), Daniela (FFM)
%
%\textbf{Addressaten}: KASAP, KMK, HRK, Akkreditierungsrat
%
\subsection*{Zu Änderungen im Akkreditierungswesen}
Die Zusammenkunft aller Physik-Fachschaften begrüßt generell eine Überarbeitung des Akkreditierungswesens. Eine solche Überarbeitung darf nicht unter Ausschluss studentischer Beteiligung stattfinden. Insbesondere erachten wir folgende Punkte als essentiell:

\begin{itemize}
\item Studentische Beteiligung an den Verfahren ist von fundamentaler Bedeutung. Eine Aufhebung der öffentlichen Begehung, selbst in Ausnahmefällen, ist als wichtiger Kontaktpunkt zwischen lokalen Fachschaften und dem Akkreditierungsverfahren abzulehnen.
\item Die Bedeutung der hochschulinternen Gremien mit unabhängiger studentischer Repräsentation für die Entwicklung von Studiengängen muss erhalten bleiben.
\item Akkreditierungszeiträume müssen kurz genug sein, so dass Missstände zeitnah erkannt und behoben werden können.
\item Die Befähigung zum zivilgesellschaftlichen Engagement muss als Studienziel erhalten bleiben.\footnote{\texttt{https://zapfev.de/resolutionen/wise12/Reso\_WiSe12\_Zivilgesellschaftliches \\ Engagement.pdf}}
\item Die Berücksichtigung der Vielfalt von Studierenden (wie etwa Belange Studierender mit Behinderung oder Studierender mit Kind) als Kriterium für die Akkreditierung darf nicht entfallen.
\end{itemize}
Aufgrund der aktuell stattfindenden gravierenden Änderungen im Akkreditierungssystem (siehe Positionspapier) ist es wichtig, zentrale Mindeststandards für das weitere Vorgehen festzusetzen. 
\vspace{-0.5\baselineskip}
    \begin{flushright}
        Verabschiedet am 31.10.2017 in Siegen
    \end{flushright}
\end{document}
%%% Local Variables:
%%% mode: latex
%%% TeX-master: t
%%% End:

\begin{comment}
Verlinkung mit anhängen
\end{comment}