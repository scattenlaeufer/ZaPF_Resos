\documentclass[DIV=calc]{scrartcl}
\usepackage[utf8]{inputenc}
\usepackage[T1]{fontenc}
\usepackage[ngerman]{babel}
\usepackage{graphicx}

\usepackage{ulem}
\usepackage[dvipsnames]{xcolor}
\usepackage{fixltx2e}
\usepackage{ellipsis}
\usepackage[tracking=true]{microtype}

\usepackage{ulem}
\usepackage{selinput}
\usepackage{lmodern}                        % Ersatz fuer Computer Modern-Schriften
\usepackage{hfoldsty}

\usepackage{fourier}             % Schriftart
\usepackage[scaled=0.81]{helvet}     % Schriftart

\usepackage{url}
\usepackage{tocloft}             % Paket für Table of Contents

\usepackage{xcolor}
\definecolor{urlred}{HTML}{660000}

\usepackage{hyperref}
\hypersetup{
    colorlinks=true,    
    linkcolor=black,    % Farbe der internen Links (u.a. Table of Contents)
    urlcolor=black,    % Farbe der url-links
    citecolor=black} % Farbe der Literaturverzeichnis-Links

\usepackage{mdwlist}     % Änderung der Zeilenabstände bei itemize und enumerate
\usepackage{draftwatermark} % Wasserzeichen ``Entwurf'' 
\SetWatermarkText{Entwurf}

\usepackage{blindtext}
\parindent 0pt                 % Absatzeinrücken verhindern
\parskip 12pt                 % Absätze durch Lücke trennen

%\usepackage{titlesec}    % Abstand nach Ãœberschriften neu definieren
%\titlespacing{\subsection}{0ex}{3ex}{-1ex}
%\titlespacing{\subsubsection}{0ex}{3ex}{-1ex}        

% \pagestyle{empty}
\setlength{\textheight}{23cm}
\usepackage{fancyhdr}
\pagestyle{fancy}
\cfoot{}
\lfoot{Zusammenkunft aller Physik-Fachschaften}
\rfoot{www.zapfev.de\\stapf@zapf.in}
\renewcommand{\headrulewidth}{0pt}
\renewcommand{\footrulewidth}{0.1pt}
\newcommand{\gen}{*innen}

\begin{document}
    \hspace{0.87\textwidth}
    \begin{minipage}{120pt}
        \vspace{-1.8cm}
        \includegraphics[width=80pt]{../../logo.pdf}
        \centering
        \small Zusammenkunft aller Physik-Fachschaften
    \end{minipage}
    \begin{center}
        \huge{Positionspapier der Zusammenkunft aller Physikfachschaften}\vspace{.25\baselineskip}\\
        \normalsize
    \end{center}
    \vspace{0cm}
%\textbf{Antragstellende}: Lena Lindenmeier (Potsdam), Lukas Koerber (Dresden), Marcus Mikorski (Alumni), Lars Vosteen (Giessen), Viet Hoang (LMU)\\\\
\textbf{Zur Rolle der Wissenschaftskommunikation}:

Die Zusammenkunft aller Physikfachschaften (ZaPF) positioniert sich für eine starke Wissen\-schafts\-kommunikation und weist auf die besondere gesellschaftliche Verantwortung von Wissenschaftler*innen hin. Bisher sehen wir die Wissen\-schafts\-kommunikation als unterschätzt an. Forschung muss kommuniziert werden. Neben der Bildung der Gesellschaft und der Verbreitung von Wissen soll Wissen\-schafts\-kommunikation ebenso der Rechtfertigung, aber auch der gesellschaftlichen Kontrolle der Wissenschaft dienen. Sie sollte Forschung transparenter machen, Neugierde wecken, zum Nachdenken anregen, Akzeptanz schaffen und insbesondere mögliche Ängste in der Gesellschaft vor wissenschaftlichen Entwicklungen aufarbeiten. Gleichzeitig müssen sich Wissenschaftler*innen aktiv in gesellschaftliche und politische Diskussionen einmischen und Unwissenschaftlichkeit entgegentreten.
\\\\Eine gute Wissenschaftskommunikation bereitet ihren Gegenstand unterhaltsam und zielgruppenorientiert auf. Ebenso wie die Kommunikation von Forschung nach innen zur Aufgabe von Wissenschaftler*innen gehört, sei es durch Abschlussarbeiten, durch das Publizieren von Papern, oder das Halten  wis\-sen\-schaft\-licher Vorträge, so sollten Sie auch nach außen wirken, z.B. je nach Zielgruppe durch Formate wie Podcasts, Blogs, Videos, Science Slams oder wissenschaftliche Artikel in Zeitschriften.
\\\\Wichtig ist hierbei das Erschließen neuer Zielgruppen und die Nach\-wuchs\-för\-der\-ung. Auf für Wissenschaftler*innen teils oft schwer zu erreichende Gruppen wie bildungsferne Schichten oder Menschen mit Migrationshintergrund soll aktiv zugegangen werden. Dialog und Integration können und sollten auch über Wissenschaft stattfinden.
\\\\Eine besondere Rolle in der Ausübung sowie der Stärkung der Wissen\-schafts-\\\-kommunikation sprechen wir den Universitäten und weiteren Hochschulen zu. Diese Rolle sollte unter anderem in der Durch\-führ\-ung von Veranstaltungen sowie der Sensibilisierung und Ausbildung von zukünftigen Wissenschaftler*innen in Wissenschaftskommunikation sein.
Wir begrüßen das Engagement für Veranstaltungen wie z.B. Lange Nächte der Wissenschaften oder Schüleruniversitäten und sehen großes Potenzial in der Einbringung von Wissenschaftskommunikation in die akamedische Ausbildung.
\vspace{-0.5\baselineskip}
    \begin{flushright}
        Verabschiedet am 31.10.2017 in Siegen
    \end{flushright}
\end{document}
%%% Local Variables:
%%% mode: latex
%%% TeX-master: t
%%% End:

% aber auch der Rückung der Wissenschaft in den öffentlichen Blickpunkt
