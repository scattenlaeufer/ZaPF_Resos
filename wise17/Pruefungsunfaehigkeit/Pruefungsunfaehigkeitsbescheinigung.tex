\documentclass[DIV=calc]{scrartcl}
\usepackage[utf8]{inputenc}
\usepackage[T1]{fontenc}
\usepackage[ngerman]{babel}
\usepackage{graphicx}

\usepackage{xcolor}
\usepackage{ulem}
\usepackage{fixltx2e}
\usepackage{ellipsis}
\usepackage[tracking=true]{microtype}

\usepackage{lmodern}                        % Ersatz fuer Computer Modern-Schriften
\usepackage{hfoldsty}

\usepackage{fourier}             % Schriftart
\usepackage[scaled=0.81]{helvet}     % Schriftart

\usepackage{url}
\usepackage{tocloft}             % Paket f�r Table of Contents

\usepackage{xcolor}
\definecolor{urlred}{HTML}{660000}

\usepackage{hyperref}
\hypersetup{
    colorlinks=true,    
    linkcolor=black,    % Farbe der internen Links (u.a. Table of Contents)
    urlcolor=black,    % Farbe der url-links
    citecolor=black} % Farbe der Literaturverzeichnis-Links

\usepackage{mdwlist}     % �nderung der Zeilenabst�nde bei itemize und enumerate
%\usepackage{draftwatermark} % Wasserzeichen ``Entwurf'' 
%\SetWatermarkText{Entwurf}

\usepackage{blindtext}
\parindent 0pt                 % Absatzeinr�cken verhindern
\parskip 12pt                 % Abs�tze durch L�cke trennen

%\usepackage{titlesec}    % Abstand nach �berschriften neu definieren
%\titlespacing{\subsection}{0ex}{3ex}{-1ex}
%\titlespacing{\subsubsection}{0ex}{3ex}{-1ex}        

% \pagestyle{empty}
\setlength{\textheight}{23cm}
\usepackage{fancyhdr}
\pagestyle{fancy}
\cfoot{}
\lfoot{Zusammenkunft aller Physik-Fachschaften}
\rfoot{www.zapfev.de\\stapf@zapf.in}
\renewcommand{\headrulewidth}{0pt}
\renewcommand{\footrulewidth}{0.1pt}
\newcommand{\gen}{*innen}

\begin{document}
    \hspace{0.87\textwidth}
    \begin{minipage}{120pt}
        \vspace{-1.8cm}
        \includegraphics[width=80pt]{../../logo.pdf}
        \centering
        \small Zusammenkunft aller Physik-Fachschaften
    \end{minipage}
    \begin{center}
        \huge{Resolution der Zusammenkunft aller Physikfachschaften}\vspace{.25\baselineskip}\\ %\large{Hier kommt eine Unterüberschrift hin} \\
        \normalsize
    \end{center}
    \vspace{0cm}   
%\textbf{Antragsteller}: Daniela (FFM), Maik (Bielefeld)

%\textbf{Adressaten}: alle Landesregierungen in deren Hochschulgesetz die Symptomplicht steht (d.i. alle außer NRW), KMK.
\section*{Zu Prüfungsunfähigkeitsbescheinigungen}


Die Zusammenkunft aller Physikfachschaften (ZaPF) spricht sich für die Schaffung von gesetzlichen Grundlagen für Prüfungsunfähigkeitsbscheinigungen analog zu Arbeitsunfähigkeitsbescheinigungen aus.

Wir verweisen an dieser Stelle auf die Resolution aus der Winter-ZaPF "Resolution zu Symptompflicht auf Attesten\grqq ~in Dresden 2016, in der wir Symptompflicht ablehnen.
\vspace{1cm}

%\textbf{Begründung}:

%Die Regelung, einmal umgesetzt, schafft eine sichere und faire rechtliche Grundlage für Studierende im Ausfallfall.

%Dies steht nicht im Widerspruch zur Resolution, aus der Winter-ZaPF "Resolution zu Symptompflicht auf Attesten" in Dresden 2016.

%Übersendung an LR NRW soll nicht stattfinden, da dort keine Symptompflicht ist aber eine neue Landesregierung, die wahrscheinlich das Hochschulgesetz ändern will und wir sie nicht auf dumme Gedanken bringen sollten.
\vspace{-0.5\baselineskip}
    \begin{flushright}
        Verabschiedet am 31.10.2017 in Siegen
    \end{flushright}
   \newpage
  
\hspace{0.87\textwidth}
\begin{minipage}{120pt}
\vspace{-1.8cm}
\includegraphics[width=80pt]{../../logo.pdf}
\centering
\small Zusammenkunft aller Physik-Fachschaften
\end{minipage}
\begin{center}
\huge{Resolution der Zusammenkunft aller Physik-Fachschaften} \\
\normalsize
\end{center}

\vspace{1cm}
\section*{Resolution zu Symptompflicht auf Attesten}
Die Zusammenkunft aller Physikfachschaften (ZaPF) spricht sich gegen  die geforderte Angabe von Symptomen auf Attesten für die Prüfungsunfähigkeitsmeldung aus. 

An vielen Universitäten ist es erforderlich, für den Nachweis der Prüfungsunfähigkeit ein ärztliches Attest mit der Angabe von Symptomen einzureichen. Der Prüfungsausschuss entscheidet darüber, ob die Symptome im jeweiligen Fall eine Prüfungsunfähigkeit darstellen. 

Aus unserer Sicht sprechen mehrere Gründe gegen diese Regelung: 
\begin{itemize}
\item Studierende müssen Ärzt*innen “freiwillig” von der Schweigepflicht entbinden 
\item Die Mitglieder der Prüfungsausschüsse haben in der Regel keine Qualifikation, um über Leistungseinschränkungen durch die angegebenen Symptome zu entscheiden. 
\item Die Weitergabe und Speicherung solcher hochsensibler Daten birgt das Risiko, dass ungewollt Dritte Kenntnis darüber gelangen 
\end{itemize}

Wir fordern die Gesetzgeber daher dazu auf, ausschließlich folgendes Verfahren zu ermöglichen: 
Eine Arbeitsunfähigkeitsbescheinigung ist einer ärztlichen Prüfungsunfähigkeitsbescheinigung gleichzusetzen.
\vfill
\begin{flushright}
Verabschiedet am 13.11.2016 in Dresden
\end{flushright}

\end{document}
%%% Local Variables:
%%% mode: latex
%%% TeX-master: t
%%% End:

\begin{comment}
\end{comment}