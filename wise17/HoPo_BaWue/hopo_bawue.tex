\RequirePackage[hyphens]{url}
\documentclass[DIV=calc]{scrartcl}

\usepackage[utf8]{inputenc}
\usepackage[T1]{fontenc}
\usepackage[ngerman]{babel}
\usepackage{graphicx}

\usepackage{ulem}
\usepackage[dvipsnames]{xcolor}

%\usepackage{fixltx2e}
\usepackage{ellipsis}
\usepackage[tracking=true]{microtype}

\usepackage{lmodern}                        % Ersatz fuer Computer Modern-Schriften
\usepackage{hfoldsty}

\usepackage{fourier}             % Schriftart
\usepackage[scaled=0.81]{helvet}     % Schriftart

\usepackage{url}
\usepackage{tocloft}             % Paket f�r Table of Contents

\usepackage{xcolor}
\definecolor{urlred}{HTML}{660000}

\usepackage{hyperref}
\hypersetup{
    colorlinks=true,    
    linkcolor=black,    % Farbe der internen Links (u.a. Table of Contents)
    urlcolor=black,    % Farbe der url-links
    citecolor=black} % Farbe der Literaturverzeichnis-Links

\usepackage{mdwlist}     % �nderung der Zeilenabst�nde bei itemize und enumerate
%\usepackage{draftwatermark} % Wasserzeichen ``Entwurf'' 
%\SetWatermarkText{Entwurf}
\usepackage{blindtext}
\parindent 0pt                 % Absatzeinr�cken verhindern
\parskip 12pt                 % Abs�tze durch L�cke trennen

%\usepackage[]{url}
%\usepackage{titlesec}    % Abstand nach �berschriften neu definieren
%\titlespacing{\subsection}{0ex}{3ex}{-1ex}
%\titlespacing{\subsubsection}{0ex}{3ex}{-1ex}        

% \pagestyle{empty}
\setlength{\textheight}{23cm}
\usepackage{fancyhdr}
\pagestyle{fancy}
\cfoot{}
\lfoot{Zusammenkunft aller Physik-Fachschaften}
\rfoot{www.zapfev.de\\stapf@zapf.in}
\renewcommand{\headrulewidth}{0pt}
\renewcommand{\footrulewidth}{0.1pt}
\newcommand{\gen}{*innen}

\begin{document}
    \hspace{0.87\textwidth}
    \begin{minipage}{120pt}
        \vspace{-1.8cm}
        \includegraphics[width=80pt]{../../logo.pdf}
        \centering
        \small Zusammenkunft aller Physik-Fachschaften
    \end{minipage}
    \begin{center}
        \huge{Resolution der Zusammenkunft aller Physik Fachschaften}\vspace{.25\baselineskip}\\
        \normalsize
    \end{center}
    \vspace{1cm}    %\textbf{Adressaten}: MWK Baden-Württemberg, allen Fraktion (Fraktionsführung) im Baden-Württembergischen Landtag, Landesastenkonferenz Baden-Württemberg, Die Grünen (Bund) (Parteiführung), Landesverband (Vorstand) der Grünen in BaWü, Kai Gering, BAG Wissenschaft Hochschule Technologiepolitik von Bündnis 90 die Grünen
    
\subsection*{Zu den aktuellen hochschulpolitischen Entwicklungen in Baden-Württemberg}    
Die ZaPF fordert

1. die Beibehaltung eines politischen Mandates der Studierendenschaften in Baden-Württemberg, damit sie ihre studentischen Interessen weiterhin vor Gesellschaft und Politik vorstellen und für diese weiterhin auch öffentlich streiten dürfen.

2. die CDU-Fraktion muss ihre anmaßenden Anschuldigungen zurückziehen. Zwischen einer studentischen Interessenvertretung, auch durch Demonstrationen, und der Unterstützung und Duldung von Straftaten besteht ein klarer Unterschied. 

3. das Ministerium für Wissenschaft und Kunst, wie auch die gesamte Landesregierung, auf, die Studierendenschaften im Land frühzeitig in Gesetzesänderungen einzubeziehen und wie in anderen Bundesländern keine Maßnahmen "...gegen die Hochschulen oder über die Hochschulen hinweg" [Pfeiffer-Poensgen, 27.9.17] durchzuführen.


4. die Abschaffung der kürzlich eingeführten Studiengebühren.

\subsection*{Begründung}

Das Ministerium für Wissenschaft und Kunst hat Pläne, das Landeshochschulgesetz zu reformieren\footnote{Zeitungsbericht, hier exemplarisch Badische Zeitung:\\
	\url{http://www.badische-zeitung.de/suedwest-1/studierendenvertretung-soll-politisches-mandat-verlieren--139913324.html}
}. Hierbei soll auch der Satz, der den Studierendenschaften das politische Mandat garantiert, gestrichen werden, was trotz zuvoriger Informations- und Diskussionsveranstaltung von MWK, Hochschulen und Studierendenschaften erst durch die Presse bekannt wurde.

\newpage
Doch was bedeutet die Streichung des politischen Mandats?

Die Grünen-Fraktion sagt: Das ändert nichts\footnote{Stellungnahme der Grünen-Fraktion:\\
	\url{https://www.gruene-landtag-bw.de/themen/wissenschaft-kultur/recht-auf-oeffentliche-stellungnahme-und-mitbestimmung-der-studierenden-bleibt-ohne-einschraenkung-erhalten.html}
}. Die CDU konstruiert derweil in der Südwestpresse\footnote{ Bericht in der Südwestpresse:\\
\url{http://www.swp.de/ulm/nachrichten/suedwestumschau/land-will-studentenvertretern-politisches-mandat-entziehen-15500464.html}
} einen Zusammenhang zwischen Krawallen und Randalierer*innen auf Demonstrationen und dem politischen Mandat, das deshalb eingeschränkt werden müsse. 

In diesem Zusammenhang stellte die CDU-Fraktion im Landtag die Große Anfrage  zu dem Thema Linksextremismus in Baden-Württemberg '(16-2642), zu deren Beantwortung am 27. September ein Schreiben\footnote{Landtagsanfrage 16-2642:\\
	\url{https://www.dropbox.com/s/euxp6pbp0x3w9bz/2017\%2009\%2027\%20Abfrageformular\%20Grosse\%20Anfrage\%2016-2642.pdf?dl=0}} an alle Hochschulen in Baden-Württemberg ging. Dieses Schreiben sollte bis zum 29.09.17 im Benehmen mit den jeweiligen Studierendenschaften beantwortet werden.

Zum Thema Studiengebühren verweisen wir auf das Positionspapier \footnote{Positionspapier zu Studiengebühren:
	\\\url{https://zapf.wiki/Datei:Positionspapier_Studiengebuehren_WiSe16.pdf}} und  die Resolution \footnote{Resolution:\\
	\url{https://zapf.wiki/images/a/ac/Reso_Studiengebuehren_WiSe16.pdf}}, in der wir uns der Positionierung der Landesstudierendenvertretung Baden-Württemberg zur Einführung von Studiengebühren für internationale Studierende, Einführung von Zweitstudiengebühren und Erhöhung des "Verwaltungskostenbeitrages
\grqq \footnote{Positionierung der Landesstudierendenvertretung Baden-Württemberg zur Einführung von Studiengebühren für internationale Studierende, Einführung von Zweitstudiengebühren und Erhöhung des "Verwaltungskostenbeitrages\grqq:\\
	\url{http://www.studis.de/lak-bawue/fileadmin/lak-bawue/PMs_und_offene_Briefe/Positionierung_der_Landesstudierendenvertretung_Baden-Wuerttemberg_zur_Einfuehrung_von_Studiengebuehren.pdf}
} anschlossen. 

\sloppy

\textbf{Anhang}: 
\\Resolution über Studiengebühren
\vspace{-0.5\baselineskip}
    \begin{flushright}
        Verabschiedet am 1.11.2017 in Siegen
    \end{flushright}
\end{document}
%%% Local Variables:
%%% mode: latex
%%% TeX-master: t
%%% End: