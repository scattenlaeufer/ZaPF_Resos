\documentclass[DIV=calc]{scrartcl}
\usepackage[utf8]{inputenc}
\usepackage[T1]{fontenc}
\usepackage[ngerman]{babel}
\usepackage{graphicx}

\usepackage{ulem}
\usepackage[dvipsnames]{xcolor}
\usepackage{paralist}
\usepackage{fixltx2e}
\usepackage{ellipsis}
\usepackage[tracking=true]{microtype}

\usepackage{lmodern}                        % Ersatz fuer Computer Modern-Schriften
\usepackage{hfoldsty}

\usepackage{fourier}             % Schriftart
\usepackage[scaled=0.81]{helvet}     % Schriftart

\usepackage{url}
\usepackage{tocloft}             % Paket für Table of Contents

\usepackage{xcolor}
\definecolor{urlred}{HTML}{660000}

\usepackage{hyperref}
\hypersetup{
    colorlinks=true,    
    linkcolor=black,    % Farbe der internen Links (u.a. Table of Contents)
    urlcolor=black,    % Farbe der url-links
    citecolor=black} % Farbe der Literaturverzeichnis-Links

\usepackage{mdwlist}     % Änderung der Zeilenabstände bei itemize und enumerate
%\usepackage{draftwatermark} % Wasserzeichen ``Entwurf'' 
%\SetWatermarkText{Entwurf}

\usepackage{blindtext}
\parindent 0pt                 % Absatzeinrücken verhindern
\parskip 12pt                 % Absätze durch Lücke trennen

%\usepackage{titlesec}    % Abstand nach Überschriften neu definieren
%\titlespacing{\subsection}{0ex}{3ex}{-1ex}
%\titlespacing{\subsubsection}{0ex}{3ex}{-1ex}        

% \pagestyle{empty}
\setlength{\textheight}{23cm}
\usepackage{fancyhdr}
\pagestyle{fancy}
\cfoot{}
\lfoot{Zusammenkunft aller Physik-Fachschaften}
\rfoot{www.zapfev.de\\stapf@zapf.in}
\renewcommand{\headrulewidth}{0pt}
\renewcommand{\footrulewidth}{0.1pt}
\newcommand{\gen}{*innen}

\begin{document}
    \hspace{0.87\textwidth}
    \begin{minipage}{120pt}
        \vspace{-1.8cm}
        \includegraphics[width=80pt]{../../logo.pdf}
        \centering
        \small Zusammenkunft aller Physik-Fachschaften
    \end{minipage}
    \begin{center}
        \huge{Positionspapier der Zusammenkunft aller Physik-Fachschaften}\vspace{.25\baselineskip}\\
        \normalsize
    \end{center}
    \vspace{1cm} 
\section*{Zur Förderung der Wissenschaftskommunikation in der akademischen Ausbildung}
Die Zusammenkunft aller Physikfachschaften (ZaPF) ist der Meinung, dass Wissenschaftskommunikation ein elementarer Bestandteil im Studium sein sollte.
Wir sehen dafür unter anderem folgende Stellen im Bachelor- sowie Masterstudium, bei denen Wissenschaftskommunikation stattfinden kann:
\begin{itemize}
    \item [] \textbf{Vortrag der Abschlussarbeiten}: Die ZaPF empfiehlt als Maßnahme, das Thema der eigenen Abschlussarbeit neben einer möglichen Verteidigung vorzustellen, um die Kompetenz, Wissenschaft zu kommunizieren, zu stärken. Sie ist der Meinung, dass ein akademischer Rahmen\footnotemark[1] sinnvoll ist und sich der Lernerfolg durch Erweiterung des Zielpublikums optimiert. Insbesondere für die Masterarbeit wird eine Ordnung für die Allgemeinheit sehr empfohlen.
\item [] \textbf{Eigenständiges Modul}: Die ZaPF empfiehlt das Angebot einer Veranstaltung, die theoretische und praktische Aspekte der Wissenschaftskommunikation vermittelt. Diese sollte mindestens als Wahlpflichtmodul vorkommen. Sinnvoll für die Umsetzung erachten wir ein Seminar und/oder eine Ringvorlesung mit folgenden Inhalten:
\begin{compactitem}
\item Rhetorik
\item Gastvorträge
\item Analyse von Kommunikation anderer (Natur-)Wissenschaftler
\item Vorbereitungsschulungen
\item eigene Präsentation
\item andere wissenschaftskommunikative Arbeit
\end{compactitem}
Ein fakultätenübergreifendes Modul wird ermutigt. Dessen Leitung kann
sowohl von universitären Lehrkräften unterschiedlicher Fachbereiche\footnotemark[2] als auch Mitarbeiter*innen zentraler Einrichtungen\footnotemark[3] oder externen Expert*innen übernommen werden. Die aus der Umsetzung des vorgeschlagenen Konzeptes
resultierende Vernetzung von Studierenden mit anderen Fachbereichen und
in der Forschung ist nur eine der positiven Auswirkungen.
\end{itemize}
Bis zum Erreichen des Masterabschlusses sollte mindestens eine solche Maßnahme durchgeführt worden sein. Die Einbindung dieses Themengebietes in das
Curriculum wird gefordert, um sowohl die Akzeptanz und Wertschätzung von
Wissenschaftskommunikation allgemein, als auch die Identifikation von Studierenden
mit Forschung sowie die Interdisziplinarität zu fördern.

\footnotetext[1]{bspw. Institutskolloqium, Konferenzvortrag,...}
\footnotetext[2]{bspw. Physik, Germanistik, Journalismus, ...}
\footnotetext[3]{bspw. Pressestelle, Kommunikationsbeauftragten, ...}
\vspace{-0.5\baselineskip}
    \begin{flushright}
        Verabschiedet am 31.10.2017 in Siegen
    \end{flushright}
\end{document}
%%% Local Variables:
%%% mode: latex
%%% TeX-master: t
%%% End:

%auch die Indentifikation ... streichen
%andere Möglichkeiten mitaufnehmen Wissenschaft zu kommunizieren.

