\documentclass[DIV=calc]{scrartcl}
\usepackage[utf8]{inputenc}
\usepackage[T1]{fontenc}
\usepackage[ngerman]{babel}
\usepackage{graphicx}

\usepackage{ulem}
\usepackage[dvipsnames]{xcolor}

\usepackage{fixltx2e}
\usepackage{ellipsis}
\usepackage[tracking=true]{microtype}

\usepackage{lmodern}                        % Ersatz fuer Computer Modern-Schriften
\usepackage{hfoldsty}

\usepackage{fourier}             % Schriftart
\usepackage[scaled=0.81]{helvet}     % Schriftart

\usepackage{url}
\usepackage{tocloft}             % Paket f�r Table of Contents

\usepackage{xcolor}
\definecolor{urlred}{HTML}{660000}

\usepackage{hyperref}
\hypersetup{
    colorlinks=true,    
    linkcolor=black,    % Farbe der internen Links (u.a. Table of Contents)
    urlcolor=black,    % Farbe der url-links
    citecolor=black} % Farbe der Literaturverzeichnis-Links

\usepackage{mdwlist}     % �nderung der Zeilenabst�nde bei itemize und enumerate
%\usepackage{draftwatermark} % Wasserzeichen ``Entwurf'' 
%\SetWatermarkText{Entwurf}

\usepackage{blindtext}
\parindent 0pt                 % Absatzeinr�cken verhindern
\parskip 12pt                 % Abs�tze durch L�cke trennen

%\usepackage{titlesec}    % Abstand nach �berschriften neu definieren
%\titlespacing{\subsection}{0ex}{3ex}{-1ex}
%\titlespacing{\subsubsection}{0ex}{3ex}{-1ex}        

% \pagestyle{empty}
\setlength{\textheight}{23cm}
\usepackage{fancyhdr}
\pagestyle{fancy}
\cfoot{}
\lfoot{Zusammenkunft aller Physik-Fachschaften}
\rfoot{www.zapfev.de\\stapf@zapf.in}
\renewcommand{\headrulewidth}{0pt}
\renewcommand{\footrulewidth}{0.1pt}
\newcommand{\gen}{*innen}

\begin{document}
    \hspace{0.87\textwidth}
    \begin{minipage}{120pt}
        \vspace{-1.8cm}
        \includegraphics[width=80pt]{../../logo.pdf}
        \centering
        \small Zusammenkunft aller Physik-Fachschaften
    \end{minipage}
    \begin{center}
        \huge{Resolution der Zusammenkunft aller Physikfachschaften}\vspace{.25\baselineskip}\\
        \normalsize
    \end{center}
    \vspace{1cm}
    %\textbf{Antragssteller}: Oliver Irtenkauf (Uni Konstanz), Torsten Umlauf (Uni Würzburg)\\\\
%\textbf{Adressaten}: KFP, Praktikumsleitertagung, DPG, jDPG, \\\\
\section*{Zu Berufsorientierenden Praktika}
Wir fordern die Fachbereiche  auf, den Studierenden die Möglichkeit zu geben, nicht nur
Forschungs- und Laborpraktika an der eigenen Universität belegen zu können, sondern auch wissenschaftsorientierte
Praktika an anderen Universitäten, in Forschungseinrichtungen und insbesondere auch in der
Industrie anrechnen lassen zu können.

An vielen Universitäten bereitet das Physikstudium vorwiegend auf eine akademische Laufbahn vor.\\
Hierbei haben Studierende jedoch kaum Gelegenheit sich während des regulären Studienverlaufes einen Einblick in mögliche Berufsfelder zu verschaffen.

\vspace{-0.5\baselineskip}
    \begin{flushright}
        Verabschiedet am 01.11.2017 in Siegen
    \end{flushright}
\end{document}
%%% Local Variables:
%%% mode: latex
%%% TeX-master: t
%%% End:
