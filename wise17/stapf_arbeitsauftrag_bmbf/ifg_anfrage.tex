\documentclass[draft,10pt,oneside]{scrartcl}

% Sprache und Encodings
\usepackage[ngerman]{babel}
\usepackage[T1]{fontenc}
\usepackage[utf8]{inputenc}
\usepackage{xcolor}

% Typographisch interessante Pakete
\usepackage{microtype} % Randkorrektur und andere Anpassungen

% References to Internet and within the document
\usepackage[pdftex,colorlinks=false,
pdftitle={Input zur Anfrage an das BMBF nach IFG},
pdfauthor={Jörg Behrmann (FUB), Björn Guth (RWTH)},
pdfcreator={pdflatex},
pdfdisplaydoctitle=true]{hyperref}

% Absaetze nicht Einruecken
\setlength{\parindent}{0pt}
\setlength{\parskip}{2pt}

% Formatierung auf A4 anpassen
\usepackage{geometry}
\geometry{paper=a4paper,left=15mm,right=15mm,top=10mm,bottom=10mm}

\begin{document}

\section*{Input zur Anfrage an das BMBF nach IFG}

\textbf{Antragsteller:} Jennifer Hartfiel (FUB), Björn Guth (RWTH)

\subsection*{Antrag}

Hiermit bitten wir die ZaPF um Input zu folgender Anfrage nach IFG an das BMBF:

\begin{quote}
	\textbf{Antrag nach dem IFG/UIG/VIG}

	Sehr geehrte Damen und Herren,

	bitte senden Sie mir Folgendes zu:

	\begin{itemize}
		\item \textcolor{red}{Die Liste aller Initiativen, die eine Förderung
				im Rahmen des Programms \glqq{}Förderung hochschulbezogener
				zentraler Maßnahmen studentischer Verbände und
				Organisationen\grqq{} für die laufende Förderperiode 2017/2018
				beantragt haben, sowie die beantragten Fördersummen.
			}
		\item \textcolor{red}{Die Liste der bisherigen Zu- und Absagen für
				beantragte Förderung aus dem oben genannten Programm für die
				laufende Förderperiode 2017/2018.
			}
		\item \textcolor{red}{Die Höhe der im Haushalt des BMBF veranschlagten
				Mittel für Förderungen, die Summe aller beantragten
				Förderungen, die Summe aller zugesagten Förderungen, sowie die
				Summe der tatsächlich verausgabten Förderungen. Dies jeweils
				bezogen auf das oben genannte Programm und für die letzten fünf
				abgeschlossenen Förderperioden.
			}
		\item \textcolor{red}{Die Kriterien, nach denen Anträge auf Förderung
				aus oben genanntem Programm innerhalb des BMBF, sowie beim DLR
				als Projektträger bewertet und nach Förderwürdigkeit gereiht
				werden.
			}
	\end{itemize}

	Dies ist ein Antrag auf Aktenauskunft nach § 1 des Gesetzes zur Regelung
	des Zugangs zu Informationen des Bundes (IFG) sowie § 3
	Umweltinformationsgesetz (UIG), soweit Umweltinformationen im Sinne des § 2
	Abs. 3 UIG betroffen sind, sowie § 1 des Gesetzes  zur Verbesserung der
	gesundheitsbezogenen Verbraucherinformation (VIG), soweit Informationen im
	Sinne des § 1 Abs. 1 VIG betroffen sind.

	Ausschlussgründe liegen m.E. nicht vor.

	M.E. handelt es sich um eine einfache Auskunft. Gebühren fallen somit  nach
	§ 10 IFG bzw. den anderen Vorschriften nicht an.  Sollte die Aktenauskunft
	Ihres Erachtens gebührenpflichtig sein, bitte ich, mir dies vorab
	mitzuteilen und dabei die Höhe der Kosten anzugeben.

	Ich verweise auf § 7 Abs. 5 IFG/§ 3 Abs. 3 Satz 2 Nr. 1 UIG/§ 4 Abs. 2 VIG
	und bitte, mir die erbetenen Informationen unverzüglich, spätestens nach
	Ablauf eines Monats zugänglich zu machen.

	Sollten Sie für diesen Antrag nicht zuständig sein, bitte ich, ihn an die
	zuständige Behörde weiterzuleiten und mich darüber zu unterrichten.

	Ich bitte um eine Antwort in elektronischer Form (E-Mail) gemäß § 8 EGovG.
	Eine Antwort an meine persönliche E-Mail-Adresse bei meinem
	Telekommunikationsanbieter FragDenStaat.de stellt keine öffentliche
	Bekanntgabe des Verwaltungsaktes nach § 41 VwVfG dar.

	Ich behalte mir vor, nach Eingang Ihrer Auskünfte um weitere ergänzende
	Auskünfte nachzusuchen.

	Ich bitte um Empfangsbestätigung und danke Ihnen für Ihre Mühe.

	Mit freundlichen Grüßen,

	\textit{Antragsteller/in}
\end{quote}

\textbf{Anmerkung:} Für den Input ist nur der rote Teil des Texts\footnote{die
Aufzählung für schwarz/weiß gedruckte Versionen} interessant, da der Rest
automatisch durch die Anfrage über die Seite fragdenstaat.de generiert wurde.

\end{document}
