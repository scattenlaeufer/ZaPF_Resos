\documentclass[draft,10pt,oneside]{scrartcl}

% Sprache und Encodings
\usepackage[ngerman]{babel}
\usepackage[T1]{fontenc}
\usepackage[utf8]{inputenc}

% Typographisch interessante Pakete
\usepackage{microtype} % Randkorrektur und andere Anpassungen

% References to Internet and within the document
\usepackage[pdftex,colorlinks=false,
pdftitle={Arbeitsauftrag an den StAPF zu Vörderungen durch das BMBF},
pdfauthor={Jörg Behrmann (FUB), Björn Guth (RWTH)},
pdfcreator={pdflatex},
pdfdisplaydoctitle=true]{hyperref}

% Absaetze nicht Einruecken
\setlength{\parindent}{0pt}
\setlength{\parskip}{2pt}

% Formatierung auf A4 anpassen
\usepackage{geometry}
\geometry{paper=a4paper,left=15mm,right=15mm,top=10mm,bottom=10mm}

\begin{document}

\section*{Arbeitsauftrag an den StAPF zu Vörderungen durch das BMBF}

\textbf{Antragsteller:} Jörg Behrmann (FUB), Björn Guth (RWTH), Patrick Haiber
(Uni Konstanz)

\subsection*{Antrag}

Hiermit beantragen wir dem StAPF foldenden Arbeitsauftrag zu erteilen:

\begin{quote}
	Die ZaPF beauftragt den StAPF auf Basis der zu stellenden Anfrage nach
	Informationsfreiheitsgesetz an das BMBF Gespräche mit der MeTaFa, dem BMBF
	und dem DLR über die mittel- und langfristige Zukunft der Finanzierung von
	Bundesfachschaftentagungen aus Mitteln des Topfes für Förderung
	hochschulbezogener zentraler Maßnahmen studentischer Verbände und anderer
	Organisationen aufzunehmen.

	Als Grundlage dieser Gespräche soll das Protokoll des Arbeitskreises zum
	Umgang mit Förderabsagen\footnote{\href{https://zapf.wiki/WiSe17_AK_Förderungsabsagen}{\url{https://zapf.wiki/WiSe17_AK_Förderungsabsagen}}}
	dienen.
\end{quote}

\end{document}
