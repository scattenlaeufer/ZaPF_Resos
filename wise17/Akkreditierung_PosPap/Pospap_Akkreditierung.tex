\documentclass[DIV=calc]{scrartcl}
\usepackage[utf8]{inputenc}
\usepackage[T1]{fontenc}
\usepackage[ngerman]{babel}
\usepackage{graphicx}

\usepackage{ulem}
\usepackage[dvipsnames]{xcolor}
\usepackage{paralist}
\usepackage{fixltx2e}
\usepackage{ellipsis}
\usepackage[tracking=true]{microtype}

\usepackage{lmodern}                        % Ersatz fuer Computer Modern-Schriften
\usepackage{hfoldsty}

\usepackage{fourier}             % Schriftart
\usepackage[scaled=0.81]{helvet}     % Schriftart

\usepackage{url}
\usepackage{tocloft}             % Paket für Table of Contents

\usepackage{xcolor}
\definecolor{urlred}{HTML}{660000}

\usepackage{hyperref}
\hypersetup{
    colorlinks=true,    
    linkcolor=black,    % Farbe der internen Links (u.a. Table of Contents)
    urlcolor=black,    % Farbe der url-links
    citecolor=black} % Farbe der Literaturverzeichnis-Links

\usepackage{mdwlist}     % Änderung der Zeilenabstände bei itemize und enumerate
%\usepackage{draftwatermark} % Wasserzeichen ``Entwurf'' 
%\SetWatermarkText{Entwurf}

\usepackage{blindtext}
\parindent 0pt                 % Absatzeinrücken verhindern
\parskip 12pt                 % Absätze durch Lücke trennen

%\usepackage{titlesec}    % Abstand nach Überschriften neu definieren
%\titlespacing{\subsection}{0ex}{3ex}{-1ex}
%\titlespacing{\subsubsection}{0ex}{3ex}{-1ex}        

% \pagestyle{empty}
\setlength{\textheight}{23cm}
\usepackage{fancyhdr}
\pagestyle{fancy}
\cfoot{}
\lfoot{Zusammenkunft aller Physik-Fachschaften}
\rfoot{www.zapfev.de\\stapf@zapf.in}
\renewcommand{\headrulewidth}{0pt}
\renewcommand{\footrulewidth}{0.1pt}
\newcommand{\gen}{*innen}

\begin{document}
    \hspace{0.87\textwidth}
    \begin{minipage}{120pt}
        \vspace{-1.8cm}
        \includegraphics[width=80pt]{../../logo.pdf}
        \centering
        \small Zusammenkunft aller Physik-Fachschaften
    \end{minipage}
    \begin{center}
        \huge{Positionspapier der Zusammenkunft aller Physikfachschaften}\vspace{.25\baselineskip}\\
        \normalsize
    \end{center}
    \vspace{0cm} 
       
%\textbf{Antragsteller}: Daniela (Frankfurt), Martin (alter Sack)\\\\
%\textbf{Empfänger}: Die Physikfachschaften \\\\
\section*{Zu Akkreditierung}
Die ZaPF beobachtet die aktuellen Entwicklungen
zur Musterrechtsverordnung (MRVO) für das Akkreditierungswesen mit Sorge und macht ausgehend von den Stellungnahmen anderer Beteiligter im Akkreditierungswesen die Fachschaften auf folgende potentiell kritische Änderungen aufmerksam.
Im Besonderen möchten wir die Fachschaften auf folgende Änderungen zu den aktuell geltenden Regeln hinweisen:
\begin{compactitem}
\item Die Notwendigkeit, örtliche Begehungen abzuhalten kann unter anderem bei einer Reakkreditierung entfallen. Dies ist aber die einzige direkte Austauschmöglichkeit zwischen Gutachtern und der betroffenen Fachschaft.
\item Die Dauer, für die Studiengänge und Qualitätssysteme akkreditiert sind, wird insbesondere bei Erstakkreditierungen deutlich erhöht. Dadurch sinkt die Dringlichkeit, Veränderungen vornehmen zu müssen (auf nun immer alle 8 Jahre anstatt wie vorher 5 bis 8 Jahre).
\item Die Aufgabenverteilung zwischen Agenturen und dem Akkreditierungsrat wird zugunsten von letzterem verschoben: Die Akkreditierungsentscheidung liegt nicht mehr bei der agenturinternen Akkreditierungskommission, sondern beim übergeordneten Akkreditierungsrat, welcher die Entscheidung nun allein auf Basis des Berichts der Agenturen fällt. Außerdem ist unklar, wie der Akkreditierungsrat diese Mehrbelastung stemmen soll.
\item Bei den Studienzielen ist die Befähigung zum gesellschaftlichen Engagement entfallen.
\item Die Vielfalt von Studierenden (wie etwa Belange Studierender mit Behinderung oder Studierender mit Kind) wird nur bei Joint Degrees explizit beachtet.
\item Bei den bisherigen Zugangsvoraussetzungen für Masterstudiengänge \glqq Zugangsvoraussetzung für einen Masterstudiengang ist in der Regel ein berufsqualifizierender Hochschulabschluss\grqq{} entfällt das \glqq in der Regel\grqq{}, was beruflich qualifizierten Bewerbern ohne Hochschulabschluss den Zugang erschwert.
\item Gebündelte Akkreditierungen von bis zu 10 Studiengängen in einem Verfahren sind möglich, ohne dass sich die Größe oder Zusammensetzung der Gutachtergruppe oder die Länge des Verfahrens ändert und unterliegt mangels klarer Definitionen kaum Einschränkungen.
\item Die Definition des Vertreters der Berufspraxis in der Gutachtergruppe wird weiter dadurch verwässert, dass diese in Verfahren für Lehramtsstudiengänge durch eine Vertreterin der Obersten Landesbehörde ersetzt werden.
\item Es soll an den Universitäten eine \glqq Lehrverfassung\grqq{} etabliert werden, aber es ist nicht klar, was das genau sein soll und wie diese zustande kommt.
\item Bei akkreditierten Kombinationsstudiengängen können weitere Teilstudiengänge hinzugefügt werden, ohne, dass diese neu begutachtet werden müssen. Insbesondere muss so auf die Studierbarkeit der neuen Teilstudiengänge in Verbindung mit den alten Teilstudiengängen keine gesonderte Rücksicht genommen werden.
\item Es wird die Möglichkeit für alternative Verfahren gegeben, die aber nicht genauer erläutert werden. In solchen Verfahren könnten Uni-interne Gremien leichter umgangen werden.
\item In einem Kommentar zu den Paragraphen der MRVO, die Auflagen regeln, steht, dass Auflagen nun nur noch in Ausnahmefällen ausgesprochen werden sollen. Die Bewertungskriterien werden in formale Kriterien (in Form eines Prüfberichts) und in fachlich-inhaltliche Kriterien (in einem Gutachten) getrennt. Diese Berichte werden von unterschiedlichen Personen erstellt. Die Konsequenzen der Umsetzung dieser Änderung sind unklar.
\end{compactitem}

Anmerkungen zur Einordnung:
Die Bestrebungen zu einer Musterrechtsverordnung für alle Länder wurden in die Wege geleitet, nachdem das Bundesverfassungsgericht die Regeln zur Akkreditierung in NRW als verfassungswidrig erklärte. An der Ausgestaltung waren nur die Hochschulrektorenkonferenz (HRK) und Kultusministerkonferenz (KMK) beteiligt.
Diese Vorgaben sollen durch einen Studienakkreditierungsstaatsvertrag umgesetzt werden, der aktuell den Landesparlamenten zur Beschlussfassung vorliegt. Der Staatsvertrag ermächtigt die Landesregierungen dazu, in einer Rechtsverordnung das Verfahren der Akkreditierung weiter zu konkretisieren. Die Umsetzung soll nach Maßgabe des Bundesverfassungsgerichts bis zum 1. Januar 2018 abgeschlossen sein. Die endgültige Änderung steht noch nicht fest und die Situation ist sehr unübersichtlich zu mal viele Stellen des MRVO unklar sind.
Aktuell wird von vielen anderen Beteiligten im Akkreditierungswesen an Stellungnahmen gearbeitet. Die teils vorläufigen Versionen davon bilden die Basis für die hier vorgebrachten Punkte.
Quellen:
\begin{compactitem}
\item FZS http://www.fzs.de/news/461957.html
\item KASAP tbd
\item KSS tdb
\item Hochschullehrerbund
\item \url{http://hlb.de/fileadmin/hlb-global/downloads/termine_aktuelle_Informationen/2017-09-25-Akkreditierung_Musterrechtverordnung_liegt_vor_fin.pdf}
\end{compactitem}

\vspace{-0.5\baselineskip}
    \begin{flushright}
        Verabschiedet am 31.10.2017 in Siegen
    \end{flushright}
\end{document}
%%% Local Variables:
%%% mode: latex
%%% TeX-master: t
%%% End:

\begin{comment}
\end{comment}
