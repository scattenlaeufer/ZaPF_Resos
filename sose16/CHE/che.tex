\documentclass[DIV=calc]{scrartcl}
\usepackage[utf8]{inputenc}
\usepackage[T1]{fontenc}
\usepackage[ngerman]{babel}
\usepackage{graphicx}

\usepackage{fixltx2e}
\usepackage{ellipsis}
\usepackage[tracking=true]{microtype}

\usepackage{lmodern}                        % Ersatz fuer Computer Modern-Schriften
\usepackage{hfoldsty}

\usepackage{fourier}             % Schriftart
\usepackage[scaled=0.81]{helvet}     % Schriftart

\usepackage{url}
\usepackage{tocloft}             % Paket für Table of Contents

\usepackage{xcolor}
\definecolor{urlred}{HTML}{660000}

\usepackage{hyperref}
\hypersetup{
    colorlinks=true,    
    linkcolor=black,    % Farbe der internen Links (u.a. Table of Contents)
    urlcolor=black,    % Farbe der url-links
    citecolor=black} % Farbe der Literaturverzeichnis-Links

\usepackage{mdwlist}     % Änderung der Zeilenabstände bei itemize und enumerate

\parindent 0pt                 % Absatzeinrücken verhindern
\parskip 12pt                 % Absätze durch Lücke trennen

%\usepackage{titlesec}    % Abstand nach Überschriften neu definieren
%\titlespacing{\subsection}{0ex}{3ex}{-1ex}
%\titlespacing{\subsubsection}{0ex}{3ex}{-1ex}        

% \pagestyle{empty}
\setlength{\textheight}{23cm}
\usepackage{fancyhdr}
\pagestyle{fancy}
\cfoot{}
\lfoot{Zusammenkunft aller Physik-Fachschaften}
\rfoot{www.zapfev.de\\stapf@googlegroups.de}
\renewcommand{\headrulewidth}{0pt}
\renewcommand{\footrulewidth}{0.1pt}
\newcommand{\gen}{*innen}

\begin{document}
    \hspace{0.87\textwidth}
    \begin{minipage}{120pt}
        \vspace{-1.8cm}
        \includegraphics[width=80pt]{logo.pdf}
        \centering
        \small Zusammenkunft aller Physik-Fachschaften
    \end{minipage}
    \begin{center}
        \huge{Positionspapier der Zusammenkunft aller Physik-Fachschaften} \\
        \normalsize
    \end{center}
    
    \vspace{1cm}
    \section*{ZaPF stellt Katalog für Umgang mit den Ranking Ergebnissen durch Presse und Dritte auf}
    
    Die Zusammenkunft aller Physik-Fachschaften (im folgenden „ZaPF“) setzt sich mit den aktuellen Entwicklungen zum Hochschulranking des Centrums für Hochschulentwicklung auseinander. Im Folgenden geht es konkret um die Berichterstattung über die Rankingergebnisse und die mediale Aufarbeitung des Rankings und nicht um dessen Methodik oder Veröffentlichung im ZEIT Studienführer. Es ist eine starke Vereinfachung bzw. Abwandlung von Ergebnissen zu beobachten, die zu falschen Eindrücken oder gar einem gänzlich verfälschten Gesamtbild führen. Aufgrund der undifferenzierten Berichterstattung über die Ergebnisse des CHE-Rankings bezieht die ZaPF Stellung dazu.
    Die ZaPF moniert insbesondere folgende Aspekte:
    
    \begin{itemize}
        \item Die Gewichtung von Indikatoren, beispielsweise in die Kategorien "Haupt- und Nebenindikatoren"
        \item Das eigenständige Erstellen von Gesamtindikatoren aus den einzelnen Indikatoren ohne entsprechende Kennzeichnung.
        \item Das Zusammenfassen der Indikatoren zu einer generellen Rangfolge der Hochschulen.
    \end{itemize}
    
    Die Gewichtung der Kategorien ist nicht Teil des Ranking und ist letztlich Interpretation der Autor\gen. Alle Indikatoren sind als gleichwertig anzusehen. Daher sind insbesondere die Indikatoren der Print-Version nicht wichtiger als alle anderen. Die Zusammenfassung der einzelnen Indikatoren ist eine eigene, auf die Wünsche der jeweiligen Verfasser\gen\ zugeschnittene Analyse und sollte als solche kenntlich gemacht werden. Eine Rangfolge als allgemein gültiges Endergebnis zu präsentieren, torpediert den multidimensionalen Ansatz des Rankings. Potentielle Studienanfänger\gen\ sollen aus der Fülle an Indikatoren eine individuelle Auswertung zusammenstellen können, um passende Studienorte zu finden. Die ZaPF bittet um einen verantwortungsvollen und reflektierten Umgang mit den Ergebnissen des
    CHE-Hochschulrankings. 
    
    Deshalb fordert die ZaPF von Print- und Onlinemedien
    sowie Hochschulen, folgende Aspekte zu beachten:
    
    \begin{itemize}
        \item Alle Indikatoren sind als gleichwertig anzusehen und entsprechend nicht zu
        gewichten.
        \item Die Indikatoren der Online-Version sollen bei der Berichterstattung gleiche
        Beachtung wie die Auswahl der Print-Version finden.
        \item Falls eine Auswahl oder Gewichtung von Indikatoren zur Analyse verwendet
        wird, ist klarzustellen, welche Indikatoren einbezogen wurden und darauf
        hinzuweisen, dass noch weitere Indikatoren existieren.
        \item Ein Zusammenfassen einzelner oder gar aller Indikatoren soll möglichst vermieden oder zumindest als solches deutlich kenntlich gemacht werden.
        \item Auf die Komplexität und Intention des Rankings soll auch bei Analysen hingewiesen werden.
        \item Bei Vergleichen zwischen den Studiengängen sollte immer klar gemacht werden, welche Indikatoren in den Vergleich eingehen.
        \item Von einer Erstellung von Rangfolgen oder -listen als Endergebnis ist abzusehen.
        \item Werden Abbildungen des CHE oder aus dem ZEIT Studienführer verwendet,
        so ist auf Vollständigkeit und korrekte Zitation zu achten und es sollen alle
        relevanten Informationen beigefügt sein. Insbesondere sollen Legenden mit
        veröffentlicht und keine eigenen Elemente in die Grafiken eingefügt werden.
    \end{itemize}
    
    Die ZaPF versteht die Schwierigkeiten der Komplexität des gesamten Rankings
    (insbesondere der Methodik) in einem einzelnen Artikel gerecht zu werden. Die
    ZaPF schätzt die Bemühungen, eine vereinfachte Interpretation und damit leichter
    verständliche Darstellung des Rankings zu liefern, und respektiert die journalistische
    Freiheit der verschiedenen Autor\gen. Die ZaPF sieht jedoch die Gefahr, dass
    durch eine zu unpräzise Vereinfachung die Aussagen des Rankings verfälscht werden
    können, wodurch Leser\gen, allen voran Schüler\gen\ und Studieninteressierte,
    in die Irre geführt werden können. Um diesem vorzubeugen, erachtet es die ZaPF
    als erforderlich, solche Interpretationsversuche immer klar zu kennzeichnen.
    
    Weitere Beschl\"usse und Ver\"offentlichungen der ZaPF (auch zum CHE-Ranking) unter:
    \href{http://www.zapfev.de/zapf/resolutionen/}{www.zapfev.de/zapf/resolutionen}
    
    \vfill
    \begin{flushright}
        Verabschiedet am 08.05.2016 in Konstanz
    \end{flushright}
    
    
    
    
\end{document}
