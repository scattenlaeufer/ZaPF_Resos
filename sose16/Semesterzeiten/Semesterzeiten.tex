\documentclass[DIV=calc]{scrartcl}
\usepackage[utf8]{inputenc}
\usepackage[T1]{fontenc}
\usepackage[ngerman]{babel}
\usepackage{graphicx}

\usepackage{fixltx2e}
\usepackage{ellipsis}
\usepackage[tracking=true]{microtype}

\usepackage{lmodern}                        % Ersatz fuer Computer Modern-Schriften
\usepackage{hfoldsty}

\usepackage{fourier} 			% Schriftart
\usepackage[scaled=0.81]{helvet} 	% Schriftart

\usepackage{url}
\usepackage{tocloft} 			% Paket für Table of Contents

\usepackage{xcolor}
\definecolor{urlred}{HTML}{660000}

\usepackage{hyperref}
\hypersetup{
  colorlinks=true,	
  linkcolor=black,	% Farbe der internen Links (u.a. Table of Contents)
  urlcolor=black,	% Farbe der url-links
  citecolor=black} % Farbe der Literaturverzeichnis-Links

\usepackage{mdwlist} 	% Änderung der Zeilenabstände bei itemize und enumerate

\parindent 0pt 				% Absatzeinrücken verhindern
\parskip 12pt 				% Absätze durch Lücke trennen

\usepackage{titlesec}	% Abstand nach Überschriften neu definieren
\titlespacing{\subsection}{0ex}{3ex}{-1ex}
\titlespacing{\subsubsection}{0ex}{3ex}{-1ex}		

% \pagestyle{empty}
\setlength{\textheight}{23cm}
\usepackage{fancyhdr}
\pagestyle{fancy}
\cfoot{}
\lfoot{Zusammenkunft aller Physik-Fachschaften}
\rfoot{www.zapfev.de\\stapf@googlegroups.de}
\renewcommand{\headrulewidth}{0pt}
\renewcommand{\footrulewidth}{0.1pt}


\begin{document}
\hspace{0.87\textwidth}
\begin{minipage}{120pt}
\vspace{-1.8cm}
\includegraphics[width=80pt]{logo.pdf}
\centering
\small Zusammenkunft aller Physik-Fachschaften
\end{minipage}
\begin{center}
\huge{Positionspapier der Zusammenkunft aller Physik-Fachschaften} \\
\normalsize
\end{center}

\vspace{1cm}
\section*{Internationale Semesterzeiten}

Die aktuellen Semesterzeiten behindern eine weitere Internationalisierung der deutschen Hochschulen. Aus diesem Grund spricht sich die ZaPF daf\"ur aus, die Semester- und Vorlesungszeiten in Deutschland und Europa anzugleichen. Dabei empfiehlt sie, dass sich alle deutschen Bundesl\"ander und Universit\"aten an der Umsetzung beteiligen. \\
Der Vorteil darin l\"age in der erh\"ohten Mobilit\"at aller Studierenden, welche durch angepasste Semesterzeiten deutschland- und europaweit besser gew\"ahrleistet w\"urde. Zudem k\"onnen dadurch internationale Tagungen, Praktika und Summer Schools von allen Angeh\"origen einer Hochschule leichter wahrgenommen werden.\\
Die Universit\"at Mannheim zeigt bereits die Machbarkeit dieser Umstellung, indem sie ihre Semester- und Vorlesungszeiten f\"ur jeweils das Sommer- und Wintersemester vorverlegt hat.

Aufgrund der Erfahrungen dort und der \"Uberlegungen der HRK w\"are eine Verschiebung um mindestens zwei, idealerweise vier Wochen anzustreben.

\vfill
\begin{flushright}
Verabschiedet am 08.05.2016 in Konstanz
\end{flushright}




\end{document}
