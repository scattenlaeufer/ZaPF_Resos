\documentclass[DIV=calc]{scrartcl}
\usepackage[utf8]{inputenc}
\usepackage[T1]{fontenc}
\usepackage[ngerman]{babel}
\usepackage{graphicx}

\usepackage{fixltx2e}
\usepackage{ellipsis}
\usepackage[tracking=true]{microtype}

\usepackage{lmodern}                        % Ersatz fuer Computer Modern-Schriften
\usepackage{hfoldsty}

\usepackage{fourier} 			% Schriftart
\usepackage[scaled=0.81]{helvet} 	% Schriftart

\usepackage{url}
\usepackage{tocloft} 			% Paket für Table of Contents

\usepackage{xcolor}
\definecolor{urlred}{HTML}{660000}

\usepackage{hyperref}
\hypersetup{
  colorlinks=true,	
  linkcolor=black,	% Farbe der internen Links (u.a. Table of Contents)
  urlcolor=black,	% Farbe der url-links
  citecolor=black} % Farbe der Literaturverzeichnis-Links

\usepackage{mdwlist} 	% Änderung der Zeilenabstände bei itemize und enumerate

\parindent 0pt 				% Absatzeinrücken verhindern
\parskip 12pt 				% Absätze durch Lücke trennen

\usepackage{titlesec}	% Abstand nach Überschriften neu definieren
\titlespacing{\subsection}{0ex}{3ex}{-1ex}
\titlespacing{\subsubsection}{0ex}{3ex}{-1ex}		

% \pagestyle{empty}
\setlength{\textheight}{23cm}
\usepackage{fancyhdr}
\pagestyle{fancy}
\cfoot{}
\lfoot{Zusammenkunft aller Physik-Fachschaften}
\rfoot{www.zapfev.de\\stapf@googlegroups.de}
\renewcommand{\headrulewidth}{0pt}
\renewcommand{\footrulewidth}{0.1pt}


\begin{document}
\hspace{0.87\textwidth}
\begin{minipage}{120pt}
\vspace{-1.8cm}
\includegraphics[width=80pt]{logo.pdf}
\centering
\small Zusammenkunft aller Physik-Fachschaften
\end{minipage}
\begin{center}
\huge{Resolution der Zusammenkunft aller Physik-Fachschaften} \\
\normalsize
\end{center}

\vspace{1cm}
\section*{Ver\"offentlichungspflicht bei Drittmittelforschung}

Die ZaPF sieht die besondere Bedeutung von Drittmitteln für die Forschung an \"offentlichen Einrichtungen. Auch wird der Gedanke, dass Forschung dem Allgemeinwohl dienen soll, als wichtig erachtet. Deswegen fordert die ZaPF, dass die Ergebnisse von Drittmittelforschung an \"offentlich finanzierten Einrichtungen der Allgemeinheit in leichtzugänglicher Form zur Verfügung gestellt werden müssen. Als Ergebnisse, zu denen die \"Offentlichkeit Zugang erhalten soll, sehen wir neben wissenschaftlichen Abschlussarbeiten (insbesondere Promotion und Habilitation) und Berichten auch die Resultate von abgeschlossenen Forschungsprojekten. Eine m\"ogliche Sperrfrist muss zeitlich beschränkt sein. Wir empfehlen einen Zeitraum von zwei Jahren.\\

\underline{Begr\"undung:}\\\\
Drittmittelforschung macht heute einen bedeutenden Teil der Arbeit an \"offentlichen Forschungseinrichtungen aus. Es entsteht das Problem, dass Ergebnisse und Abschlussarbeiten bei industriegef\"orderter Forschung teils mit Sperrvermerken versehen werden. Dies hat zur Folge, dass die \"Offentlichkeit keinen Zugriff darauf hat und Abschlussarbeiten als pers\"onliche Leistung nicht verwendet werden k\"onnen. Da bei der Durchführung von wissenschaftlicher Forschung an \"offentlichen Einrichtungen immer staatlich finanzierte Infrastruktur und Ressourcen mitgenutzt werden, erachten wir es als notwendig, dass die Allgemeinheit auch Zugang zu den Ergebnissen der durch sie unterstützten Forschung erhält. Uns ist bewusst, dass Unternehmen ein wirtschaftliches Interesse an den Ergebnissen der gef\"orderten Forschung haben. Um den Unternehmen die n\"otige Zeit für die wirtschaftliche Verwertung sowie für die Vorbereitung einer Ver\"offentlichung zu geben, erkennen wir die Notwendigkeit einer angemessenen Frist an.

\vfill
\begin{flushright}
Verabschiedet am 08.05.2016 in Konstanz
\end{flushright}




\end{document}

