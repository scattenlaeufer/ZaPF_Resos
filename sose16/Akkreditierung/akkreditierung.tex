\documentclass[DIV=calc]{scrartcl}
\usepackage[utf8]{inputenc}
\usepackage[T1]{fontenc}
\usepackage[ngerman]{babel}
\usepackage{graphicx}

\usepackage{fixltx2e}
\usepackage{ellipsis}
\usepackage[tracking=true]{microtype}

\usepackage{lmodern}                        % Ersatz fuer Computer Modern-Schriften
\usepackage{hfoldsty}

\usepackage{fourier}             % Schriftart
\usepackage[scaled=0.81]{helvet}     % Schriftart

\usepackage{url}
\usepackage{tocloft}             % Paket für Table of Contents

\usepackage{xcolor}
\definecolor{urlred}{HTML}{660000}

\usepackage{hyperref}
\hypersetup{
    colorlinks=true,    
    linkcolor=black,    % Farbe der internen Links (u.a. Table of Contents)
    urlcolor=black,    % Farbe der url-links
    citecolor=black} % Farbe der Literaturverzeichnis-Links

\usepackage{mdwlist}     % Änderung der Zeilenabstände bei itemize und enumerate

\parindent 0pt                 % Absatzeinrücken verhindern
\parskip 12pt                 % Absätze durch Lücke trennen

%\usepackage{titlesec}    % Abstand nach Überschriften neu definieren
%\titlespacing{\subsection}{0ex}{3ex}{-1ex}
%\titlespacing{\subsubsection}{0ex}{3ex}{-1ex}        

% \pagestyle{empty}
\setlength{\textheight}{23cm}
\usepackage{fancyhdr}
\pagestyle{fancy}
\cfoot{}
\lfoot{Zusammenkunft aller Physik-Fachschaften}
\rfoot{www.zapfev.de\\stapf@googlegroups.de}
\renewcommand{\headrulewidth}{0pt}
\renewcommand{\footrulewidth}{0.1pt}
\newcommand{\gen}{*innen}

\begin{document}
    \hspace{0.87\textwidth}
    \begin{minipage}{120pt}
        \vspace{-1.8cm}
        \includegraphics[width=80pt]{logo.pdf}
        \centering
        \small Zusammenkunft aller Physik-Fachschaften
    \end{minipage}
    \begin{center}
        \huge{Positionspapier der Zusammenkunft aller Physik-Fachschaften} \\
        \normalsize
    \end{center}
    
    \vspace{1cm}
    \section*{Positionspapier zum deutschen Akkreditierungssystem}
    
    Die ZaPF spricht sich f\"ur ein gutachter{\gen}zentriertes Verfahren aus, das
    der Qualit\"atspr\"ufung von Studieng\"angen und Qualit\"atsmanagementsystemen zur
    Erstellung und Weiterentwicklung von Studieng\"angen dient.
    
    Zur Gew\"ahrleistung von objektiven und qualitativ hochwertigen Verfahren ist
    eine hohe Qualifizierung sowie die Unabh\"angigkeit der Gutachter\gen\
    notwendig. Die Zusammensetzung der Gutachter{\gen}gruppe aus Mitgliedern
    aller relevanten Interessengruppen sichert die bestm\"ogliche Evaluierung.
    Insbesondere erachtet es die ZaPF als wichtig, dass Studierende sowohl im
    Verfahren selbst als auch in der endg\"ultigen Entscheidungsfindung als
    vollwertige Mitglieder vertreten sind.
    
    Die Umsetzung dieser Aspekte z\"ahlt zu den St\"arken des aktuellen deutschen
    Akkreditierungssystems.
    
    Im Gegensatz dazu kritisiert die ZaPF folgende Punkte, die im Wesentlichen
    aus dem offenen Wettbewerb der verschiedenen Akkreditierungsagenturen
    resultieren:
    
    \begin{itemize}
        \item Die Qualit\"at der Verfahren leidet unter dem Preisdruck der
        Agenturen. Dies ist beispielsweise der Fall, wenn Studieng\"ange aus
        Einsparungs- anstatt fachlichen Gr\"unden in oftmals zu großen Paketen zu
        gemeinsamen Verfahren zusammengefasst werden.
        \item Außerdem  besteht das Risiko, dass wirtschaftliche Interessen, bedingt durch den Konkurrenzdruck der Agenturen, Akkreditierungsentscheidungen beeinflussen k\"onnen.
        \item Der Entscheidungsspielraum der unterschiedlichen Agenturen
        hinsichtlich ihrer Struktur und der Verfahrensgestaltung ist zu
        groß. Die Vergleichbarkeit der Akkreditierungsverfahren und die
        Transparenz des Akkreditierungswesens werden dadurch gef\"ahrdet.
        Beispielsweise fehlt eine einheitliche Regelung f\"ur die Auswahl und 
        Zusammensetzung von Akkreditierungskommissionen. Da diese f\"ur ein
        Verfahren das entscheidungsf\"allende Organ darstellen, sollte auch
        ihre Zusammensetzung einheitlich gestaltet werden. Auch die nach
        außen sichtbaren Ergebnisse der Verfahren, die Gutachten, sind
        derzeit zu uneinheitlich gestaltet. Sie unterscheiden sich teilweise
        wesentlich in Aufbau und Umfang und sind somit kaum durch die
        Hochschul\"offentlichkeit vergleichbar.
        \item Die Agenturen sind als
        gemeinn\"utzige Vereine oder  Stiftungen organisiert, in denen
        auch Hochschulen Mitglied sein k\"onnen.  Dadurch kann eine
        Befangenheit der Agenturen gegen\"uber bestimmten  Hochschulen,
        insbesondere bei Systemakkreditierungen, nicht  ausgeschlossen
        werden.
    \end{itemize}
    
    Die ZaPF strebt eine Ver\"anderung des Akkreditierungssystems unter
    Ber\"ucksichtigung der oben genannten Kritikpunkte an.
    
    \vfill
    \begin{flushright}
        Verabschiedet am 08.05.2016 in Konstanz
    \end{flushright}
    
    
    
    
\end{document}
