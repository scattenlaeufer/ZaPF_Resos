\documentclass[DIV=calc]{scrartcl}
\usepackage[utf8]{inputenc}
\usepackage[T1]{fontenc}
\usepackage[ngerman]{babel}
\usepackage{graphicx}

\usepackage{fixltx2e}
\usepackage{ellipsis}
\usepackage[tracking=true]{microtype}

\usepackage{lmodern}                        % Ersatz fuer Computer Modern-Schriften
\usepackage{hfoldsty}

\usepackage{fourier} 			% Schriftart
\usepackage[scaled=0.81]{helvet} 	% Schriftart

\usepackage{url}
\usepackage{tocloft} 			% Paket für Table of Contents

\usepackage{xcolor}
\definecolor{urlred}{HTML}{660000}

\usepackage{hyperref}
\hypersetup{
  colorlinks=true,	
  linkcolor=black,	% Farbe der internen Links (u.a. Table of Contents)
  urlcolor=black,	% Farbe der url-links
  citecolor=black} % Farbe der Literaturverzeichnis-Links

\usepackage{mdwlist} 	% Änderung der Zeilenabstände bei itemize und enumerate

\parindent 0pt 				% Absatzeinrücken verhindern
\parskip 12pt 				% Absätze durch Lücke trennen

\usepackage{titlesec}	% Abstand nach Überschriften neu definieren
\titlespacing{\subsection}{0ex}{3ex}{-1ex}
\titlespacing{\subsubsection}{0ex}{3ex}{-1ex}		

% \pagestyle{empty}
\setlength{\textheight}{23cm}
\usepackage{fancyhdr}
\pagestyle{fancy}
\cfoot{}
\lfoot{Zusammenkunft aller Physik-Fachschaften}
\rfoot{www.zapfev.de\\stapf@googlegroups.de}
\renewcommand{\headrulewidth}{0pt}
\renewcommand{\footrulewidth}{0.1pt}


\begin{document}
\hspace{0.87\textwidth}
\begin{minipage}{120pt}
\vspace{-1.8cm}
\includegraphics[width=80pt]{logo.pdf}
\centering
\small Zusammenkunft aller Physik-Fachschaften
\end{minipage}
\begin{center}
\huge{Positionspapier der Zusammenkunft aller Physik-Fachschaften} \\
\normalsize
\end{center}

\vspace{1cm}
\section*{Fachdidaktik im Lehramtsstudium}

\textbf{Motivation:}

Ausgehend von folgenden Stellungnahmen des Nationalen MINT-Forums, der DPG und der Expertenkommission des Landes Nordrhein-Westfalen

\glqq Lehrkräfte stehen vor sich ständig ändernden Anforderungen. Physikunterricht findet in Lerngruppen statt, die ebenso wie die Rahmenbedingungen Veränder\-ungen unterworfen sind. [...] Für diese Ausbildung ist in der ersten Phase ein eigenständiger physikdidaktischer Studienbereich erforderlich, in dem die Lehre durch mindestens eine Fachdidaktikprofessur forschungsbasiert vertreten wird. Die fachdidaktischen Institute bzw. Arbeitsgruppen bilden hierbei die Brücke zwischen Fach, Erziehungswissenschaft und Berufspraxis.\grqq\footnote{Positionspapier des Fachverbands Didaktik der Physik der Deutschen Physikalischen Gesellschaft, Stand Mai 2015.}

\glqq Die Fachdidaktik ist als „Berufswissenschaft der Lehrkräfte“ zentral für die
Lehramtsausbildung. [...] Da [Fachdidaktikerinnen und Fachdidaktiker] gleichzeitig in Inhalten und Methoden von Forschung und Lehre den Erziehungswissenschaften nahe stehen, können sie eine Klammerfunktion in der Lehramtsausbildung einnehmen. Im Zentrum der Fachdidaktik steht eine gute Ausbildung für den Unterricht an Schulen. Gerade im MINT-Bereich ist die Ausstattung mit Fachdidaktiken aber oftmals dünn, was besonders dann gilt, wenn man sie in Beziehung zur Zahl der Lehramtsstudierenden setzt. [...] Jedes Fach, das Lehrkräfte ausbildet, muss daher über mindestens eine reguläre Professur für Fachdidaktik verfügen, die ausgezeichnet ausgestattet ist, um sich so den vielfältigen Aufgaben widmen und die Interessen der Lehramtsausbildung im jeweiligen Fachbereich mit dem notwendigen Gewicht vertreten zu können. Im Rahmen einer solchen Professur ist einerseits eine interdisziplinär anschlussfähige Forschung zu leisten, andererseits in der Lehre auf die Verbindung der unterschiedlichen Bereiche zu achten, sodass eine solide, fachbezogene Ausbildung stattfinden kann.\grqq\footnote{Nationales MINT Forum (Hrsg.): Zehn Thesen und Forderungen zur MINT-Lehramtsausbildung – Empfehlungen des Nationalen MINT Forums Nr. 1, München: Herbert Utz Verlag 2013.}

\glqq Die Fachdidaktiken sind an vielen Standorten nur unzureichend ausgebaut. [...] [Die Kommission] rät dringend, auch bei schlechter Bewerberlage in einigen Fachdidaktiken hohe wissenschaftliche Standards an die Besetzung entsprechender Professuren anzulegen und der Versuchung zu widerstehen, ausgeschriebene Stellen unterqualifiziert oder nicht einschlägig zu besetzen (z.B. mit Fachwissenschaftlern, die in der fachdidaktischen Forschung nicht ausgewiesen sind). Über längere Zeit werden Überbrückungsmaßnahmen, die aber keinesfalls zu Dauerlösungen werden dürfen, notwendig sein.\grqq\footnote{Ausbildung von Lehrerinnen und Lehrern in
Nordrhein-Westfalen - Empfehlungen der Expertenkommission zur Ersten Phase, AQAS e.V. und Ministerium für Innovation, Wissenschaft, Forschung und Technologie des Landes Nordrhein-Westfalen und Ministerium für Schule und Weiterbildung des Landes Nordrhein-Westfalen, Bonn 2007.},positioniert sich die Zusammenkunft aller deutschsprachigen Physik-Fachschaften wie folgt:
\begin{itemize}
\item An allen lehramtsausbildenden Universitäten sollte wenigstens eine fachdidaktische Planstelle besetzt werden. Diese Stelle sollte auch bei schwieriger Bewerber*innenlage ausschließlich mit qualifizierten Fachdidaktiker*innen besetzt werden.
\item Insbesondere in der ersten Ausbildungsphase\footnote{Studium an der Universität.} soll die Fachdidaktik bereits vermittelt werden und nicht in die zweite Phase\footnote{Referendariat.} geschoben werden. Pädagogische und didaktische Begleitung muss auch während der Praxisphasen gewährleistet sein.
\item Um den fachdidaktischen Austausch und über die Ausbildung hinausgehende Qualifizierungen der Studierenden zu fördern, sind fachdidaktische Summer Schools\footnote{Außercurriculare Qualifizierungsveranstaltung im Block/Workshop.} und Kolloquien erstrebenswert.
\end{itemize}

\vfill
\begin{flushright}
Verabschiedet am 08.05.2016 in Konstanz
\end{flushright}




\end{document}

