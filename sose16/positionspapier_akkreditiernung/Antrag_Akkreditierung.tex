\documentclass[draft,10pt,oneside]{scrartcl}

% Sprache und Encodings
\usepackage[ngerman]{babel}
\usepackage[T1]{fontenc}
\usepackage[utf8]{inputenc}

% Typographisch interessante Pakete
\usepackage{microtype} % Randkorrektur und andere Anpassungen

% References to Internet and within the document
\usepackage[pdftex,colorlinks=false,
pdftitle={Antrag zur Änderung der Geschäftsordnung für Plenen der ZaPF},
pdfauthor={Jörg Behrmann (FUB)},
pdfcreator={pdflatex},
pdfdisplaydoctitle=true]{hyperref}

% Absaetze nicht Einruecken
\setlength{\parindent}{0pt}
\setlength{\parskip}{2pt}

% Formatierung auf A4 anpassen
\usepackage{geometry}
\geometry{paper=a4paper,left=15mm,right=15mm,top=10mm,bottom=10mm}

\begin{document}

\section*{Positionspapier zum deutschen Akkerditierungssystem}

\textbf{Antragsteller:} Anna ??? (TU Wien), Björn Guth (RWTH), Jenny Hartfiel (FUB), Jörg Behrmann (FUB), Margret Heinze (LMU)

\subsection*{Antrag}

Die ZaPF möge beschließen:

\begin{quote}
	Die ZaPF spricht sich für ein gutachter*innenzentriertes Verfahren aus, das
	der Qualitätsprüfung von Studiengängen und Qualitätsmanagementsystemen zur
	Erstellung und Weiterentwicklung von Studiengängen dient.

	Zur Gewährleistung von objektiven und qualitativ hochwertigen Verfahren ist
	eine hohe Qualifizierung sowie die Unabhängigkeit der Gutachter*innen
	notwendig. Die Zusammensetzung der Gutachter*innengruppe aus Mitgliedern
	aller relevanten Interessengruppen sichert die bestmögliche Evaluierung.
	Insbesondere erachtet es die ZaPF als wichtig, dass Studierende sowohl im
	Verfahren selbst, als auch in der endgültigen Entscheidungsfindung als
	vollwertige Mitglieder vertreten sind.

	Die Umsetzung dieser Aspekte zählt zu den Stärken des aktuellen deutschen
	Akkreditierungssystems.

	Im Gegensatz dazu kritisiert die ZaPF folgende Punkte, die im Wesentlichen
	aus dem offenen Wettbewerb der verschiedenen Akkreditierungsagenturen
	resultieren:

	\begin{itemize}
		\item Die Qualität der Verfahren leidet unter dem Preisdruck der
			Agenturen. Dies ist beispielsweise der Fall, wenn Studiengänge aus
			Einsparungs- anstatt fachlichen Gründen in oftmals zu großen Paketen zu
			gemeinsamen Verfahren zusammengefasst werden.
		\item Außerdem  besteht das
			Risiko,  dass wirtschaftliche Interessen, bedingt durch den Konkurrenzdruck
			der  Agenturen, Akkreditierungsentscheidungen beeinflussen können.
		\item Der Entscheidungsspielraum der unterschiedlichen Agenturen
			hinsichtlich ihrer Struktur und der Verfahrensgestaltung ist zu
			groß. Die Vergleichbarkeit der Akkreditierungsverfahren und die
			Transparenz des Akkreditierungswesens werden dadurch gefährdet.
			Beispielsweise fehlt eine einheitliche Regelung für die Auswahl und 
			Zusammensetzung von Akkreditierungskommissionen. Da diese für ein
			Verfahren das entscheidungsfällende Organ darstellen, sollte auch
			ihre Zusamensetzung einheitlich gestaltet werden. Auch die nach
			außen sichtbaren Ergebnisse der Verfahren, die Gutachten, sind
			derzeit zu uneinheitlich gestaltet. Sie unterscheiden sich teilweise
			wesentlich in Aufbau und Umfang und sind somit kaum durch die
			Hochschulöffentlichkeit vergleichbar.
		\item Die Agenturen sind als
			gemeinnützige Vereine oder  Stiftungen organisiert, in denen
			auch Hochschulen Mitglied sein können.  Dadurch kann eine
			Befangenheit der Agenturen gegenüber bestimmten  Hochschulen,
			insbesondere bei Systemakkreditierungen, nicht  ausgeschlossen
			werden.
	\end{itemize}

	Die ZaPF strebt eine Verbesserung des Akkreditierungssystems unter
	Berücksichtigung der oben genannten Kritikpunkte an.
\end{quote}

\subsection*{Begründung}

entfällt aufgrund der späten Stunde


\end{document}
