%---------------------------------------------------------------------------
% scrlttr2.tex v0.3. (c) by Juergen Fenn <juergen.fenn@gmx.de>
% Template for a letter to be typeset with scrlttr2.cls from KOMA-Script.
% Latest version of the LaTeX Project Public License is applicable. 
% File may not be modified and redistributed under the same name 
% without the author's prior consent.
%---------------------------------------------------------------------------
\documentclass%%
%---------------------------------------------------------------------------
  [fontsize=12pt,%%          Schriftgroesse
%---------------------------------------------------------------------------
% Satzspiegel
   paper=a4,%%               Papierformat
   enlargefirstpage=on,%%    Erste Seite anders
   firstfoot=off,
   pagenumber=right,%%   Seitenzahl oben mittig
%---------------------------------------------------------------------------
% Layout
   headsepline=off,%%         Linie unter der Seitenzahl
   parskip=half,%%           Abstand zwischen Absaetzen
%---------------------------------------------------------------------------
% Briefkopf und Anschrift
   fromalign=right,%%        Plazierung des Briefkopfs
   fromphone=off,%%           Telefonnummer im Absender
   fromrule=off,%%           Linie im Absender (aftername, afteraddress)
   fromfax=off,%%            Faxnummer
   fromemail=on,%%          Emailadresse
   fromurl=off,%%            Homepage
   fromlogo=off,%%           Firmenlogo
   addrfield=on,%%           Adressfeld fuer Fensterkuverts
   backaddress=on,%%          ...und Absender im Fenster
   subject=beforeopening,%%  Plazierung der Betreffzeile
   locfield=narrow,%%        zusaetzliches Feld fuer Absender
   foldmarks=on,%%           Faltmarken setzen
   numericaldate=off,%%      Datum numerisch ausgeben
   refline=narrow,%%         Geschaeftszeile im Satzspiegel
%---------------------------------------------------------------------------
% Formatierung
   draft=off,%%                Entwurfsmodus
   %backaddress=plain%%        Keine Linie unter Absender
  DIV=12
]{scrlttr2}
%---------------------------------------------------------------------------
\usepackage[english,ngerman]{babel}
\usepackage[T1]{fontenc}
\usepackage[utf8]{inputenc}
\usepackage{url}
\usepackage{lmodern}
\usepackage{csquotes}
\usepackage{pdfpages}
\usepackage{graphics}
\usepackage{pdfpages}
%---------------------------------------------------------------------------
% Fonts
\setkomafont{fromname}{\sffamily}
\setkomafont{fromaddress}{\sffamily}%% statt \small
\setkomafont{pagenumber}{\sffamily}
\setkomafont{subject}{\mdseries}
\setkomafont{backaddress}{\mdseries}
\usepackage{mathptmx}%% Schrift Times
%\usepackage{mathpazo}%% Schrift Palatino
%\setkomafont{fromname}{\LARGE}
%---------------------------------------------------------------------------
\usepackage{lastpage}

\renewcommand*{\pagemark}{{\usekomafont{pagenumber}{%
    \pagename\
    \thepage\ von\ \pageref{LastPage}}}}

\begin{document}
%---------------------------------------------------------------------------
% Briefstil und Position des Briefkopfs
\LoadLetterOption{DIN} %% oder: DINmtext, SN, SNleft, KOMAold.
\makeatletter
\@setplength{firstheadvpos}{20mm}
\@setplength{firstheadwidth}{\paperwidth}
\ifdim \useplength{toaddrhpos}>\z@
  \@addtoplength[-2]{firstheadwidth}{\useplength{toaddrhpos}}
\else
  \@addtoplength[2]{firstheadwidth}{\useplength{toaddrhpos}}
\fi
\@setplength{foldmarkhpos}{6.5mm}
\@addtoplength{firstfootvpos}{-10mm}
\makeatother
%---------------------------------------------------------------------------
% Absender
\setkomavar{fromlogo}{\vspace{-1cm}\includegraphics[width=4cm]{logo.pdf}}
\setkomavar{backaddress}{ZaPF e.V.\\Max-von-Laue-Str. 1\\60438 Frankfurt / Main}
\setkomavar{fromname}{\includegraphics[width=2cm]{logo.pdf}\\Zusammenkunft aller Physik-Fachschaften\\c/o ZaPF e.V.\\Goethe Universität Frankfurt\\Raum \_\_.208}
\setkomavar{fromaddress}{Max-von-Laue-Str. 1\\60438 Frankfurt / Main}
%\setkomavar{fromphone}{}
%\renewcommand{\phonename}{Telefon}
\setkomavar{fromemail}{stapf@googlegroups.com}
\setkomavar{backaddressseparator}{ – }
\setkomavar{signature}{Karola Schulz\\Sprecherin des StAPF}
%\setkomavar{frombank}{}
%\setkomavar{location}{\\[8ex]\raggedleft{\footnotesize{\usekomavar{fromaddress}\\
%      Telefon:\ usekomavar{fromphone}}}}%% Neben dem Adressfenster
%---------------------------------------------------------------------------
%\firsthead{Frei gestalteter Briefkopf}
%---------------------------------------------------------------------------
%\firstfoot{}
%---------------------------------------------------------------------------
% Geschaeftszeilenfelder
\setkomavar{place}{Berlin}
\setkomavar{placeseparator}{, den }
\setkomavar{date}{\today}
%\setkomavar{yourmail}{1. 1. 2003}%% 'Ihr Schreiben...'
%\setkomavar{yourref} {abcdefg}%%    'Ihr Zeichen...'
%\setkomavar{myref}{Test}%%      Unser Zeichen
%\setkomavar{invoice}{123}%% Rechnungsnummer
%\setkomavar{phoneseparator}{}
%---------------------------------------------------------------------------
% Versendungsart
%\setlengthtoplength{\fill}{specialmailindent}
%\setkomavar{specialmail}{Einschreiben mit Rückschein}
%---------------------------------------------------------------------------
% Anlage neu definieren
\renewcommand{\enclname}{Anlagen}
\setkomavar{enclseparator}{: }
%---------------------------------------------------------------------------
% Seitenstil
%\pagestyle{plain}%% keine Header in der Kopfzeile
%---------------------------------------------------------------------------
\newcommand{\massmail}[2]{
\begin{letter}{#1}
%---------------------------------------------------------------------------
% Weitere Optionen
\KOMAoptions{%%
}
%---------------------------------------------------------------------------
\setkomavar{subject}{\textbf{Resolution zu Zwei-Klassen-Studiensystemen}}
%---------------------------------------------------------------------------
\opening{#2}

die Zusammenkunft aller Physik-Fachschaften hat auf ihrer letzten Tagung am 08.05.2016 in Konstanz die folgende Resolution zu Zwei-Klassen-Studiensystemen verabschiedet.

Für Kommentare und Rückfragen stehen wir Ihnen jederzeit zur Verfügung und verbleiben 

Mit freundlichen Grüßen,

\closing{Im Auftrag der ZaPF}
%---------------------------------------------------------------------------
%\ps{PS: 134}
%\encl{}
%\cc{Die Menschheit}
%---------------------------------------------------------------------------

\newpage
\includepdf[pages=-]{Zweiklassenstudiensysteme.pdf}

\end{letter}
}
\massmail{Hochschulrektorenkonferenz\\Ahrstraße 39\\53175 Bonn}{Sehr geehrte Damen und Herren,}
\massmail{Wolfgang Tiefensee\\Max-Reger-Straße 4-8\\99096 Erfurt }{Sehr geehrter Herr Minister Tiefensee,}






%---------------------------------------------------------------------------
\end{document}
%---------------------------------------------------------------------------
