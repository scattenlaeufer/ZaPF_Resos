\documentclass[DIV=calc]{scrartcl}
\usepackage[utf8]{inputenc}
\usepackage[T1]{fontenc}
\usepackage[ngerman]{babel}
\usepackage{graphicx}

\usepackage{fixltx2e}
\usepackage{ellipsis}
\usepackage[tracking=true]{microtype}

\usepackage{lmodern}                        % Ersatz fuer Computer Modern-Schriften
\usepackage{hfoldsty}

\usepackage{fourier} 			% Schriftart
\usepackage[scaled=0.81]{helvet} 	% Schriftart

\usepackage{url}
\usepackage{tocloft} 			% Paket für Table of Contents

\usepackage{xcolor}
\definecolor{urlred}{HTML}{660000}

\usepackage{hyperref}
\hypersetup{
  colorlinks=true,	
  linkcolor=black,	% Farbe der internen Links (u.a. Table of Contents)
  urlcolor=black,	% Farbe der url-links
  citecolor=black} % Farbe der Literaturverzeichnis-Links

\usepackage{mdwlist} 	% Änderung der Zeilenabstände bei itemize und enumerate

\parindent 0pt 				% Absatzeinrücken verhindern
\parskip 12pt 				% Absätze durch Lücke trennen

\usepackage{titlesec}	% Abstand nach Überschriften neu definieren
\titlespacing{\subsection}{0ex}{3ex}{-1ex}
\titlespacing{\subsubsection}{0ex}{3ex}{-1ex}		

% \pagestyle{empty}
\setlength{\textheight}{23cm}
\usepackage{fancyhdr}
\pagestyle{fancy}
\cfoot{}
\lfoot{Zusammenkunft aller Physik-Fachschaften}
\rfoot{www.zapfev.de\\stapf@googlegroups.de}
\renewcommand{\headrulewidth}{0pt}
\renewcommand{\footrulewidth}{0.1pt}


\begin{document}
\hspace{0.87\textwidth}
\begin{minipage}{120pt}
\vspace{-1.8cm}
\includegraphics[width=80pt]{logo.pdf}
\centering
\small Zusammenkunft aller Physik-Fachschaften
\end{minipage}
\begin{center}
\huge{Resolution der Zusammenkunft aller Physik-Fachschaften} \\
\normalsize
\end{center}

\vspace{1cm}
\section*{Resolution zu Zwei-Klassen-Studiensystemen}


Wir begrüßen grundsätzlich die Weiterentwicklung von Studiengängen. Dafür ist jedoch Zeit und besondere Sorgfalt notwendig.

Konkret stellt die ZaPF folgende Forderungen:
\begin{itemize}
        \item Die Studierendenschaften und Fachschaften müssen in die entsprechenden Prozesse von Anfang an einbezogen werden.
        \item Es dürfen keine Zwei-Klassen-Studiensysteme geschaffen werden.
        \item Ein unkompliziertes  Anrechnen von Leistungen zwischen neuen und bereits existierenden Studiengängen muss möglich sein.
\end{itemize}

Auf Grundlage dieser Forderungen möchten wir besonders die überhastete Neugestaltung von ingenieurswissenschaftlichen
Diplomstudiengängen in Thüringen, beziehungsweise und speziell in Ilmenau kritisieren.

\vfill
\begin{flushright}
Verabschiedet am 08.05.2016 in Konstanz
\end{flushright}




\end{document}


