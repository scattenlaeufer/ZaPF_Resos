\documentclass[DIV=calc]{scrartcl}
\usepackage[utf8]{inputenc}
\usepackage[T1]{fontenc}
\usepackage[ngerman]{babel}
\usepackage{graphicx}

\usepackage{fixltx2e}
\usepackage{ellipsis}
\usepackage[tracking=true]{microtype}

\usepackage{lmodern}                        % Ersatz fuer Computer Modern-Schriften
\usepackage{hfoldsty}

\usepackage{fourier} 			% Schriftart
\usepackage[scaled=0.81]{helvet} 	% Schriftart

\usepackage{url}
\usepackage{tocloft} 			% Paket für Table of Contents

\usepackage{xcolor}
\definecolor{urlred}{HTML}{660000}

\usepackage{hyperref}
\hypersetup{
  colorlinks=true,	
  linkcolor=black,	% Farbe der internen Links (u.a. Table of Contents)
  urlcolor=black,	% Farbe der url-links
  citecolor=black} % Farbe der Literaturverzeichnis-Links

\usepackage{mdwlist} 	% Änderung der Zeilenabstände bei itemize und enumerate

\parindent 0pt 				% Absatzeinrücken verhindern
\parskip 12pt 				% Absätze durch Lücke trennen

\usepackage{titlesec}	% Abstand nach Überschriften neu definieren
\titlespacing{\subsection}{0ex}{3ex}{-1ex}
\titlespacing{\subsubsection}{0ex}{3ex}{-1ex}		

% \pagestyle{empty}
\setlength{\textheight}{23cm}
\usepackage{fancyhdr}
\pagestyle{fancy}
\cfoot{}
\lfoot{Zusammenkunft aller Physik-Fachschaften}
\rfoot{www.zapfev.de\\stapf@googlegroups.de}
\renewcommand{\headrulewidth}{0pt}
\renewcommand{\footrulewidth}{0.1pt}


\begin{document}
\hspace{0.87\textwidth}
\begin{minipage}{120pt}
\vspace{-1.8cm}
\includegraphics[width=80pt]{logo.pdf}
\centering
\small Zusammenkunft aller Physik-Fachschaften
\end{minipage}
\begin{center}
\huge{Resolution der Zusammenkunft aller Physik-Fachschaften} \\
\normalsize
\end{center}

\vspace{1cm}
\section*{Resolution zur Exzellenzinitiative III}

Die ZaPF befürwortet Maßnahmen zur Verbesserung der Lehre, den Ausbau demokratischer Strukturen der Hochschulen und eine auskömmliche Grundfinanzierung als Eckpunkte für eine positive Hochschulentwicklung.

Die ersten zwei Runden der Exzellenzinitiative waren dem nicht zuträglich, wie die Evaluation (Imboden-Bericht\footnote{http://www.gwk-bonn.de/fileadmin/Papers/Imboden-Bericht-2016.pdf}) gezeigt hat. Die Vorschläge der gemeinsamen Wissenschaftskonferenz zur dritten Runde\footnote{http://www.gwk-bonn.de/fileadmin/Pressemitteilungen/pm2016-04.pdf} korrigieren dies nur unwesentlich:

\begin{itemize}
\item Die Exzellenzinitiative gefährdet die Einheit von Forschung und Lehre. Sie ist vollständig forschungsorientiert, die Qualität der Lehre tritt in den Hintergrund.
\item Die Exzellenzinitiative negiert Forschungskooperationen, weil sie die Wissenschaftler*innen verschiedener Hochschulen dazu anhält, einander als Konkurrenz oder höchstens als taktische Partner*innen zu betrachten, statt zu kooperieren und gewonnene Einsichten zu teilen, um darauf gemeinsam für weiteren Erkenntnisfortschritt aufbauen zu können. Auch innerhalb der Hochschulen wird das Konkurrenzdenken gefördert. Entscheidungen darüber, welche Forschungsvorhaben finanziert werden, werden von demokratisch legitimierten (Hochschul-) Gremien in intransparente, subjektive Begutachtungsverfahren verschoben.
\item Die Exzellenzinitiative steht einer flächendeckenden Ausfinanzierung der Hochschulen entgegen.
\end{itemize}

Mit Blick auf die Vorschläge der GWK\footnotemark[2] fordern wir die Fachschaften dazu auf, sich aktiv in die Debatten über die Bewerbung ihrer Hochschulen einzubringen.


\vfill
\begin{flushright}
Verabschiedet am 08.05.2016 in Konstanz
\end{flushright}




\end{document}

